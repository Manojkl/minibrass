 Many (perhaps most) industrial combinatorial optimization problems occurring in practice
tend to be over-constrained, according to \cite{van2011over}. Usually, this fact leads to several
refinements of an initial constraint model by manually weakening or dropping constraints until a 
solution can be found. The situation is more severe if the concrete 
instance, i.e., all parameters to a problem are known only at runtime when a
system has to make autonomous decisions. Simply failing with \texttt{unsatisfiable} is
not an option, a compromise solution is necessary.

The problem has been recognized for many years (see \cref{sec:related-work}) but still lacks 
ready-to-use implementations for modern constraint platforms. Most often, softening constraints is achieved by encoding violations in one
or more cost variables that need to be minimized, leading to the framework of weighted CSP.
Assigning weights is comparatively simple, expressive, and enables techniques such 
as soft global constraints~\cite{van2011over} or soft arc consistency~\cite{cooper2004arc}\todo{better use
sac-revisited}. However, there are certain drawbacks of expressing graded satisfaction only
in terms of weights.


\begin{itemize}
\item Why do we really want a graph of constraints?
\begin{itemize}
\item Numbers are hard to pick
\item Often, there is more than one cost value and an aggregation has to be performed, most
likely using a weighted sum of cost values. When ``emulating'' a lexicographic ordering over these objectives, 
one has to pick sufficiently large weights to give precedence to a more important objective.
Not only may this cause overflows and other numerical issues\todo{cite minisearch} but also lead to
rather weak propagation of bounds in CP-solvers\todo{cite Pierre Schaus LNS}. Moreover, choosing suitable
weights is hard if the objectives' domains do not show natural bounds but depend on choices made for 
other decision variables.

\item Clear and well-defined semantics such as lexicographic ordering.

\end{itemize}
\end{itemize}
Constraint relationships have first been introduced in combination with a translation to weighted CSP in \cite{Schiendorfer13}.
Theoretical concerns, such as the relationship to algebraic frameworks and hierarchical constraints have been addressed 
in \cite{knapp-schiendorfer2014ictai} and \cite{SchiendorferPvs2015}, respectively. This paper, by contrast, 
aims to present the formalism and its relations to related concept in a unified way as well as to provide 
a reusable implementation using MiniZinc/MiniSearch. 
