%%%%%%%%%%%%%%%%%%%%%%% file template.tex %%%%%%%%%%%%%%%%%%%%%%%%%
%
% This is a general template file for the LaTeX package SVJour3
% for Springer journals.          Springer Heidelberg 2010/09/16
%
% Copy it to a new file with a new name and use it as the basis
% for your article. Delete % signs as needed.
%
% This template includes a few options for different layouts and
% content for various journals. Please consult a previous issue of
% your journal as needed.
%
%%%%%%%%%%%%%%%%%%%%%%%%%%%%%%%%%%%%%%%%%%%%%%%%%%%%%%%%%%%%%%%%%%%
%


%
\RequirePackage{fix-cm}
%
%\documentclass{svjour3}                     % onecolumn (standard format)
\documentclass[smallcondensed]{svjour3}     % onecolumn (ditto)
%\documentclass[smallextended]{svjour3}       % onecolumn (second format)
%\documentclass[twocolumn]{svjour3}          % twocolumn
%
\smartqed  % flush right qed marks, e.g. at end of proof
%
\usepackage[colorinlistoftodos,prependcaption]{todonotes}

% TODOs and Notes
\newcommand{\todobox}[1]{\todo[inline]{#1}}                         % shorter version of "\todo[inline]" command
\newcommand{\todot}[1]{\sethlcolor{yellow} \hl{\textbf{TODO:} #1}}  % yellow TODO in text
\newcommand{\noteb}[1]{\sethlcolor{cyan} \hl{\textbf{NOTE:} #1}}    % blue Note in text
\newcommand{\noteg}[1]{\sethlcolor{green} \hl{\textbf{NOTE:} #1}}   % green Note in text
\newcommand{\removet}[1]{\sethlcolor{red} \hl{\textbf{remove?:} #1}} % green Note in text
\newcommand{\remove}[2][1=]{\todo[linecolor=blue,backgroundcolor=blue!25,bordercolor=blue,#1]{\textbf{Remove?}\newline\noindent#2}}

\usepackage[utf8]{inputenc}
\usepackage{graphicx}
\usepackage{stmaryrd,amsmath,amssymb}
\usepackage{listings}
\usepackage{xcolor}
\usepackage{times}
%\usepackage{courier}
\lstdefinelanguage{MiniZinc}
{
	morekeywords={array,let,tuple,diff,union,where,exists,set,min,max,float,int,var,predicate,minimize,sum,in,forall,include,to,card},
	sensitive=false,
	morecomment=[l]{\%},
	morecomment=[s]{/*}{*/},
	morestring=[b]",
}

\definecolor{lightlightgray}{gray}{0.99}
\definecolor{forestgreen}{HTML}{009B55}
\definecolor{thermicred}{rgb}{0.82, 0.1, 0.26}

\lstset
{
	basicstyle={\ttfamily\fontsize{7pt}{8pt}\selectfont},
	commentstyle=\ttfamily\color{forestgreen},
	stringstyle=\ttfamily\color{thermicred},
	keywordstyle=\ttfamily\bfseries\color{blue},
	tabsize=2,
	showstringspaces=false,
	stepnumber=5,
	numberfirstline=false,
	%flexiblecolumns=true,
	firstnumber=1,
	numbers=left,
	captionpos=b,	
	backgroundcolor=\color{lightlightgray},
	frame=single,
	%xleftmargin=\parindent,
	numberstyle={\ttfamily\fontsize{6pt}{6pt}\selectfont}
}

\lstset{language=MiniZinc}
%
\usepackage{mathptmx}      % use Times fonts if available on your TeX system
%
% insert here the call for the packages your document requires
%\usepackage{latexsym}
% etc.
%
% please place your own definitions here and don't use \def but
% \newcommand{}{}
%
% Insert the name of "your journal" with
% \journalname{myjournal}
%
\begin{document}

\title{Qualitative Soft Constraints\thanks{This research is partly sponsored by the 
German Research Foundation (DFG) in the project ``OC-Trust'' (FOR 1085).}
}
%\subtitle{Do you have a subtitle?\\ If so, write it here}

%\titlerunning{Short form of title}        % if too long for running head

\author{Alexander Schiendorfer \and Alexander Knapp \and Wolfgang Reif %etc.
}

%\authorrunning{Short form of author list} % if too long for running head

\institute{Alexander Schiendorfer \at
              University of Augsburg\\
              \email{schiendorfer@isse.de}           %  \\
%             \emph{Present address:} of F. Author  %  if needed
           \and
           Alexander Knapp \at
           University of Augsburg\\
              \email{knapp@isse.de}   
           \and
           Wolfgang Reif \at
           University of Augsburg\\
              \email{reif@isse.de}  
}

\date{Received: date / Accepted: date}
% The correct dates will be entered by the editor


\maketitle

\begin{abstract}
Insert your abstract here. Include keywords, PACS and mathematical
subject classification numbers as needed.
\keywords{First keyword \and Second keyword \and More}
% \PACS{PACS code1 \and PACS code2 \and more}
% \subclass{MSC code1 \and MSC code2 \and more}
\end{abstract}

\section{Introduction}
\label{intro}
Your text comes here. Separate text sections with
We started working on \cite{Schiendorfer13}

\section{Related Work}
\label{sec:related-work}
Pioneering attempts to generalize the rigid constraint formalism
were offered by the formalism of \emph{partial constraint 
satisfaction}~\cite{FreuderW92}. The core idea was to define
a metric that measures the distance $d$ of an assignment $\theta$
to the solution space of the original problem
with $d=0$ indicating solutions. Proposed
choices included the number of domain items to be added to make 
$\theta$ feasible or the number of \emph{violated constraints} of 
$\theta$. The latter is now better known as \emph{Max-CSP}.

\emph{Valued constraints} took on that idea to label constraints
with \emph{values} from a totally ordered set that can be seen as 
penalties for violating constraints.

Similarly, \emph{semiring-based} soft constraint frameworks
labels each assignment with a satisfaction degree of a so-called 
\emph{c-semiring}, i.e., an algebraic structure composed of a 
multiplication operator for \emph{combining} satisfaction degrees of 
several soft constraints as well as an addition operator (i.e., a supremum) that induces a partial order for ranking solutions.

\begin{itemize}
\item \todobox{T. Petit, J.C. Régin, and C. Bessière. Meta-constraints on violations for over constrained problems.} \cite{petit2000meta} introduces reified variables for cost values, we rely
on that technique
\item \todobox{Conflict-Directed A* Search for Soft Constraints} \cite{sachenbacher2006conflict} based on reified variables, uses propagation in combination with A star, i.e., in principle a best first search where propagation of reified variables may lower the objective (if one satisfied soft constraint propagates the violation of many others, try another assignment first). Not a classical branch and bound approach since
the node with the highest promising objective value is branched on. Requires quite some manual bookkeeping. Not clear how much benefit this brings over conventional search using variable/value heuristics.  
\item \cite{van2011over} soft global constraints
\end{itemize}


\section{Foundations}

\section{Implementation}

\renewcommand{\lstlistingname}{Code}
\lstinputlisting{../../source-code/minizinc/spd_better.mzn}
\section{Evaluation}

\section{Conclusion}

%\begin{acknowledgements}
%If you'd like to thank anyone, place your comments here
%and remove the percent signs.
%\end{acknowledgements}

% BibTeX users please use one of
%\bibliographystyle{spbasic}      % basic style, author-year citations
\bibliographystyle{spmpsci}      % mathematics and physical sciences
%\bibliographystyle{spphys}       % APS-like style for physics
\bibliography{constraint-preferences}   % name your BibTeX data base

\end{document}
% end of file template.tex

