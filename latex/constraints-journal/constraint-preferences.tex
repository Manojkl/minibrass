%%%%%%%%%%%%%%%%%%%%%%% file template.tex %%%%%%%%%%%%%%%%%%%%%%%%%
%
% This is a general template file for the LaTeX package SVJour3
% for Springer journals.          Springer Heidelberg 2010/09/16
%
% Copy it to a new file with a new name and use it as the basis
% for your article. Delete % signs as needed.
%
% This template includes a few options for different layouts and
% content for various journals. Please consult a previous issue of
% your journal as needed.
%
%%%%%%%%%%%%%%%%%%%%%%%%%%%%%%%%%%%%%%%%%%%%%%%%%%%%%%%%%%%%%%%%%%%
%


%
\RequirePackage{fix-cm}

% unfortunately, we need this hack for cleref with svjour3
\makeatletter
\def\cl@chapter{}
\makeatother

%
%\documentclass{svjour3}                     % onecolumn (standard format)
\documentclass[smallcondensed]{svjour3}     % onecolumn (ditto)
%\documentclass[smallextended]{svjour3}       % onecolumn (second format)
%\documentclass[twocolumn]{svjour3}          % twocolumn
%
\smartqed  % flush right qed marks, e.g. at end of proof
%

\usepackage[colorinlistoftodos,prependcaption]{todonotes}

% TODOs and Notes
\newcommand{\todobox}[1]{\todo[inline]{#1}}                         % shorter version of "\todo[inline]" command
\newcommand{\todot}[1]{\sethlcolor{yellow} \hl{\textbf{TODO:} #1}}  % yellow TODO in text
\newcommand{\noteb}[1]{\sethlcolor{cyan} \hl{\textbf{NOTE:} #1}}    % blue Note in text
\newcommand{\noteg}[1]{\sethlcolor{green} \hl{\textbf{NOTE:} #1}}   % green Note in text
\newcommand{\removet}[1]{\sethlcolor{red} \hl{\textbf{remove?:} #1}} % green Note in text
\newcommand{\remove}[2][1=]{\todo[linecolor=blue,backgroundcolor=blue!25,bordercolor=blue,#1]{\textbf{Remove?}\newline\noindent#2}}
    % for useful commands
% a sequence (list), e.g., values [1, 2, 3]
\providecommand{\seq}[1]{\langle #1 \rangle}

% a concrete variable (in an example)
\providecommand{\concVar}[1]{\mathtt{#1}}

% just shorthand for typing set braces
\providecommand{\set}[1]{\{ #1 \}}

% --------------------------------------------------
% Elementary definitions
\providecommand{\Vars}{X}
% a generic variable for iterating etc
\providecommand{\GenVar}{x}
\providecommand{\Dom}{D}
\providecommand{\Cons}{C}

% problem identifier
\providecommand{\Prob}{\mathit{CP}}

% a generic constraint
\providecommand{\GenCons}{c}

% macro for the search space
\providecommand{\SSpace}{[\Vars \to \Dom]}

% macro for the set of boolean values
\providecommand{\Bool}{\mathbb{B}}

% a generic objective function
\providecommand{\GenObj}{f}

% --------------------------------------------------

% a generic element of a meet monoid / PVS / c-semiring describing satisfaction, i.e., the "solution degree"
\providecommand{\GenSolDegree}{m}

% a generic variable assignment, i.e., an element taken from [X \to D]
\providecommand{\GenAssignment}{\theta}

% the solution operator giving the set of assignments that satisfy all hard constraints of a problem P
\providecommand{\solns}{\mathrm{sol}}

% the operator mapping assignments to their solution degrees
\providecommand{\val}{\mathrm{val}}    % should be rather exchangeable

\usepackage[utf8]{inputenc}
\usepackage{graphicx}
\usepackage{stmaryrd,amsmath,amssymb}
\usepackage{listings}
\usepackage{xcolor}
\usepackage[capitalise,noabbrev]{cleveref}
\usepackage{times}
%\usepackage{courier}
\lstdefinelanguage{MiniZinc}
{
	morekeywords={array,solve,constraint,let,tuple,diff,function,repeat,union,where,exists,set,min,max,float,int,var,predicate,minimize,sum,in,forall,include,to,card,if,then,else,endif,print,search}, 
	sensitive=false,
	morecomment=[l]{\%},
	morecomment=[s]{/*}{*/},
	morestring=[b]",
}

\definecolor{lightlightgray}{gray}{0.99}
\definecolor{forestgreen}{HTML}{009B55}
\definecolor{thermicred}{rgb}{0.82, 0.1, 0.26}

\lstset
{
	basicstyle={\ttfamily\fontsize{7pt}{8pt}\selectfont},
	commentstyle=\ttfamily\color{forestgreen},
	stringstyle=\ttfamily\color{thermicred},
	keywordstyle=\ttfamily\bfseries\color{blue},
	tabsize=2,
	showstringspaces=false,
	stepnumber=5,
	numberfirstline=false,
	%flexiblecolumns=true,
	firstnumber=1,
	numbers=left,
	captionpos=b,	
	backgroundcolor=\color{lightlightgray},
	frame=single,
	%xleftmargin=\parindent,
	numberstyle={\ttfamily\fontsize{6pt}{6pt}\selectfont}
}

\lstset{language=MiniZinc}
%
\usepackage{mathptmx}      % use Times fonts if available on your TeX system
%
% insert here the call for the packages your document requires
%\usepackage{latexsym}
% etc.
%
% please place your own definitions here and don't use \def but
% \newcommand{}{}
%
% Insert the name of "your journal" with
% \journalname{myjournal}
%
\begin{document}

\title{Qualitative Soft Constraints\thanks{This research is partly sponsored by the 
German Research Foundation (DFG) in the project ``OC-Trust'' (FOR 1085).}
}
%\subtitle{Do you have a subtitle?\\ If so, write it here}

%\titlerunning{Short form of title}        % if too long for running head

\author{Alexander Schiendorfer \and Alexander Knapp \and Wolfgang Reif %etc.
}

%\authorrunning{Short form of author list} % if too long for running head

\institute{Alexander Schiendorfer \at
              University of Augsburg\\
              \email{schiendorfer@isse.de}           %  \\
%             \emph{Present address:} of F. Author  %  if needed
           \and
           Alexander Knapp \at
           University of Augsburg\\
              \email{knapp@isse.de}   
           \and
           Wolfgang Reif \at
           University of Augsburg\\
              \email{reif@isse.de}  
}

\date{Received: date / Accepted: date}
% The correct dates will be entered by the editor


\maketitle

% This is not the actual abstract (much too boring)
% but should serve as a rough outline
\begin{abstract}
Over-constrained problems are ubiquitous in industrial and other real-world problems.
Plenty of formalisms have been proposed and abstracted in algebraic structures.
Specialized areas diverged such as cost function networks or semiring-based soft constraints.
Integration with conventional constraint problems are implemented using soft global constraints.
Most of them involve a cost variable / cost functions. This is not good, if costs can be defined
by different agents/modelers that lack a common currency idea. We propose a formalism that is purely 
qualitative, relies on the satisfaction/dissatisfaction of soft constraints. Preferences are written
as a directed acyclic graph. We show how to proceed from this graph by suggesting set orderings over
sets of violated constraints, prove that these orderings are mathematically enforced by common axioms
of monoidal soft constraints, discuss the relationship to other formalisms in terms of expressiveness,
implement various optimization strategies using MiniZinc/MiniSearch, and provide experimental results.

\keywords{Soft Constraints \and Over-constrained Problems \and Modeling Formalisms \and MiniZinc}
\end{abstract}
\todo[inline]{Argument: There are plenty bespoke algorithms for weighted CSPs but
there is little support for higher-level modeling languages such as MiniZinc, Essence
or OPL.}

\todo[inline]{Many articles discuss how to adequately model preferences and
the comparison of such formalisms. But support for common modeling languages
exploiting features of conventional constraint solvers are rare.}

\section{Introduction}
\label{sec:introduction}
 Many (perhaps most) industrial combinatorial optimization problems occurring in practice
tend to be over-constrained, according to \cite{van2011over}. Usually, this fact leads to several
refinements of an initial constraint model by manually weakening or dropping constraints until a 
solution can be found. The situation is more severe if the concrete 
instance, i.e., all parameters to a problem are known only at runtime when a
system has to make autonomous decisions. Simply failing with \texttt{unsatisfiable} is
not an option, a compromise solution is necessary.

The problem has been recognized for many years (see \cref{sec:related-work}) but still lacks 
ready-to-use implementations for modern constraint platforms. Most often, softening constraints is achieved by encoding violations in one
or more cost variables that need to be minimized, leading to the framework of weighted CSP.
Assigning weights is comparatively simple, expressive, and enables techniques such 
as soft global constraints~\cite{van2011over} or soft arc consistency~\cite{cooper2004arc}\todo{better use
sac-revisited}. However, there are certain drawbacks of expressing graded satisfaction only
in terms of weights.


\begin{itemize}
\item Why do we really want a graph of constraints?
\begin{itemize}
\item Numbers are hard to pick
\item Often, there is more than one cost value and an aggregation has to be performed, most
likely using a weighted sum of cost values. When ``emulating'' a lexicographic ordering over these objectives, 
one has to pick sufficiently large weights to give precedence to a more important objective.
Not only may this cause overflows and other numerical issues\todo{cite minisearch} but also lead to
rather weak propagation of bounds in CP-solvers\todo{cite Pierre Schaus LNS}. Moreover, choosing suitable
weights is hard if the objectives' domains do not show natural bounds but depend on choices made for 
other decision variables.

\item Clear and well-defined semantics such as lexicographic ordering.

\end{itemize}
\end{itemize}
Constraint relationships have first been introduced in combination with a translation to weighted CSP in \cite{Schiendorfer13}.
Theoretical concerns, such as the relationship to algebraic frameworks and hierarchical constraints have been addressed 
in \cite{knapp-schiendorfer2014ictai} and \cite{SchiendorferPvs2015}, respectively. This paper, by contrast, 
aims to present the formalism and its relations to related concept in a unified way as well as to provide 
a reusable implementation using MiniZinc/MiniSearch. 


\section{Related Work}
\label{sec:related-work}
Pioneering attempts to generalize the rigid constraint formalism
were offered by the formalism of \emph{partial constraint 
satisfaction}~\cite{FreuderW92}. The core idea was to define
a metric that measures the distance $d$ of an assignment $\theta$
to the solution space of the original problem
with $d=0$ indicating solutions. Proposed
choices included the number of domain items to be added to make 
$\theta$ feasible or the number of \emph{violated constraints} of 
$\theta$. The latter is now better known as \emph{Max-CSP}.

\emph{Valued constraints} took on that idea to label constraints
with \emph{values} from a totally ordered set that can be seen as 
penalties for violating constraints.

Similarly, \emph{semiring-based} soft constraint frameworks
labels each assignment with a satisfaction degree of a so-called 
\emph{c-semiring}, i.e., an algebraic structure composed of a 
multiplication operator for \emph{combining} satisfaction degrees of 
several soft constraints as well as an addition operator (i.e., a supremum) that induces a partial order for ranking solutions.

\begin{itemize}
\item \todobox{T. Petit, J.C. Régin, and C. Bessière. Meta-constraints on violations for over constrained problems.} \cite{petit2000meta} introduces reified variables for cost values, we rely
on that technique
\item \todobox{Conflict-Directed A* Search for Soft Constraints} \cite{sachenbacher2006conflict} based on reified variables, uses propagation in combination with A star, i.e., in principle a best first search where propagation of reified variables may lower the objective (if one satisfied soft constraint propagates the violation of many others, try another assignment first). Not a classical branch and bound approach since
the node with the highest promising objective value is branched on. Requires quite some manual bookkeeping. Not clear how much benefit this brings over conventional search using variable/value heuristics.  
\item \cite{van2011over} soft global constraints
\end{itemize}


\section{Foundations}
\label{sec:foundations}
As usual, a constraint problem $\Prob = (\Vars, \Dom, \Cons)$ is described
by a set of decision variables $\Vars$, their associated family of domains of possible values
$\Dom = (\Dom_\GenVar)_{\GenVar \in \Vars}$, and a set of constraints $C$ that restrict valid assignments.
An assignment $\GenAssignment$ is a mapping from $\Vars$ to $\Dom$, written as $\GenAssignment \in \SSpace$, such that each variable $\GenVar$ maps to
a value in $\Dom_\GenVar$. A constraint $\GenCons \in \Cons$ is understood as a map $\GenCons : \SSpace \to \Bool$
where we write $\GenAssignment \models \GenCons$ for $\GenCons(\GenAssignment) = \mathit{true}$. We call
an assignment $\GenAssignment$ a solution if $\GenAssignment \models \GenCons$ holds for all $\GenCons \in \Cons$ and
write the set of all solutions of a constraint problem $\Prob$ as $\solns(\Prob)$. If the main task of $\Prob$ is
to find a solution, we call it a \emph{constraint satisfaction problem}~(CSP).

We obtain \emph{constraint optimization problems}~(COP) by adding an objective function $\GenObj : \SSpace \to P$
where $(P, \leq_P)$ is a partial order, i.e., $\leq_P$ is a reflexive, antisymmetric, and transitive relation over $P$.
Elements of $P$ are interpreted as \emph{solution degrees}, denoting quality. Without loss of generality, we interpret
$m <_P n$ as $m$ being inferior to $n$ and restrict our attention to maximization problems.
Hence, a solution degree $\GenSolDegree$ is optimal with respect to a constraint problem $\Prob$, 
if for all solutions $\GenAssignment \in \solns(\Prob)$ it holds either that $\GenObj(\GenAssignment) \leq_P \GenSolDegree$ or 
$\GenObj(\GenAssignment) \parallel_P \GenSolDegree$, expressing incomparability.  It is \emph{reachable} if there is a solution $\GenAssignment \in \solns(\Prob)$ such that 
$\GenObj(\GenAssignment) = \GenSolDegree$. 

\begin{itemize}
\item In a PVS, e.g., $\varepsilon_M$ is trivially optimal.
\item Non-reachable optimal solution degrees emerge, e.g., as
 upper bounds or from the supremum operator in c-semirings. 


\end{itemize}
% -----------------------
\section{Implementation}
\label{sec:implementation}
\lstinputlisting{../../source-code/minizinc/spd_better.mzn}

\lstinputlisting{../../source-code/minizinc/tpd_better.mzn}

Some points that should be emphasized 

\begin{enumerate}
\item Variable ordering can be used to assign reified variables first.
\item Order them decreasingly by importance (can be done using data manipulation in MiniZinc)
\item Assign \texttt{true} first, hoping to find a solution to all soft constraints.
\item Discuss/prove that this search tree, in a static ordering, assures that violation degrees
are tried in TPD-order.
\item SPD-order can probably achieved by using a restarts strategy $\rightarrow$ use new scope 
every time, add only those soft constraints that are supposed to hold.
\item Discuss similarities/differences to conflict-directed A* search.
\item \emph{Quite a nifty point:} in addition to posting the constraint 
\texttt{xpd\_better(lb, violatedScs)}, meaning that the set of violated soft constraints should
be better in the next solution, we can post something like \texttt{penalty(lb) > penalty(violatedScs)}
\emph{as well}! Why does that make sense?
\begin{itemize}
\item \emph{If} we find a solution $\theta_2$ that is XPD-better than $\theta_1$, then it is also penalty-better (by implication).
\item \texttt{violatedScs} is a bounded set variable (at least below by $\emptyset$, above by \texttt{soft-constraints}).
From these bounds, the solver can immediately derive bounds for \texttt{penalty(violatedScs)}. Propagation of reified variables
during search (one satisfied constraint implying the violation of another etc.) is then properly handled. Search can stop,
once the best case minimal penalty violation cannot go below the imposed upper bound for violation.
\item This constraint is \emph{redundant}, i.e., if for any solution $\theta_2$ XPD-better $\theta_1$ holds, penalty better holds
as well.
\item But we do not have a \emph{propagator} for XPD-better. The set variable \texttt{violatedScs} is bounded below by the number
of minimally (definitely, already) violated soft constraints, above by the maximally possible violations (cmp. $\alpha$, $\zeta$).
We could, in principle, build a simple bounds propagator for XPD-better that restricts the domain of  \texttt{violatedScs}
to values strictly above the last found lower bound.
\item It's unreasonable to assume a default behavior like bounds propagation on a user-defined predicate such as XPD-better.
It could be that the predicate is only true ``somewhere'' in the middle between lower and upper set-bound. Then bounds propagation
would incorrectly cut partial assignments.
\item But in lieu of a dedicated XPD-propagator, we can benefit from the redundant penalty constraint. For instance, suppose 
we have seen a solution violating $\{c_2,c_3\}$, with penalty $2$. Assume we are in a search tree and the lower bound for our \texttt{violatedScs}
is $\{c_1\}$ violated constraints, and the upper bound be $\{c_1,c_2,c_3\}$. Thus, the penalty is at least $3$, at most $5$. We can cut the
search since we posted \texttt{penalty(violatedScs) < 2}.
\item It turns out that propagation is smarter than I thought \ldots since the witness is just a bunch of additional variables we get much more
propagation than initially thought, which is nice.
\item The penalty trick still cuts down the number of failures.
\end{itemize}
\item Global constraints aren't as easily reified. Probably in the future \todo{look for reification of global constraints paper}. Better use a restarts approach for now.
\item It turns out that, since XDP-better is formulated as a MiniZinc predicate in terms of a conjunction of other (global) constraints, we inherit some propagation capability.
\end{enumerate}

\subsection{Search}

The first search strategy corresponds to classical branch-and-bound~(BAB) search
in propagation engines. For every found solution, a constraint is imposed that 
the next solution has to be strictly better.

\begin{lstlisting}
function ann: strictlyBetterBAB(var set of SOFTCONSTRAINTS: violatedScs) =
       repeat(
           if next() then
               let {
                 set of SOFTCONSTRAINTS: lb = sol(violatedScs); 
               } in (
                 print("Intermediate solution:") /\ print() /\
                 commit() /\ post(spd_better(lb, violatedScs, 
                                  SOFTCONSTRAINTS, edges))
               )
           else break endif
       );
       
[...]
       
search strictlyBetterBAB_TPD(violatedScs) /\ 
   if hasSol() then print("Final solution: ") /\ print() 
   else print("No solution found\n") endif;
\end{lstlisting}

While this procedure yields optimal solutions, it is not ideal for partially ordered objectives since 
another optimum need not be better than the current solution. Instead, we need to impose that any coming 
solution \emph{must not be dominated} by \emph{any} solution seen so far. Technically,
we have a set of lower bounds (the satisfaction degrees of previous solutions) $L = \{l_1, \ldots, l_m \}$ and
require that it must not be the case that $\exists l \in L : \mathrm{obj} \leq_M l$ for any partial valuation
structure $M$. The next solution must either be strictly better than any one of the maxima of $L$ or incomparable to
all of them. We would also like to remove lower bounds from $L$ when we find stricter ones (as one would do with
totally ordered objective spaces) but this is not as easy due to partiality (which constraint was to blame/should
stop being imposed).

\begin{lstlisting}
function ann: onlyNotDominated(var set of SOFTCONSTRAINTS: violatedScs) =
       repeat(
           if next() then
               let {
                 set of SOFTCONSTRAINTS: lb = sol(violatedScs); 
               } in (
                 print("Intermediate solution:") /\ print() /\
                 commit() /\ post(not (spd_better(violatedScs, lb, 
                                         SOFTCONSTRAINTS, edges) 
                                         \/ violatedScs = lb ) )
               )
           else break endif
       );
\end{lstlisting}

The difference is best explained by an example. Consider the following oversimplified constraint model.
\begin{lstlisting}
var 1..3: x; 

constraint x = 1 <-> violated[1];
constraint x = 2 <-> violated[2];
constraint x = 3 <-> violated[3];

solve 
:: int_search([x], input_order, indomain_max, complete)
search strictlyBetterBAB_TPD(violatedScs);
\end{lstlisting}
We explore $\concVar{x}$ in a decreasing order. This 
results in the sequence $\seq{ \{3\}, \{2\}, \{1\} }$ of 
satisfaction degrees. $\set{3}$ and $\set{2}$ both 
dominate $\set{1}$ but are incomparable; $\set{1} <_M \set{3}$,
$\set{1} <_M \set{2}$ but $\set{2} \parallel_M \set{3}$. The reachable optima
of this problem are clearly $\set{ \set{2}, \set{3} }$
\begin{verbatim}
Intermediate solution:Obj: 1 by violating {3..3 } : x -> 3
----------
==========
\end{verbatim}
Each assignment to $\concVar{x}$ violates precisely one soft constraint. 
% -----------------------

\section{Evaluation}
\label{sec:evaluation}
\subsection{Tailoring MiniZinc to Accomodate Softness}
Here we'll have a short introduction into how we actually 
modeled our benchmark problems. Includes how we added several
notions of constraints, different levels (few: 3-5, medium: 5-10, many: 10+)

What are the aspects that need motivation?
\begin{enumerate}
\item Enhanced propagation by redundant constraints
\item Search heuristics
\item Comparing BaB with LNS (depending on MiniSearch's progress)
\end{enumerate}

\begin{enumerate}
\item \textbf{Factors}: type of heuristics / variable ordering, enhanced propagation
\item \textbf{Dependent Variables}: runtime / failures / nb. problems solved / optimal value after 5 minutes
\end{enumerate}

\subsection{Overhead from Softness}

Here, we'll compare the plain MiniZinc problems with softened but more constrained problems.
Central question: How much harder / slower will it be to find (optimal) solutions?
We shall, of course, present the altered models online! Including some documentation about
how we modified the original problems.

\begin{enumerate}
\item Measure if a problem that was easy in the beginning is still easy
\item Measure if hard problems get tremendously harder (play a bit with search heuristics)
\end{enumerate}

\subsection{Comparison with Related Work}

Here, we'll demonstrate that we do not really have a competing product, since no solver
truly implements algebraic structures for soft constraints. So nothing is
really offered in a qualitative setting.

But in the quantitative realm (with all its disadvantages), there are two competing solvers that 
offer bespoke solutions for weighted CSP. Even though we do not want to really compete in
terms of runtime, we hope to see the differences

\begin{enumerate}
\item WSimply 
\item Toulbar2, perhaps interfaced by Numberjack
\end{enumerate}

\todo[inline]{Figure out workflow of WSimply with MiniZinc models -- that'd be ideal!}
\todo[inline]{Try a model with Numberjack}
\todo[inline]{Find a cool scheduling/packing problem involving lots of global constraints 
and make a translation to WCSP/Numberjack Format for toulbar2}



\section{Conclusion}

%\begin{acknowledgements}
%If you'd like to thank anyone, place your comments here
%and remove the percent signs.
%\end{acknowledgements}

% BibTeX users please use one of
%\bibliographystyle{spbasic}      % basic style, author-year citations
\bibliographystyle{spmpsci}      % mathematics and physical sciences
%\bibliographystyle{spphys}       % APS-like style for physics
\bibliography{constraint-preferences}   % name your BibTeX data base

\end{document}
% end of file template.tex

