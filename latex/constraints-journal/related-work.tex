Pioneering attempts to generalize the rigid constraint formalism
were offered by the formalism of \emph{partial constraint 
satisfaction}~\cite{FreuderW92}. The core idea was to define
a metric that measures the distance $d$ of an assignment $\theta$
to the solution space of the original problem
with $d=0$ indicating solutions. Proposed
choices included the number of domain items to be added to make 
$\theta$ feasible or the number of \emph{violated constraints} of 
$\theta$. The latter is now better known as \emph{Max-CSP}.

\emph{Valued constraints} took on that idea to label constraints
with \emph{values} from a totally ordered set that can be seen as 
penalties for violating constraints.

Similarly, \emph{semiring-based} soft constraint frameworks
labels each assignment with a satisfaction degree of a so-called 
\emph{c-semiring}, i.e., an algebraic structure composed of a 
multiplication operator for \emph{combining} satisfaction degrees of 
several soft constraints as well as an addition operator (i.e., a supremum) that induces a partial order for ranking solutions.

\begin{itemize}
\item \todobox{T. Petit, J.C. Régin, and C. Bessière. Meta-constraints on violations for over constrained problems.} introduces reified variables for cost values, we rely
on that technique
\item \todobox{Conflict-Directed A* Search for Soft Constraints} based on reified variables, uses propagation in combination with A star, i.e., in principle a best first search where propagation of reified variables may lower the objective (if one satisfied soft constraint propagates the violation of many others, try another assignment first). Not a classical branch and bound approach since
the node with the highest promising objective value is branched on. Requires quite some manual bookkeeping. Not clear how much benefit this brings over conventional search using variable/value heuristics. 
\end{itemize}
