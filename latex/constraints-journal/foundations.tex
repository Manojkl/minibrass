As usual, a constraint problem $\Prob = (\Vars, \Dom, \Cons)$ is described
by a set of decision variables $\Vars$, their associated family of domains of possible values
$\Dom = (\Dom_\GenVar)_{\GenVar \in \Vars}$, and a set of constraints $C$ that restrict valid assignments.
An assignment $\GenAssignment$ is a mapping from $\Vars$ to $\Dom$, written as $\GenAssignment \in \SSpace$, such that each variable $\GenVar$ maps to
a value in $\Dom_\GenVar$. A constraint $\GenCons \in \Cons$ is understood as a map $\GenCons : \SSpace \to \Bool$
where we write $\GenAssignment \models \GenCons$ for $\GenCons(\GenAssignment) = \mathit{true}$. We call
an assignment $\GenAssignment$ a solution if $\GenAssignment \models \GenCons$ holds for all $\GenCons \in \Cons$ and
write the set of all solutions of a constraint problem $\Prob$ as $\solns(\Prob)$. If the main task of $\Prob$ is
to find a solution, we call it a \emph{constraint satisfaction problem}~(CSP).

We obtain \emph{constraint optimization problems}~(COP) by adding an objective function $\GenObj : \SSpace \to P$
where $(P, \leq_P)$ is a partial order, i.e., $\leq_P$ is a reflexive, antisymmetric, and transitive relation over $P$.
Elements of $P$ are interpreted as \emph{solution degrees}, denoting quality. Without loss of generality, we interpret
$m <_P n$ as $m$ being inferior to $n$ and restrict our attention to maximization problems.
Hence, a solution degree $\GenSolDegree$ is optimal with respect to a constraint problem $\Prob$, 
if for all solutions $\GenAssignment \in \solns(\Prob)$ it holds either that $\GenObj(\GenAssignment) \leq_P \GenSolDegree$ or 
$\GenObj(\GenAssignment) \parallel_P \GenSolDegree$, expressing incomparability.  It is \emph{reachable} if there is a solution $\GenAssignment \in \solns(\Prob)$ such that 
$\GenObj(\GenAssignment) = \GenSolDegree$. 

\begin{itemize}
\item In a PVS, e.g., $\varepsilon_M$ is trivially optimal.
\item Non-reachable optimal solution degrees emerge, e.g., as
 upper bounds or from the supremum operator in c-semirings. 
\item Thus, the answer returned from dynamic programming for c-semirings
usually just returns an upper bound that is not necessarily reachable.

\end{itemize}

\subsection{Category Theory}
\todobox{This is just verbatim from Wikipedia, Abstract Algebra, better reformulate}
\begin{itemize}

\item Algebraic structures, with their associated homomorphisms, form mathematical categories. 
Category theory is a powerful formalism for analyzing and comparing different algebraic structures.

\item Mathematical categories are composed of \emph{objects} (e.g,. algebraic structures) and
\emph{arrows} (also called morphisms) between them. Each arrow $f$ admits a domain $A$
and codomain $B$, both being objects, and is written as $f : A \to B$. For each arrow
$f : A \to B$ and $g : B \to C$ there has to be a composite arrow $(g \circ f) : A \to C$.
Arrow composition $\circ$ needs to be associative and for each object $A$, there has to be an identity arrow 
$\idfun{A} : A \to A$ acting as ``neutral element'' with respect to composition, i.e., $\idfun{B} \circ f = f \circ \idfun{A} = f$.
The most straightforward example is given by the category $\setcat$, where objects are sets, arrows are functions, 
composition is function composition, and the identity arrows are just the identity function.
A slightly more elaborate example is given by $\POcat$, the category of partially-ordered sets, that 
has partial orders as objects and partial order homomorphisms (i.e., monotone functions) as arrows.
Note that this definition is proper since monotone functions are closed under function composition, i.e., if $\varphi : |P| \to |Q|$ and 
$\psi : |Q| \to |R|$ are monotone functions, so is $\psi \circ \varphi$.

\item In categorical arguments, it is important to distinguish algebraic structures from their
underlying components, e.g., underlying sets, functions, or relations. Taking
partially-ordered sets as examples, we refer to the structure $P = (|P|, \leq_P)$ as 
\emph{partial order}, to $|P|$ as the \emph{underlying set}, and to the binary, reflexive, antisymmetric, 
and transitive relation $\leq_P {} \subseteq |P| \times |P|$
as \emph{ordering} (relation). All applications dealt with in this paper are examples
of so-called \emph{concrete categories} where objects are sets (perhaps) with additional structures
and arrows are set-theoretic functions satisfying certain axioms (preserving structure as in homomorphisms).

\item More technically, $| \cdot |$ is a an example of a \emph{functor}, i.e., a mapping $F$ between categories
$C$ and $D$ that sends every $C$-object $A$ to a $D$-object $F(A)$ and every $C$-arrow $f : A \to B$ to 
a $D$-arrow $F(f) : F(A) \to F(B)$. In this case, $| \cdot | : \POcat \to \setcat$ is a mapping such that if
$P = (X, \leq_P)$, $|P| = X$ and each PO-homomorphism $\varphi : P \to Q$ is mapped to its underlying function $|\varphi| : |P| \to |Q|$.

\end{itemize}