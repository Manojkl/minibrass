\documentclass[english]{notes}
\usepackage[utf8]{inputenc}

\usepackage{hyperref}
\usepackage{graphicx}
\usepackage{todonotes}
\usepackage{mathrsfs}
\usepackage{changes}
\usetikzlibrary{matrix,shapes,arrows,arrows.meta,calc}

\usepackage[backend=bibtex,
            style=numeric-comp,
            maxnames=999,
            isbn=false, doi=false, eprint=false]{biblatex}

\newcommand{\code}[1]{\normalfont\texttt{\spaceskip=3pt\frenchspacing\def\{{\char123}\def\}{\char125}\def\^{\char94}\def\_{\char95}#1}}
\newcommand{\varit}[1]{{\frenchspacing\ensuremath{\normalfont\textsl{#1}}}}
\newcommand{\macit}[1]{{\frenchspacing\ensuremath{\normalfont\textsf{#1}}}}
\newcommand{\Eta}{\mathrm{H}}
\newcommand{\Mu}{\mathrm{M}}
\newcommand{\Nu}{\mathrm{N}}

\newcommand{\NZ}{\mathbb{N}}
\newcommand{\RZ}{\mathbb{R}}
\newcommand{\RZpos}{\RZ_{> 0}}
\newcommand{\RZp}{\RZ_{\geq 0}}
\newcommand{\powerset}{\mathcal{P}}
\newcommand{\limp}{\mathrel{\Rightarrow}}
\newcommand{\compfun}{\mathbin{\circ}}
\newcommand{\isorel}{\mathrel{\cong}}
\newcommand{\restrict}[2]{{#1}\mathnormal{\upharpoonright}{#2}}
\newcommand{\natto}{\mathrel{\dot{\mathnormal{\to}}}}
\let\lbagold\lbag
\let\rbagold\rbag
\def\lbag{\mathopen{\lbagold}}
\def\rbag{\mathclose{\rbagold}}

\DeclareMathOperator{\Minop}{\mathrm{Min}}
\newcommand{\Min}[1]{\Minop^{#1}}
\DeclareMathOperator{\Maxop}{\mathrm{Max}}
\newcommand{\Max}[1]{\Maxop^{#1}}
\DeclareMathOperator{\finsets}{\mathcal{P}_{\mathrm{fin}}}
\DeclareMathOperator{\nefinsets}{\mathcal{P}_{\mathrm{fin}^+}}
\DeclareMathOperator{\incsets}{\mathcal{I}}
\DeclareMathOperator{\incfinsets}{\incsets_{\mathrm{fin}}}
\newcommand{\lowersubseteq}[1]{\mathrel{\subseteq_{#1}}}
\newcommand{\lowersupseteq}[1]{\mathrel{\supseteq_{#1}}}
\newcommand{\lowersubset}[1]{\mathrel{\subset_{#1}}}
\newcommand{\lowersupset}[1]{\mathrel{\supset_{#1}}}
\newcommand{\uppersubseteq}[1]{\mathrel{\subseteq^{#1}}}
\newcommand{\uppersupseteq}[1]{\mathrel{\supseteq^{#1}}}
\newcommand{\uppersubset}[1]{\mathrel{\subset^{#1}}}
\newcommand{\uppersupset}[1]{\mathrel{\supset^{#1}}}
\newcommand{\lowercup}[1]{\mathbin{\cup_{#1}}}
\newcommand{\uppercup}[1]{\mathbin{\cup^{#1}}}

\DeclareMathOperator{\finmsets}{\mathcal{M}_{\mathrm{fin}}}
\DeclareMathOperator{\nefinmsets}{\mathcal{M}_{\mathrm{fin}^+}}
\newcommand{\mcup}{\mathbin{\mathnormal{\cup}\llap{\text{\fontsize{10pt}{10pt}\selectfont$-$}}}}
\newcommand{\submseteq}{%
\mathrel{\mathchoice%
{\mathnormal{\subseteq}\llap{\text{\raisebox{0.3pt}{\fontsize{10pt}{10pt}\selectfont\rotatebox{90}{$-$}\hspace{2.5pt}}}}}%
{\mathnormal{\subseteq}\llap{\text{\raisebox{0.3pt}{\fontsize{10pt}{10pt}\selectfont\rotatebox{90}{$-$}\hspace{2.5pt}}}}}%
{\mathnormal{\subseteq}\llap{\text{\raisebox{-0.3pt}{\fontsize{7pt}{7pt}\selectfont\rotatebox{90}{$-$}\hspace{1pt}}}}}%
{\mathnormal{\subseteq}\llap{\text{\raisebox{-0.3pt}{\fontsize{7pt}{7pt}\selectfont\rotatebox{90}{$-$}\hspace{1pt}}}}}%
}}
\newcommand{\supmseteq}{\mathrel{\reflectbox{$\submseteq$}}}
\newcommand{\lowersubmseteq}[1]{\mathrel{\submseteq_{#1}}}
\newcommand{\uppersubmseteq}[1]{\mathrel{\submseteq^{#1}}}
\newcommand{\lowersupmseteq}[1]{\mathrel{\supmseteq_{#1}}}
\newcommand{\uppersupmseteq}[1]{\mathrel{\supmseteq^{#1}}}
\newcommand{\submset}{%
\mathrel{\mathchoice%
{\mathnormal{\subset}\llap{\text{\raisebox{-0.8pt}{\fontsize{10pt}{10pt}\selectfont\rotatebox{90}{$-$}\hspace{2.5pt}}}}}%
{\mathnormal{\subset}\llap{\text{\raisebox{-0.8pt}{\fontsize{10pt}{10pt}\selectfont\rotatebox{90}{$-$}\hspace{2.5pt}}}}}%
{\mathnormal{\subset}\llap{\text{\raisebox{-0.3pt}{\fontsize{7pt}{7pt}\selectfont\rotatebox{90}{$-$}\hspace{1pt}}}}}%
{\mathnormal{\subset}\llap{\text{\raisebox{-0.3pt}{\fontsize{7pt}{7pt}\selectfont\rotatebox{90}{$-$}\hspace{1pt}}}}}%
}}
\newcommand{\lowersubmset}[1]{\mathrel{\submset_{#1}}}
\newcommand{\uppersubmset}[1]{\mathrel{\submset^{#1}}}
\newcommand{\supmset}{\mathrel{\reflectbox{$\submset$}}}
\newcommand{\lowersupmset}[1]{\mathrel{\supmset_{#1}}}
\newcommand{\uppersupmset}[1]{\mathrel{\supmset^{#1}}}

\DeclareMathOperator{\collapseset}{\mathcal{C}}

\newcommand{\category}[1]{\mathrm{#1}}
\newcommand{\POcat}{\category{PO}}
\newcommand{\uSLcat}{\category{uSL}}
\newcommand{\pocMoncat}{\category{pocMon}}
\newcommand{\jMoncat}{\category{jMon}}
\newcommand{\mMoncat}{\category{mMon}}
\newcommand{\xMoncat}{{x}\category{Mon}}
\newcommand{\cpocMoncat}{\category{cpocMon}}
\newcommand{\imMoncat}{\category{imMon}}
\newcommand{\cmMoncat}{\category{cmMon}}
\newcommand{\cSRngcat}{\category{cSRng}}
\newcommand{\DAGcat}{\category{DAG}}

\newcommand{\idfun}[1]{1_{#1}}
\newcommand{\functor}[1]{\mathit{#1}}
\DeclareMathOperator{\POfun}{\functor{PO}}
\DeclareMathOperator{\uSLfun}{\functor{uSL}}
\DeclareMathOperator{\pocMonfun}{\functor{pocMon}}
\DeclareMathOperator{\cpocMonfun}{\functor{cpocMon}}
\DeclareMathOperator{\jMonfun}{\functor{jMon}}
\DeclareMathOperator{\mMonfun}{\functor{mMon}}
\DeclareMathOperator{\xMonfun}{\text{$x$}\functor{Mon}}
\DeclareMathOperator{\imMonfun}{\functor{imMon}}
\DeclareMathOperator{\cmMonfun}{\functor{cmMon}}
\DeclareMathOperator{\cSRngfun}{\functor{cSRng}}
\DeclareMathOperator{\DAGfun}{\functor{DAG}}

\newcommand{\uSLfree}[1]{\uSLfun\langle#1\rangle}
\newcommand{\uSLeta}{\eta^{\uSLcat}}
\newcommand{\uSLetaat}[1]{\uSLeta_{#1}}
\newcommand{\uSLlift}[1]{{#1}^{\sharp_{\uSLcat}}}

\newcommand{\pocMonfree}[1]{\pocMonfun\langle#1\rangle}
\newcommand{\pocMoneta}{\eta^{\pocMoncat}}
\newcommand{\pocMonetaat}[1]{\pocMoneta_{#1}}
\newcommand{\pocMonlift}[1]{{#1}^{\sharp_{\pocMoncat}}}

\newcommand{\jMonfree}[1]{\jMonfun\langle#1\rangle}
\newcommand{\jMoneta}{\eta^{\jMoncat}}
\newcommand{\jMonetaat}[1]{\jMoneta_{#1}}
\newcommand{\jMonlift}[1]{{#1}^{\sharp_{\jMoncat}}}

\newcommand{\mMonfree}[1]{\mMonfun\langle#1\rangle}
\newcommand{\mMoneta}{\eta^{\mMoncat}}
\newcommand{\mMonetaat}[1]{\mMoneta_{#1}}
\newcommand{\mMonlift}[1]{{#1}^{\sharp_{\mMoncat}}}

\newcommand{\xMonfree}[1]{\xMonfun\langle#1\rangle}
\newcommand{\xMoneta}{\eta^{\xMoncat}}
\newcommand{\xMonetaat}[1]{\xMoneta_{#1}}
\newcommand{\xMonlift}[1]{{#1}^{\sharp_{\xMoncat}}}

\newcommand{\cSRngfree}[1]{\cSRngfun\langle#1\rangle}
\newcommand{\cSRngeta}{\eta^{\cSRngcat}}
\newcommand{\cSRngetaat}[1]{\cSRngeta_{#1}}
\newcommand{\cSRnglift}[1]{{#1}^{\sharp_{\cSRngcat}}}

\newcommand{\POfree}[1]{\POfun\langle#1\rangle}
\newcommand{\POeta}{\eta^{\POcat}}
\newcommand{\POetaat}[1]{\POeta_{#1}}
\newcommand{\POlift}[1]{{#1}^{\sharp_{\POcat}}}

\newcommand{\mtimes}[1]{\mathbin{\tilde{\cdot}_{#1}}}
\newcommand{\mplus}[1]{\mathbin{\tilde{\cup}_{#1}}}
\newcommand{\ctimes}[1]{\mathbin{\tilde{\otimes}_{#1}}}
\newcommand{\cplus}[1]{\mathbin{\tilde{\oplus}_{#1}}}

\DeclareMathOperator{\scope}{\mathrm{sc}}
\DeclareMathOperator{\defdom}{\mathrm{def}}

\newcommand{\reflclos}[1]{\mathrel{(#1)^=}}
\newcommand{\transclos}[2][+]{\mathrel{(#2)^{#1}}}
\newcommand{\refltransclos}[1]{\mathrel{(#1)^*}}

\newcommand{\XPDrel}[2][p]{\rightsquigarrow^{#1}_{#2}}
\newcommand{\XPDreleq}[2][p]{\rightsquigarrow^{#1, =}_{#2}}
\newcommand{\XPDord}[2][p]{>^{#1}_{#2}}
\newcommand{\XPDordeq}[2][p]{\geq^{#1}_{#2}}
\newcommand{\XPDleq}[2][p]{\leq^{#1}_{#2}}
\newcommand{\XPDgeq}[2][p]{\geq^{#1}_{#2}}
\newcommand{\XPDw}[2][p]{w^{#1}_{#2}}
\newcommand{\XPDW}[2][p]{W^{#1}_{#2}}
\newcommand{\XPDk}[2][p]{k^{#1}_{#2}}

\newcommand{\SPDrel}{\XPDrel[\mathrm{SPD}]}
\newcommand{\SPDreleq}{\XPDreleq[\mathrm{SPD}]}
\newcommand{\SPDleq}{\XPDleq[\mathrm{SPD}]}
\newcommand{\SPDgeq}{\XPDgeq[\mathrm{SPD}]}
\newcommand{\SPDord}{\XPDord[\mathrm{SPD}]}
\newcommand{\SPDw}{\XPDw[\mathrm{SPD}]}
\newcommand{\SPDW}{\XPDW[\mathrm{SPD}]}
\newcommand{\DPDrel}{\XPDrel[\mathrm{DPD}]}
\newcommand{\DPDreleq}{\XPDreleq[\mathrm{DPD}]}
\newcommand{\DPDord}{\XPDord[\mathrm{DPD}]}
\newcommand{\DPDw}{\XPDw[\mathrm{DPD}]}
\newcommand{\DPDW}{\XPDW[\mathrm{DPD}]}
\newcommand{\TPDrel}{\XPDrel[\mathrm{TPD}]}
\newcommand{\TPDreleq}{\XPDreleq[\mathrm{TPD}]}
\newcommand{\TPDleq}{\XPDleq[\mathrm{TPD}]}
\newcommand{\TPDgeq}{\XPDgeq[\mathrm{TPD}]}
\newcommand{\TPDord}{\XPDord[\mathrm{TPD}]}
\newcommand{\TPDw}{\XPDw[\mathrm{TPD}]}
\newcommand{\TPDW}{\XPDW[\mathrm{TPD}]}

\newcommand{\archleq}[1]{\mathrel{\lesssim^{\alpha}_{#1}}}
\newcommand{\archeq}[1]{\mathrel{\sim^{\alpha}_{#1}}}

\newcommand{\cancleq}[1]{\mathrel{\lesssim^{:}_{#1}}}
\newcommand{\canceq}[1]{\mathrel{\sim^{:}_{#1}}}

\newcommand{\downset}[2]{#1{\downarrow_{#2}}}


\addbibresource{c-semirings.bib}

\author{Alexander Knapp \and Alexander Schiendorfer}
\title{Embedding Constraint Relationships into C-Semirings}

\begin{document}
%\pagestyle{plain}
\maketitle
\tableofcontents

\section*{Introduction}

These notes provide technical details that are required to embed
constraint relationships into the c-semiring framework presented in
terms of category theory.  It contains all steps required to map a dag
to a partial order (Section~\ref{sec:po-graph}), construct the
\emph{free} meet monoid from this partial order
(Section~\ref{sec:poc-monoid}) as well as the \emph{free} c-semiring
(Section~\ref{sec:csemirings}). A constraint solving algorithm based
on branch-and-bound search is presented in §\ref{par:maxSolDegs} for
c-semirings and in §\ref{par:meetMonoidMaxSolDegs} for meet monoids.
A concrete instantiation for constraint relationships along with an
example soft constraint problem concludes the report in
Section~\ref{sec:constraint-relationships}.


\section{Partial Orders and Directed Acyclic Graphs}
\label{sec:po-graph}

\paragraph
A \emph{partial order} $(X, {\leq})$ is given by a set $X$ and a
binary relation ${\leq} \subseteq X \times X$ such that $\leq$ is
reflexive, transitive, and anti-symmetric on $X$.  For $x, y \in X$ we
write $x < y$ if $x \leq y$ and $x \neq y$, and $x \geq y$ resp.\ $x >
y$ if $y \leq x$ resp.\ $y < x$, and $x \parallel y$ if neither $x
\leq y$ nor $x \geq y$.

A \emph{partial order homomorphism} $\varphi : P \to Q$ from a partial
order $P = (|P|, {\leq_P})$ to a partial order $Q = (|Q|, {\leq_Q})$
is given by a map $\varphi : |P| \to |Q|$ such that $\varphi(p) \leq_Q
\varphi(p')$ if $p \leq_P p'$ for all $p, p' \in |P|$.

The category $\POcat$ of partial orders has the partial orders as
objects and the partial order homomorphisms as morphisms.

\paragraph
A \emph{directed acyclic graph}, or \emph{dag}, $(X, {\rightarrow})$
is given by a set $X$ and a binary relation ${\rightarrow} \subseteq X
\times X$ such that $\rightarrow^+$ is irreflexive.  If $x \rightarrow
y$, then $x$ is a \emph{predecessor} of $y$, and $y$ is a
\emph{successor} of $x$.

A \emph{dag homomorphism} $\varphi : G \to H$ from a dag $G = (|G|,
{\rightarrow_G})$ to a dag $H = (|H|, {\rightarrow_H})$ is given by a
map $\varphi : |G| \to |H|$ such that $\varphi(g) \rightarrow_H
\varphi(g')$ if $g \rightarrow_G g'$ for all $g, g' \in |G|$.

The category $\DAGcat$ of dags has the dags as objects and the dag
homomorphisms as morphisms.

\paragraph
Define the functor $\POfree{-} : \DAGcat \to \POcat$ by
%
\begin{gather*}
  \POfree{G} = (|G|, {\rightarrow_G^*})
\ \text{,}
\\
  \POfree{\varphi : G \to H} = \varphi
\ \text{.}
\end{gather*}

Define the functor $\DAGfun : \POcat \to \DAGcat$ by
%
\begin{gather*}
  \DAGfun(P) = (|P|, {<_P})
\ \text{,}
\\
  \DAGfun(\varphi : P \to Q) = \varphi
\ \text{.}
\end{gather*}

For each $G \in |\DAGcat|$, define $\POetaat{G} : G \to
\DAGfun(\POfree{G})$ by $\POetaat{G}(g) = g$.  Then $\POeta =
(\POetaat{G})_{G \in |\DAGcat|}$ is a natural transformation from
$\idfun{\DAGcat}$ to $\DAGfun \compfun \POfree{-}$.

Let $G \in |\DAGcat|$, $P \in |\POcat|$, and $\varphi : G \to
\DAGfun(P)$.  Define $\POlift{\varphi} : \POfree{G} \to P$ by
%
\begin{equation*}
  \POlift{\varphi}(g) = \varphi(g)
\ \text{.}
\end{equation*}

Then $\DAGfun(\POlift{\varphi})(\POetaat{G}(g)) = \varphi(g)$ and
$\POlift{\varphi}$ is unique with this property.

\begin{lemma}
$\POfree{G}$ is the free partial order over the dag
$G$.\qed
\end{lemma}


\section{Upper Semi-Lattices}

\paragraph
A (\emph{bounded}) \emph{upper semi-lattice} $(X, {\sqcup}, \bot)$ is
given by a set $X$, a binary operation ${\sqcup} : X \times X \to X$,
and a constant $\bot \in X$ such that the following axioms are
satisfied for all $x, y, z \in X$:
%
\begin{enumerate}
  \item $(x \sqcup y) \sqcup z = x \sqcup (y \sqcup z)$

  \item $x \sqcup y = y \sqcup x$

  \item $x \sqcup x = x$

  \item $x \sqcup \bot = x$
\end{enumerate}
%
In words, $\sqcup$ is associative, commutative, and idempotent, and
has $\bot$ as neutral element.

A (\emph{bounded}) \emph{upper semi-lattice homomorphism} $\varphi : U
\to V$ from an upper semi-lattice $U = (|U|, {\sqcup_U}, \bot_U)$ to
an upper semi-lattice $V = (|V|, {\sqcup_V}, \bot_V)$ is given by a
map $\varphi : |U| \to |V|$ such that for all $u_1, u_2 \in |U|$:
%
\begin{enumerate}
  \item $\varphi(u_1 \sqcup_U u_2) = \varphi(u_1) \sqcup_V
\varphi(u_2)$

  \item $\varphi(\bot_U) = \bot_V$
\end{enumerate}

The category $\uSLcat$ of upper semi-lattices has the upper
semi-lattices as objects and the upper semi-lattice homomorphisms as
morphisms.

\paragraph\label{par:PO2uSL}
Let $P$ be a partial order.  Let $\incfinsets(P)$ denote the set of
finite subsets of $|P|$ which only contain pairwise incomparable
elements w.r.t\ $\leq_P$.  For a subset $S \subseteq |P|$, let
$\Max{\leq_P}(S)$ denote the set of maximal elements of $S$ w.r.t.\
$\leq_P$ (in particular, if $S$ is finite,
$\Max{\leq_P}(S) \in \incfinsets(P)$).

Define the binary operation $\lowercup{P} :
\incfinsets(P) \times \incfinsets(P) \to
\incfinsets(P)$ by
%
\begin{equation*}
  I \cup_P J = \Max{\leq_P} (I \cup J)
\ \text{.}
\end{equation*}

\begin{lemma}
$(\incfinsets(P), {\lowercup{P}}, \emptyset)$ is an upper semi-lattice.
\end{lemma}
\begin{proof}
Let $I, J, K \in \incfinsets(P)$.  For the associativity of $\lowercup{P}$
we have
%
\begin{gather*}
  I \lowercup{P} (J \lowercup{P} K)
=
  \Max{\leq_P}(I \cup \Max{\leq}(J \cup K))
=
  \Max{\leq_P}(I \cup J \cup K)
={}\\\qquad
  \Max{\leq_P}(\Max{\leq_P}(I \cup J) \cup K)
=
  (I \lowercup{P} J) \lowercup{P} K
\ \text{,}
\end{gather*}
%
since $\Max{\leq_P}(I \cup \Max{\leq_P} X) = \Max{\leq_P}(I \cup X)$
for all $X \in \finsets |P|$.  $\lowercup{P}$ inherits commutativity from
$\cup$. For the idempotency of $\lowercup{P}$ we have
%
\begin{equation*}
  I \lowercup{P} I = \Max{\leq_P} (I \cup I) = \Max{\leq_P} I = I
\ \text{,}
\end{equation*}
%
since $I \in \incfinsets(P)$.  Finally, we have $I \lowercup{P}
\emptyset = I$.
\end{proof}

Define the functor $\uSLfree{-} : \POcat \to
\uSLcat$ by
%
\begin{gather*}
  \uSLfree{P}
=
  (\incfinsets(P), {\lowercup{P}}, \emptyset)
\ \text{,}
\\
  \uSLfree{\varphi : P \to Q}
=
  \lambda \{ p_1, \dots, p_n \} \in \incfinsets(P) \,.\, \Max{\leq_Q} \{ \varphi(p_1), \dots, \varphi(p_n) \}
\ \text{.}
\end{gather*}

\paragraph
Each upper semi-lattice $U$ induces a partial ordering ${\leq_U}
\subseteq |U| \times |U|$ on $|U|$ given by
%
\begin{equation*}
  u_1 \leq_U u_2 \iff u_1 \sqcup_U u_2 = u_2
\ \text{.}
\end{equation*}
%
Indeed, $\leq_U$ is reflexive on $|U|$ by the idempotency of
$\sqcup_U$, $\leq_U$ is transitive by the associativity of $\sqcup_U$,
and $\leq_U$ is anti-symmetric by the commutativity of $\sqcup_U$.
Furthermore, $\bot_U$ is the smallest element w.r.t.\ $\leq_U$, i.e.,
$\bot_U \leq_U u$ for all $u \in |U|$, by the neutrality of $\bot_U$.

Define the functor $\POfun : \uSLcat \to \POcat$ by
\begin{gather*}
  \POfun(U)
=
  (|U|, {\leq_U})
\ \text{,}
\\
  \POfun(\varphi : U \to V)
=
  \varphi
\ \text{,}
\end{gather*}
%
which is well-defined on objects by the remarks above and also
morphisms since if $u_1 \leq_U u_2$, i.e., $u_1 \sqcup_U u_2 = u_2$,
then $\varphi(u_1) \sqcup_V \varphi(u_2) = \varphi(u_1 \sqcup_U u_2) =
\varphi(u_2)$, i.e., $\varphi(u_1) \sqcup_V \varphi(u_2)$.

For each $P \in |\POcat|$, define $\uSLetaat{P} : P \to
\POfun(\uSLfree{P})$ by $\uSLetaat{P}(p) = \{ p \}$.  Then $\uSLeta =
(\uSLetaat{P})_{P \in |\POcat|}$ is a natural transformation from
$\idfun{\POcat}$ to $\POfun \compfun \uSLfree{-}$.

Let $P \in |\POcat|$, $U \in |\uSLcat|$, and $\varphi : P
\to \POfun(U)$.  Define $\uSLlift{\varphi} : \uSLfree{P} \to U$ by
%
\begin{gather*}
  \uSLlift{\varphi}(\{ p_1, \dots, p_n \}) = \varphi(p_1) \sqcup_U \cdots \sqcup_U \varphi(p_n)
\end{gather*}
%
for all $\{ p_1, \dots, p_n \} \in \incfinsets(P)$, where, if $n =
0$, the right hand side is to be understood as $\bot_U$;
$\uSLlift{\varphi}$ is indeed an upper semi-lattice homomorphism,
since for each $\{ p_1', \dots,\allowbreak p_n' \} \in \finsets |P|$
we have $\uSLlift{\varphi}(\Max{\leq_P} \{ p_1', \dots, p_n' \}) =
\varphi(p_1') \sqcup_U \cdots \sqcup_U \varphi(p_n')$: if $p_i' \leq_P
p_j'$, then $\varphi(p_i') \leq_{\POfun(U)} \varphi(p_j')$, i.e.,
$\varphi(p_i') \sqcup_U \varphi(p_j') = \varphi(p_j')$.

Then $\POfun(\uSLlift{\varphi})(\uSLetaat{P}(p)) = \varphi(p)$ and
$\uSLlift{\varphi}$ is unique with this property.

\begin{lemma}
$\uSLfree{P}$ is the free upper semi-lattice over the partial order
$P$.\qed
\end{lemma}

\paragraph\label{par:incfinsets-orderings}
The partial ordering of $\POfun(\uSLfree{P})$ on $\incfinsets(P)$
for a partial order $P$ is called the \emph{lower} or \emph{Hoare}
ordering on $\incfinsets(P)$ which we denote by $\lowersubseteq{P}$; it
is explicitly given by
%
\begin{align*}
  I \lowersubseteq{P} J &\iff \Max{\leq_P} (I \cup J) = J
\\
                     &\iff \forall p \in I \,.\, \exists q \in J \,.\, p \leq_P q
\end{align*}
%
for $I, J \in \incfinsets(P)$.  It is $\emptyset \lowersubseteq{P} I$
for all $I \in \incfinsets(P)$.

The dual of the Hoare ordering is the \emph{upper} or \emph{Smyth
  ordering} $\uppersubseteq{P}$ on $\incfinsets(P)$ defined by $I
\uppersubseteq{P} J$ if, and only if, $J \lowersubseteq{P^{-1}} I$, where
$P^{-1} = (|P|, {\geq_P})$.  Explicitly, the Smyth ordering is given
by
\begin{align*}
  I \uppersubseteq{P} J &\iff \Min{\leq_P} (I \cup J) = I
\\
                     &\iff \forall q \in J \,.\, \exists p \in I \,.\, p \leq_P q
\end{align*}
%
where $\Min{\leq_P}(S)$ is the set of minimal elements of $S \subseteq
|P|$.  In particular, the Smyth ordering also
induces a binary operation ${\uppercup{P}} : \incfinsets(P) \times
\incfinsets(P) \to \incfinsets(P)$ given by
%
\begin{equation*}
  I \uppercup{P} J = \Min{\leq_P}(I \cup J)
\ \text{,}
\end{equation*}
%
which is also associative, commutative, and idempotent.  Here,
$I \uppercup{P} \emptyset = I$, i.e., $\emptyset$ is again a neutral
element for $\uppercup{P}$, but $I \uppersubseteq{P} \emptyset$ for
all $I \in \incfinsets(P)$, i.e., $\emptyset$ is the greatest element
of $\incfinsets(P)$ w.r.t.\ $\uppersubseteq{P}$.

The \emph{convex} or \emph{Plotkin ordering} on $\incfinsets(P)$ is
defined by the intersection of $\lowersubseteq{P}$ and
$\uppersubseteq{P}$, which means
%
\begin{equation*}
  I \mathrel{({\lowersubseteq{P}} \cap {\uppersubseteq{P}})} J
\iff
  (\forall p \in I \,.\, \exists q \in J \,.\, p \leq_P q) \land
  (\forall q \in J \,.\, \exists p \in I \,.\, p \leq_P q)
\end{equation*}
%
for $I, J \in \incfinsets(P)$.

Finally, $\lowercup{P}$ is monotonic w.r.t.\ $\lowersubseteq{P}$, and
$\uppercup{P}$ is monotonic w.r.t.\ $\uppersubseteq{P}$, i.e., for all
$I, J, K \in \incfinsets(P)$,
%
\begin{gather*}
  I \lowersubseteq{P} J \text{ implies } I \lowercup{P} K \lowersubseteq{P} J \lowercup{P} K
\ \text{,}
\\
  I \uppersubseteq{P} J \text{ implies } I \uppercup{P} K \uppersubseteq{P} J \uppercup{P} K
\ \text{.}
\end{gather*}


\section{Partially Ordered Commutative Monoids}
\label{sec:poc-monoid}

\paragraph
A \emph{partially ordered commutative monoid} $(X, {\cdot},
\varepsilon, {\leq})$ is given by a set $X$, a binary operation
${\cdot} : X \times X \to X$, a constant $\varepsilon \in X$, and a
partial order relation ${\leq} \subseteq X \times X$ such that the
following axioms are satisfied for $x, x', y, y', z \in X$:
%
\begin{enumerate}
  \item $(x \cdot y) \cdot z = x \cdot (y \cdot z)$

  \item $x \cdot y = y \cdot x$

  \item $x \cdot \varepsilon = x$

  \item if $x \leq x'$ and $y \leq y'$, then $x \cdot y \leq x' \cdot y'$
\end{enumerate}
%
In words, $(X, {\cdot}, \varepsilon)$ is a commutative monoid with
unity $\varepsilon$ and $\leq$ is monotone w.r.t.\ $\cdot$.  (By
commutativity, it is enough to require that $x \leq x'$ implies
$x \cdot y \leq x' \cdot y$ to achieve monotonicity of $\leq$ w.r.t.\
$\cdot$.)\footnote{Partially ordered commutative monoids have also
  been called \emph{preference degree
    structures}~\cite{fargier-rollon-wilson:cj:2010}.}

A \emph{partially ordered commutative monoid homomorphism}
$\varphi : M \to N$ from a partially ordered commutative monoid
$M = (|M|, {\cdot}_M, \varepsilon_M, {\leq_M})$ to a partially ordered
commutative monoid $N = (|N|, {\cdot}_N, \varepsilon_N, {\leq}_N)$ is
given by a map $\varphi : |M| \to |N|$ such that for all
$m, n \in |M|$:
%
\begin{enumerate}
  \item $\varphi(m \cdot_M n) = \varphi(m) \cdot_N \varphi(n)$

  \item $\varphi(\varepsilon_M) = \varepsilon_N$

  \item if $m \leq_M n$, then $\varphi(m) \leq_N \varphi(n)$
\end{enumerate}

The category $\pocMoncat$ of partially ordered commutative monoids has
the partially ordered commutative monoids as objects and the partially
ordered commutative monoids homomorphisms as morphisms.

\paragraph
A partially ordered commutative monoid $M$ is a \emph{join monoid} if
for all $m, n \in |M|$
%
\begin{equation*}
  m \leq_M m \cdot_M n
\ \text{.}
\end{equation*}
%
This requirement is equivalent to requiring that $\varepsilon_M$ is
the smallest element w.r.t.\ $\leq_M$.  Indeed, if $m \leq_M m \cdot_M
n$ holds for all $m, n \in |M|$, then $\varepsilon_M \leq_M
\varepsilon_M \cdot_M n = n$ for all $n \in |M|$.  Conversely, if
$\varepsilon_M \leq_M n$ for all $n \in |M|$, then $m = m \cdot_M
\varepsilon_M \leq_M m \cdot_M n$ for all $m, n \in |M|$ by the
monotonicity of $\leq_M$.

Dually, a partially ordered commutative monoid $M$ is a \emph{meet
  monoid} if for all $m, n \in |M|$
%
\begin{equation*}
  m \cdot_M n \leq_M m
\ \text{,}
\end{equation*}
%
and this requirement is equivalent to requiring that $\varepsilon_M$ is
the greatest element w.r.t.\ $\leq_M$.\footnote{Meet monoids have also
  been referred to as \emph{partial valuation
    structures}~\cite{Gadducci2013} and \emph{ic-monoids}~\cite{Holzl2009}.

For soft constraint applications, $m \leq_M n$ will represent the fact
that value $m$ is ``worse than'' $n$, so $\varepsilon$ will be the top
(and best) element of the ordering. Think of $m$ and $n$ as abstract
weights where more is worse (``bad points'', ``penalty points'' or
heavy items).  To illustrate the meaning of ``$\leq$'' think of a
teeter-totter (or a weighing scale) where on the left side the heavier
item (or kid) sits and the lightest item (or kid) is the best.  The
position of the weighing scale then resembles $<$ (precisely
$\diagup$). Similarly, a solution violating many constraints will
carry a heavy weight and we search for the lightest solution.}

The full sub-categories of $\pocMoncat$ having all join monoids
respectively meet monoids as objects are denoted by $\jMoncat$ and
$\mMoncat$, respectively.

There are functors
%
\begin{equation*}
\begin{array}[t]{@{}l@{\quad\quad\quad}l@{}}
  \jMonfun : \mMoncat \to \jMoncat
&
  \mMonfun : \jMoncat \to \mMoncat
\\[.5ex]
  \jMonfun(M) = (|M|, {\cdot_M}, {\varepsilon_M}, {\leq_M^{-1}})
&
  \mMonfun(M) = (|M|, {\cdot_M}, {\varepsilon_M}, {\leq_M^{-1}})
\\[.5ex]
  \jMonfun(\varphi : M \to N) = \varphi
&
  \mMonfun(\varphi : M \to N) = \varphi
\end{array}
\end{equation*}
%
such that $\jMonfun \compfun \mMonfun = \idfun{\jMoncat}$ and
$\mMonfun \compfun \jMonfun = \idfun{\mMoncat}$.

\paragraph
For a set $X$ let $\finmsets(X)$ be the set of finite multisets over
$X$.  We write $\lbag x_1, \ldots, x_m\rbag$ with $x_i \in X$ for $1
\leq i \leq m$ or $\lbag l_1 x_1, \ldots, l_n x_n \rbag$ with $x_i \in
X$ and $l_i \in \NZ$ for $1 \leq i \leq n$ for an element of
$\finmsets(X)$, $T \mcup U$ for the multiset union of the multisets
$T$ and $U$, and $T \submseteq U$ for the sub-multiset relation, which
is a partial ordering relation on $\finmsets(X)$.

For a partial order $P$, the \emph{lower} or \emph{Hoare ordering} on
$\finmsets |P|$ is the binary relation ${\lowersubmseteq{P}}
\subseteq (\finmsets |P|) \times (\finmsets |P|)$ given by the
transitive closure of
%
\begin{gather*}
  T \submseteq U \text{ implies } T \lowersubmseteq{P} U
\ \text{,}
\\
  p \leq_P q \text{ implies } T \mcup \lbag p \rbag \lowersubmseteq{P} T \mcup \lbag q \rbag
\ \text{.}
\end{gather*}

If $T \lowersubmseteq{P} U$, then $T \mcup \lbag r \rbag
\lowersubmseteq{P} U \mcup \lbag r \rbag$ for all $r \in X$, since
this holds for both defining clauses of the ordering.

For an element $T = \lbag l_1 x_1, \dots, l_n x_n \rbag \in
\finmsets(X)$ with $l_1, \dots, l_n > 0$, $x_i \neq x_j$ for all $1
\leq i \neq j \leq n$, and $n \geq 0$ let $\mathcal{S}(T) = \bigcup_{1
  \leq i \leq n} \{ (j, x_i) \mid 1 \leq j \leq l_i \}$.

\begin{lemma}
$T \lowersubmseteq{P} U$ if, and only if, there is an injective
mapping $f : \mathcal{S}(T) \to \mathcal{S}(U)$ with $p \leq_P q$ if
$f(j, p) = (k, q)$ for all $(j, p) \in \mathcal{S}(T)$.
\end{lemma}
\begin{proof}
Let first $T \lowersubmseteq{P} U$ hold.  Then there are an $n > 1$
and $T_1, \ldots, T_n \in \finmsets(X)$ such that $T_1 = T$, $T_n =
U$, and either $T_i \submseteq T_{i+1}$ or $T_i = T_i' \mcup \lbag p
\rbag$ and $T_{i+1} = T_i' \mcup \lbag q \rbag$ with $p \leq_P q$ for
all $1 \leq i < n$.  For each $1 \leq i < n$ there is a map $f_i :
\mathcal{S}(T_i) \to \mathcal{S}(T_{i+1})$ as required in the claim as
follows: If $T_{n-1} \submseteq T_n$, then we choose $f_{i} =
\idfun{\mathcal{S}(T_i)}$.  If $T_i = T_i' \mcup \lbag p \rbag$ and
$T_{i+1} = T_i' \mcup \lbag q \rbag$ with $p \leq_P q$, then we choose
$f_{i} = \idfun{\mathcal{S}(T_i')} \cup \{ (j, p) \mapsto (k, q) \}$
where $j = |\{ l \mid (l, p) \in \mathcal{S}(T_i') \}|+1$ and $k = |\{
l \mid (l, q) \in \mathcal{S}(T_i') \}|+1$.  Then $f_n \compfun \ldots
\compfun f_1 : \mathcal{S}(T) \to \mathcal{S}(U)$ as required in the
claim.

For the converse, we prove that if $f : \mathcal{S}(T) \to
\mathcal{S}(U)$ is a mapping as required in the claim, then $T
\lowersubmseteq{P} U$ by induction on the cardinality of
$\mathcal{S}(T)$.  Let $f : \mathcal{S}(T) \to \mathcal{S}(U)$ be
given.  If $|\mathcal{S}(T)| = 0$, then $\lbag\rbag = T \submseteq U$.
Now let $|\mathcal{S}(T)| > 0$ and let $(j, p) \in \mathcal{S}(T)$
such that $j$ is maximal.  Then $f(j, p) = (k, q)$ with $p \leq_P q$.
Define $g : \mathcal{S}(U) \to \mathcal{S}(U) \setminus \{ (k, q) \}$
by $g(l, r) = (l, r)$ if $r \neq q$ or $l < k$, and $g(l, q) = (l-1,
q)$ if $l > k$.  Let $T', U' \in \finmsets(X)$ be defined by $T = T'
\mcup \lbag p \rbag$ and $U = U' \mcup \lbag q \rbag$.  Then
$\mathcal{S}(T') = \mathcal{S}(T) \setminus \{ (j, p) \}$ and $f' :
\mathcal{S}(T') \to \mathcal{S}(U')$ defined by $f'(l, r) = g(f(l,
r))$ for all $(l, r) \in \mathcal{S}(T')$ is an injective mapping as
required in the claim.  By induction hypothesis $T' \lowersubmseteq{P}
U'$ and thus, by the remark above, $T = T' \mcup \lbag p \rbag
\lowersubmseteq{P} U' \mcup \lbag p \rbag \lowersubmseteq{P} U' \mcup
\lbag q \rbag = U$.
\end{proof}

We call such a map a \emph{witness} for $T \lowersubmseteq{P} U$.

The relation $\lowersubmseteq{P}$ is obviously transitive and reflexive on
$\finmsets |P|$.  It is also antisymmetric: Assume for a
contradiction that there are $T$ and $U$ with $T \lowersubmseteq{P} U$ and
$U \lowersubmseteq{P} T$, but $T \neq U$ and choose an $T$ with minimal
cardinality satisfying this property.  Then $T \neq \lbag \rbag$.  Let
$f : \mathcal{S}(T) \to \mathcal{S}(U)$ and $g : \mathcal{S}(U) \to
\mathcal{S}(T)$ be witnessing maps for $T \lowersubmseteq{P} U$ and $U
\lowersubmseteq{P} T$.  Choose an element $(j, p) \in \mathcal{S}(T)$ such
that $p$ is maximal w.r.t.\ $\leq_P$ in $T$.  Let $f(j, p) = (k, q)
\in \mathcal{S}(U)$.  Then $p \leq_P q$.  If $p \neq q$, there would
be a $(j', p') \in \mathcal{S}(T)$ with $p \leq_P q \leq_P p'$ but $p
\neq p'$ contradicting the maximality of $p$ in $T$; thus $f(j, p) =
(k, p)$. Assume, without loss of generality, that $j$ and $k$ are maximal.
Remove the occurrence of $p$ from $T$, obtaining $T'$, and
from $U$, obtaining $U'$.  Then $T' \lowersubmseteq{P} U'$ and $U'
\lowersubmseteq{P} T'$, since $f' : \mathcal{S}(T') \to \mathcal{S}(U')$
with $f'(x) = f(x)$ if $x \neq (j, p)$ and $g' : \mathcal{S}(U') \to
\mathcal{S}(T')$ with $g'(y) = g(y)$ if $y \neq (k, p)$ are witnessing
maps, contradicting the minimality of $T$.

\paragraph
For a partial order $P$, multiset union $\mcup$ is monotonic w.r.t.\
$\lowersubmseteq{P}$, i.e., for all $T, U, V \in \finmsets |P|$,
%
\begin{equation*}
  T \lowersubmseteq{P} U \text{ implies } T \mcup V \lowersubseteq{P} U \mcup V
\ \text{;}
\end{equation*}
%
and $\lbag \rbag \lowersubmseteq{P} T$ for all $T \in \finmsets |P|$.
Since $\mcup$ also is associative and commutative, we have

\begin{lemma}
$(\finmsets |P|, {\mcup}, \lbag \rbag, {\submseteq_P})$ is a join
monoid.\qed
\end{lemma}

Define the functor $\jMonfree{-} : \POcat \to
\jMoncat$ by
%
\begin{gather*}
  \jMonfree{P}
=
  (\finmsets |P|, {\mcup}, \lbag \rbag, {\lowersubmseteq{P}})
\ \text{,}
\\
  \jMonfree{\varphi : P \to Q}
=
  \lambda {\lbag p_1, \dots, p_n \rbag} \in \finmsets |P| \,.\, {\lbag \varphi(p_1), \dots, \varphi(p_n) \rbag}
\ \text{.}
\end{gather*}

\paragraph
Dually, the \emph{upper} or \emph{Smyth ordering} on $\finmsets |P|$
is the binary relation ${\uppersubmseteq{P}} \subseteq (\finmsets |P|)
\times (\finmsets |P|)$, defined by $T \uppersubmseteq{P} U$ if, and only
if, $U \lowersubmseteq{P^{-1}} T$; more explicitly, the Smyth ordering on
$\finmsets |P|$ is given by the transitive closure of
%
\begin{gather*}
  T \supmseteq U \text{ implies } T \uppersubmseteq{P} U
\ \text{,}
\\
  p \leq_P q \text{ implies } T \mcup \lbag p \rbag \uppersubmseteq{P} T \mcup \lbag q \rbag
\ \text{,}
\end{gather*}
%
i.e., $T \uppersubmseteq{P} U$ if, and only if, there is an injective
mapping $g : \mathcal{S}(U) \to \mathcal{S}(T)$ with $p \leq_P q$ if
$g(k, q) = (j, p)$ for all $(k, q) \in \mathcal{S}(U)$; we call such a
map a \emph{witness} for $T \uppersubmseteq{P} U$.  The relation
$\uppersubmseteq{P}$ is also a partial ordering on $\finmsets |P|$, and,
again, $\mcup$ is monotonic w.r.t.\ $\uppersubmseteq{P}$, i.e.,
%
\begin{equation*}
  T \uppersubmseteq{P} U \text{ implies } T \mcup V \uppersubseteq{P} U \mcup V
\ \text{;}
\end{equation*}
%
and $T \uppersubmseteq{P} \lbag \rbag$ for all $T \in \finmsets |P|$.  Thus,

\begin{lemma}
$(\finmsets |P|, {\mcup}, \lbag \rbag, {\uppersubmseteq{P}})$ is a meet
monoid.\qed
\end{lemma}

This meet monoid over a partial order $P$ does not show suprema of
finite sets, in general: Consider the partial order $\mathrm{P} = (\{
\mathrm{a}, \mathrm{b}, \mathrm{c} \}, \{ \mathrm{a} < \mathrm{c},
\mathrm{b} < \mathrm{c} \})$ (which does show suprema). In the meet
monoid $(\finmsets |\mathrm{P}|, {\mcup}, \lbag \rbag,
{\uppersubmseteq{\mathrm{P}}})$, we have
%
\begin{gather*}
  \lbag \mathrm{c} \rbag \uppersubmset{\mathrm{P}} \lbag \mathrm{a} \rbag, \lbag \mathrm{b} \rbag
\ \text{,}\\
  \lbag \mathrm{a}, \mathrm{b} \rbag \uppersubmset{\mathrm{P}} \lbag \mathrm{a} \rbag, \lbag \mathrm{b} \rbag
\end{gather*}
%
and no $T \in \finmsets{|\mathrm{P}|}$ exists with $\lbag \mathrm{a},
\mathrm{b} \rbag, \lbag \mathrm{c} \rbag \uppersubmset{\mathrm{P}} T
\uppersubmset{\mathrm{P}} \lbag \mathrm{a} \rbag, \lbag \mathrm{b}
\rbag$ since, e.g., for $\lbag \mathrm{c} \rbag
\uppersubmset{\mathrm{P}} T$, $T$ can only be $\lbag \rbag$ by the
first rule (with $\lbag \mathrm{a}, \mathrm{b} \rbag, \lbag \mathrm{c}
\rbag \uppersubmset{\mathrm{P}} \lbag \rbag$), or $\lbag \mathrm{a}
\rbag$ or $\lbag \mathrm{b} \rbag$ by the second rule; but $\lbag
\mathrm{a} \rbag$ and $\lbag \mathrm{b} \rbag$ are incomparable
w.r.t.\ $\uppersubmseteq{\mathrm{P}}$.

Define the functor $\mMonfree{-} : \POcat \to
\mMoncat$ by
%
\begin{gather*}
  \mMonfree{P}
=
  (\finmsets |P|, {\mcup}, \lbag \rbag, {\uppersubmseteq{P}})
\ \text{,}
\\
  \mMonfree{\varphi : P \to Q}
=
  \lambda {\lbag p_1, \dots, p_n \rbag} \in \finmsets |P| \,.\, {\lbag \varphi(p_1), \dots, \varphi(p_n) \rbag}
\ \text{.}
\end{gather*}

In particular, $\jMonfree{P} = \jMonfun(\mMonfree{P^{-1}})$ and
$\mMonfree{P} = \mMonfun(\jMonfree{P^{-1}})$.

\paragraph
Finally, the \emph{convex} or \emph{Plotkin ordering} on $\finmsets
|P|$ is the intersection of $\lowersubmseteq{P}$ and $\uppersubmseteq{P}$.
Then $T \mathrel{({\lowersubmseteq{P}} \cap {\uppersubmseteq{P}})} U$ if,
and only if, there is a bijective mapping $h : \mathcal{S}(T) \to
\mathcal{S}(U)$ with $p \leq_P q$ if $h(j, p) = (k, q)$ for all $(j,
p) \in \mathcal{S}(T)$; we again call such a map a \emph{witness} for
$T \mathrel{({\lowersubmseteq{P}} \cap {\uppersubmseteq{P}})} U$.  The
relation ${\lowersubmseteq{P}} \cap {\uppersubmseteq{P}}$ is also a partial
ordering on $\finmsets |P|$, and again $\mcup$ is monotonic w.r.t.\
this ordering.

\begin{lemma}
$(\finmsets |P|, {\mcup}, \lbag \rbag, {\lowersubmseteq{P}} \cap
{\uppersubmseteq{P}})$ is a partially ordered commutative monoid.\qed
\end{lemma}

Define the functor $\pocMonfree{-} : \POcat \to
\pocMoncat$ by
%
\begin{gather*}
  \pocMonfree{P}
=
  (\finmsets |P|, {\mcup}, \lbag \rbag, {\lowersubmseteq{P}} \cap {\uppersubmseteq{P}})
\ \text{,}
\\
  \pocMonfree{\varphi : P \to Q}
=
  \lambda {\lbag p_1, \dots, p_n \rbag} \in \finmsets |P| \,.\, {\lbag \varphi(p_1), \dots, \varphi(p_n) \rbag}
\ \text{.}
\end{gather*}

\paragraph
Define the functor $\POfun : \pocMoncat \to \POcat$ by
%
\begin{gather*}
  \POfun(M)
=
  (|M|, {\leq_M})
\ \text{,}
\\
  \POfun(\varphi : M \to N)
=
  \varphi
\ \text{.}
\end{gather*}

For each $P \in |\POcat|$ and each $x \in \{ \functor{poc},
\functor{j}, \functor{m} \}$ define the partial order homomorphisms
$\xMonetaat{P} : P \to \POfun(\xMonfree{P})$ by $\xMonetaat{P}(p) =
\lbag p \rbag$.  Then each $\xMoneta = (\xMonetaat{P})_{P \in
  |\POcat|}$ is a natural transformation from $\idfun{\POcat}$ to
$\POfun \compfun \xMonfree{-}$.

Let $x \in \{ \functor{poc}, \functor{j}, \functor{m} \}$, $P \in
|\POcat|$, $M \in |\xMoncat|$, and $\varphi : P \to \POfun(M)$.
Define $\xMonlift{\varphi} : \xMonfree{P} \to M$ by
%
\begin{equation*}
  \xMonlift{\varphi}(\lbag p_1, \dots, p_n \rbag) = \varphi(p_1) \cdot_M \ldots \cdot_M \varphi(p_n)
\end{equation*}
%
for all $\lbag p_1, \ldots, p_n \rbag \in \finmsets |P|$, where, if $n = 0$,
the right hand side is to be understood as $\varepsilon_M$.

Then for all $x \in \{ \functor{poc}, \functor{j}, \functor{m} \}$,
$\POfun(\xMonlift{\varphi})(\xMonetaat{P}(p)) = \varphi(p)$ and all
$\xMonlift{\varphi}$ are unique with this property.

\begin{lemma}
$\pocMonfree{P}$, $\jMonfree{P}$, and $\mMonfree{P}$ are the free
partially ordered commutative, join, and meet monoids over the partial
order $P$, respectively.\qed
\end{lemma}

\paragraph
Each upper semi-lattice can also be viewed as a join monoid.  Indeed,
define the functor $\jMonfun : \uSLcat \to \jMoncat$ by
%
\begin{gather*}
  \jMonfun(U) = (|U|, {\sqcup_U}, \bot_U, {\leq_U})
\ \text{,}
\\
  \jMonfun(\varphi : U \to V) = \varphi
\ \text{.}
\end{gather*}

In particular, by §\ref{par:PO2uSL} and
§\ref{par:incfinsets-orderings}, for each partial order $P$,
$\jMonfun(\uSLfree{P}) = (\incfinsets(P), {\cup_P}, \emptyset,
{\subseteq_P})$ is a join monoid.

\subsection{Constructing Meet Monoids from Dags}

\paragraph\label{par:spd}
Let $G$ be a dag.  Consider the \emph{single-predecessor lifting}
${\SPDrel{G}} \subseteq (\finmsets |G|) \times (\finmsets |G|)$ of $G$
to the finite multisets $\finmsets |G|$ over the elements of $G$,
given by 
%
\begin{gather*}
  T \SPDrel{G} T \mcup \lbag g \rbag
\ \text{,}
\\
  g \rightarrow_G h \text{ implies } T \mcup \lbag g \rbag \SPDrel{G} T \mcup \lbag h \rbag
\ \text{.}
\end{gather*}
%
Then $G^{\mathrm{SPD}} = (\finmsets |G|, {\SPDrel{G}})$ is a dag, where
$T \SPDrel{G} U$ expresses that $U$ is \emph{worse} than $T$.  We write
$\SPDleq{G}$ for
${\geq_{\POfree{G^{\mathrm{SPD}}}}} = ((\SPDrel{G})^*)^{-1}$.  The
defining clauses for $\SPDrel{G}$ correspond to the defining clauses for
the upper ordering $\uppersubmseteq{\POfree{G}^{-1}}$.  Thus,
$(\finmsets |G|, {\mcup}, \lbag \rbag,
{\SPDleq{G}})$ is a meet monoid, and

\begin{lemma}
For each dag $G$, $\mMonfree{\POfree{G}^{-1}} \isorel (\finmsets |G|,
{\mcup}, \lbag \rbag, {\SPDleq{G}})$.\qed
\end{lemma}

\paragraph\label{par:tpd}
Let $G$ be a dag.  Consider the \emph{transitive-predecessors lifting}
${\TPDrel{G}} \subseteq (\finmsets |G|) \times (\finmsets |G|)$ of $G$
to the finite multisets $\finmsets |G|$ over the elements of $G$,
given by 
%
\begin{gather*}
  T \TPDrel{G} T \mcup \lbag g \rbag
\ \text{,}
\\
  g_1, \ldots, g_n \rightarrow_G^+ h \text{ implies } T \mcup \lbag g_1, \ldots, g_n \rbag \TPDrel{G} T \mcup \lbag h \rbag
\ \text{.}
\end{gather*}
%
Then $G^{\mathrm{TPD}} = (\finmsets |G|, {\TPDrel{G}})$ is a dag;
we write $\TPDleq{G}$ for ${\geq_{\POfree{G^{\mathrm{TPD}}}}} =
((\TPDrel{G})^*)^{-1}$.  Furthermore, $\TPDleq{G}$ is monotonic w.r.t.\
multiset union and thus $(\finmsets |G|, {\mcup}, \lbag \rbag,
{\TPDleq{G}})$ is a meet monoid.  We have

\begin{lemma}
Let $G$ be a dag, let $S_1, S_2 \in \finsets |G|$, and let $\bar{S}_1,
\bar{S}_2 \in \finmsets |G|$ be the finite multisets corresponding to
$S_1$ and $S_2$, respectively.  Then
%
\begin{equation*}
  \bar{S}_1 \TPDleq{G} \bar{S}_2
\ \text{implies}\
  \Max{\POfree{G}}(S_2) \lowersubseteq{\POfree{G}} \Max{\POfree{G}}(S_1)
\ \text{.}
\end{equation*}
\end{lemma}
\begin{proof}
Let $\bar{S}_1 \TPDleq{G} \bar{S}_2$ hold.  It suffices to prove the
claim for the two defining clauses for $\TPDrel{G}$.  For the case
that there is a $g \in S_1$ such that $S_2 = S_1 \setminus \{ g \}$,
we have
%
\begin{equation*}
  \Max{\POfree{G}}(S_2)
=
  \Max{\POfree{G}}(S_1 \setminus \{ g \})
\lowersubseteq{\POfree{G}}
  \Max{\POfree{G}}(S_1) \setminus \{ g \}
\lowersubseteq{\POfree{G}}
  \Max{\POfree{G}}(S_1)
\ \text{.}
\end{equation*}
%
For the case that there is an $h \in S_1$ and $g_1, \ldots, g_n \in
S_2$ with $g_1, \ldots, g_n \rightarrow_G^+ h$ such that $S_2 = (S_1
\setminus \{ h \}) \cup \{ g_1, \ldots, g_n \}$, we have
%
\begin{gather*}
  \Max{\POfree{G}}(S_2)
=
  \Max{\POfree{G}}(S_1 \cup \{ g_1, \ldots, g_n \}) \setminus \{ h \}
\lowersubseteq{\POfree{G}}{}\\\qquad
  \Max{\POfree{G}}(S_1) \setminus \{ h \}
\lowersubseteq{\POfree{G}}
  \Max{\POfree{G}}(S_1)
\ \text{,}
\end{gather*}
since $h \in S_1$.
\end{proof}

The converse is in general wrong: Let $G = (\{ g, h \}, \{ (g, h)
\})$, i.e., $g \rightarrow_G h$; then indeed we have
$\Max{\POfree{G}}(\{ g, h \}) \lowersubseteq{\POfree{G}}
\Max{\POfree{G}}(\{ h \})$, but $\lbag h \rbag \not\TPDleq{G} \lbag g,
h \rbag$.  However, if $S_2 \in \incfinsets(\POfree{G})$, then the
converse follows from the map $f : \Max{\POfree{G}}(S_2) \to
\Max{\POfree{G}}(S_1)$ witnessing $S_2 = \Max{\POfree{G}}(S_2)
\lowersubseteq{\POfree{G}} \Max{\POfree{G}}(S_1)$.

\paragraph
Let $P$ be a partial order.  Define
${\sqsubseteq_P} \subseteq (\finmsets |P|) \times (\finmsets |P|)$ by
%
\begin{equation*}
  T \mcup V \sqsubseteq_P U \mcup V
\iff
  \forall p \in \Max{\leq_P}(U) \,.\, \exists q \in \Max{\leq_P}(T)
  \,.\, p <_P q
\end{equation*}
%
where $T$ and $U$ have no common elements.  Then $\sqsubseteq_P$ is
obviously reflexive and antisymmetric.  In order to show transitivity,
let $W_1 \sqsubseteq_P W_2$ and $W_2 \sqsubseteq_P W_3$ hold.  Then there
are unique $T_i, U_i, V_i$ for $1 \leq i \leq 3$ such that $T_i$ and
$U_i$ have no common elements and
%
\begin{gather*}
  W_1 = T_1 \mcup V_1 = T_3 \mcup V_3
\\
  W_2 = U_1 \mcup V_1 = T_2 \mcup V_2
\\
  W_3 = U_3 \mcup V_3 = U_2 \mcup V_2 
\end{gather*}
%
Without loss of generality, we assume that there is no common element in
all three of $W_1$, $W_2$, and $W_3$.  We have to prove that for every
$p \in \Max{\leq_P}(U_3)$ there is a $q \in \Max{\leq_P}(T_3)$ such that
$p <_P q$.  The plan of the proof is shown is the following figure,
where $\hat{T}_i$ and $\hat{U}_i$ stand for $\Max{\leq_P}(T_i)$ and
$\Max{\leq_P}(U_i)$ for $1 \leq i \leq 3$ respectively, and we follow a
single element from the left to the right according to the arrows, where
the dashed arrows stand for $\leq_P$, the solid arrows for $<_P$, and
the double arrows for equality:
%
\bgroup\abovedisplayskip0pt\belowdisplayskip0pt
\begin{equation*}
\begin{tikzpicture}[scale=1,auto]
\tikzstyle{moveeq}=[draw,double distance=.3pt,-{>[length=2pt]},font={\fontsize{8pt}{8pt}\selectfont}]
\tikzstyle{movele}=[draw,-{>[length=2pt]},font={\fontsize{8pt}{8pt}\selectfont}]
\tikzstyle{moveleq}=[draw,dashed,-{>[length=2pt]},font={\fontsize{8pt}{8pt}\selectfont}]
\node (U3) at (0, 0) {$\hat{U}_3$};
\node (V3) at (0, -1) {$V_3$};
\node (U2) at (1.5, 0) {$\hat{U}_2$};
\node (V2) at (1.5, -1) {$V_2$};
\node (T2) at (3, 0) {$\hat{T}_2$};
\node (V2') at (3, -1) {$V_2$};
\node (U1) at (4.5, 0) {$\hat{U}_1$};
\node (V1) at (4.5, -1) {$V_1$};
\node (T1) at (6, 0) {$\hat{T}_1$};
\node (V1') at (6, -1) {$V_1$};
\node (T3) at (7.5, 0) {$\hat{T}_3$};
\node (V3') at (7.5, -1) {$V_3$};
%
\draw[moveleq] (U3) -- (U2);
\draw[moveeq] (U3) -- (V2);
\draw[moveleq] (V3) -- (U2);
\draw[moveeq] (V3) -- (V2);
%
\draw[movele] (U2) -- (T2);
\draw[moveeq] (V2) -- (V2');
%
\draw[moveleq] (T2) -- (U1);
\draw[moveeq] (T2) -- (V1);
\draw[moveleq] (V2') -- (U1);
\draw[moveeq] (V2') -- (V1);
%
\draw[movele] (U1) -- (T1);
\draw[moveeq] (V1) -- (V1');
%
\draw[moveleq] (T1) -- (T3);
\draw[moveeq] (T1) -- (V3');
\draw[moveleq] (V1') -- (T3);
\draw[moveeq] (V1') -- (V3');
%
\draw[moveeq] (V3') edge[out=-120, in=-60, distance=10, double distance=.3pt] (V3);
\end{tikzpicture}
\end{equation*}
\egroup

Let thus $p_3 \in \Max{\leq_P}(U_3)$.  We make a case distinction on
where $p_3$ resides w.r.t.\ the partition $W_3 = U_2 \mcup V_2$:

\smallskip%
1.~$p_3 \in U_2$: Then there is a $p_2 \in \hat{U}_2$ with
$p_3 \leq_P p_2$ and a $q_2 \in \hat{T}_2$ with $p_2 <_P q_2$.

1.1.~$q_2 \in U_1$: Then there is a $p_1 \in \hat{U}_1$ with
$q_2 \leq_P p_1$ and a $q_1 \in \hat{T}_1$ with $p_1 <_P q_1$.  If
$q_1 \in T_3$, then there is a $q_3 \in \hat{T}_3$ with
$q_1 \leq_P q_3$, and we are done; or $q_1 \in V_3$.

1.2.~$q_2 \in V_1$: If $q_2 \in T_3$, then there is a
$q_3 \in \hat{T}_3$ with $q_2 \leq_P q_3$, and we are done; or
$q_2 \in V_3$.

\smallskip%
2.~$p_3 \in V_2$:  Then $p_3 \in U_1$ or $p_3 \in V_1$.

2.1.~$p_3 \in U_1$: Then there is a $p_1 \in \hat{U}_1$ with
$p_3 \leq_P p_1$ and a $q_1 \in \hat{T}_1$ with $p_1 <_P q_1$.  If
$q_1 \in T_3$, then there is a $q_3 \in \hat{T}_3$ with
$q_1 \leq_P q_3$, and we are done; or $q_1 \in V_3$.

2.2.~$p_3 \in V_1$: If $p_3 \in T_3$, then there is a
$q_3 \in \hat{T}_3$ with $p_3 \leq_P q_3$; or $p_3 \in V_3$.  But both
$p_3 = q_3$ and $p_3 \in V_3$ are impossible, since otherwise there is
a common element of all $W_1$, $W_2$, and $W_3$; thus it must be
$p_3 <_P q_3$ with $q_3 \in \hat{T}_3$, and we are done.

\smallskip%
In the remaining unsettled cases of (1.1), (1.2), and (2.1) we get a
$q \in V_3$ with $p_3 <_P q$.  But for each $q \in V_3$ there is a
$q'$ with $q \leq_P q'$ and $q' \in \hat{T}_3$ or $q' \in V_3$, by the
same reasoning as for $p_3$ above.  Again, $q' = q$ is
impossible since then $q$ is a common element
of all $W_1$, $W_2$, and $W_3$.  If $q' \in \hat{T}_3$ with $q <_P q'$,
then $p_3 <_P q <_P q'$, and we are done.  Finally, if otherwise $q'
\in V_3$, then, by repeating the argument, we could build an infinite
strictly ascending chain in $V_3$ w.r.t.\ $\leq_P$ which is impossible
since $V_3$ is finite.

\subsection{Initial, Terminal, and Direct Product Meet Monoids}

\paragraph
Consider the singleton meet monoid $\mathrm{S} = (\{ \ast \}, {\cdot},
{\ast}, \{ (\ast, \ast) \})$ where $\ast \cdot \ast = \ast$.

\begin{lemma}
$\mathrm{S}$ is initial and terminal in $\mMoncat$.\qed
\end{lemma}

\paragraph
Let $M = (|M|, {\cdot_M}, \varepsilon_M, {\leq_M})$ and $N = (|N|,
{\cdot_N}, \varepsilon_N, {\leq_N})$ be meet monoids.  Define
${\cdot}_{M \times N} : (|M| \times |N|) \times (|M| \times |N|) \to |M|
\times |N|$ and ${\leq_{M \times N}} \subseteq (|M| \times |N|) \times
(|M| \times |N|)$ by
%
\begin{gather*}
  (m_1, n_1) \cdot_{M \times N} (m_2, n_2) = (m_1 \cdot_M m_2, n_1 \cdot_N n_2)
\\
  (m_1, n_1) \leq_{M \times N} (m_2, n_2) \iff m_1 \leq_M m_2 \land n_1 \leq_N n_2
\end{gather*}

Then $M \times N = (|M| \times |N|, {\cdot_{M \times N}},
(\varepsilon_M, \varepsilon_N), {\leq_{M \times N}})$ is a meet
monoid.

Define $\pi_1 : M \times N \to M$ by $\pi_1(m, n) = m$ and $\pi_2 : M
\times N \to N$ by $\pi_2(m, n) = n$.  Then $\pi_1$ and $\pi_2$ are
meet monoid homomorphisms.  Furthermore, for any meet monoid $P = (|P|,
{\cdot_P}, \varepsilon_P, {\leq_P})$ and two
meet monoid homomorphisms $\varphi_1 : P \to M$ and $\varphi_2 : P \to
N$, the meet monoid homomorphism $\langle \varphi_1, \varphi_2\rangle :
P \to M \times N$ defined by $\langle \varphi_1, \varphi_2\rangle(p) =
(\varphi_1(p), \varphi_2(p))$ is unique for the property $\varphi_1 =
\langle \varphi_1, \varphi_2\rangle \mathbin{;} \pi_1$ and $\varphi_2
= \langle \varphi_1, \varphi_2\rangle \mathbin{;} \pi_2$.

\begin{lemma}
$\mMoncat$ has finite products.\qed
\end{lemma}


\subsection{Lexicographic Products of Meet Monoids}
\label{subsec:lexprod}

\paragraph\label{par:lexmeetmonoids}
For a meet monoid $M$, define its set of \emph{collapsing
  elements}~\cite{Gadducci2013} by
%
\begin{equation*}
  \collapseset(M) = \{ m \in |M| \mid \exists m_1, m_2 \in |M| \,.\, m_1 <_M m_2 \land m_1 \cdot_M m = m_2 \cdot_M m \}
\ \text{.}
\end{equation*}

\begin{example}
Let $P$ be a partial order.

\smallskip

(1)~The set of collapsing elements of the meet monoid
$\mMonfree{P} = (\finmsets |P|, {\mcup}, \lbag \rbag,
{\uppersubmseteq{P}})$
is empty: If $T \uppersubmset{P} U$, then
$T \mcup \lbag p \rbag = U \mcup \lbag p \rbag$ for some $p \in |P|$
would imply that $T = U$.

\smallskip

(2)~The set of collapsing elements of the meet monoid
$\mMonfun(\jMonfun(\uSLfree{P})) = (\incfinsets(P), \cup_P,
\emptyset, {\lowersupseteq{P}})$
is $\incfinsets(P) \setminus \{ \emptyset \}$: If
$\emptyset \neq I \in \incfinsets(P)$, then
$I \lowersupset{P} \emptyset$, but
$\emptyset \lowercup{P} I = I = I \lowercup{P} I$.\qed
\end{example}

Generalising the first example, if $M$ is a \emph{strict} meet monoid,
i.e., $m <_M n$ implies $m \cdot_M o <_M n \cdot_M o$ for all $m, n, o
\in |M|$, then $\collapseset(M) = \emptyset$.  Generalising the second
example, all idempotent elements of a meet monoid $M$ which are
different from $\varepsilon_M$ are collapsing: If $m \in |M|$ such
that $m \neq \varepsilon_M$ and $m \cdot_M m = m$, then $m <_M
\varepsilon_M$ but $m \cdot_M m = m = \varepsilon_M \cdot_M m$.

\begin{lemma}
$|M| \setminus \collapseset(M)$ is closed under $\cdot_M$. 
\end{lemma}
\begin{proof}
We show that $m \cdot_M n \in \collapseset(M)$ if, and only if,
$m \in \collapseset(M)$ or $n \in \collapseset(M)$: If
$m \in \collapseset(M)$ or $n \in \collapseset(M)$, then obviously
$m \cdot_M n \in \collapseset(M)$.  Conversely, if
$m \cdot_M n \in \collapseset(M)$, then there are $m_1, m_2 \in |M|$
with $m_1 <_M m_2$, and
$m_1 \cdot_M (m \cdot_M n) = m_2 \cdot_M (m \cdot_M n)$; but if
$m \notin \collapseset(M)$, then $m_1 \cdot_M m <_M m_2 \cdot_M m$,
and if also $n \notin \collapseset(M)$, then
$(m_1 \cdot_M m) \cdot_M n <_M (m_2 \cdot_M m) \cdot_M n$.
\end{proof}

Note that in particular $\varepsilon_M \notin \collapseset(M)$ since
$m_1 <_M m_2$ implies $m_1 \cdot_M \varepsilon_M <_M m_2 \cdot_M
\varepsilon_M$.  Thus, $(|M| \setminus \collapseset(M), {\cdot_M},
\varepsilon_M, {\leq_M})$ forms a meet monoid.

\paragraph
A meet monoid $M$ is \emph{bounded} if $|M|$ has a smallest element
w.r.t.\ $\leq_M$; we denote this element by $\bot_M$ if it exists.

In a bounded meet monoid $M$ it holds that $m \cdot_M \bot_M = \bot_M$
for all $m \in |M|$, i.e., $\bot_M$ is an absorbing element, since
$m \cdot_M \bot_M \leq_M \varepsilon_M \cdot_M \bot_M = \bot_M$.
Furthermore, $\bot_M \in \collapseset(M)$ if
$\bot_M \neq \varepsilon_M$.  We call a bounded meet monoid
\emph{weakly strict} if $m <_M n$ implies
$m \cdot_M o <_M n \cdot_M o$ for all $m, n \in |M|$ and
$\bot_M \neq o \in |M|$.  Each meet monoid $M$ can be \emph{lifted}
into a bounded meet monoid
$M_{\bot} = (|M| \uplus \{ \bot \}, {\cdot_{M_{\bot}}},
\varepsilon_{M_{\bot}}, {\leq_{M_{\bot}}})$
setting $m \cdot_{M_{\bot}} \bot = \bot$,
$\varepsilon_{M_{\bot}} = \varepsilon_M$, and $\bot \leq_{M_{\bot}} m$
for all $m \in |M| \uplus \{ \bot \}$.

\paragraph
Let $M$ be a meet monoid and let $N$ be a bounded meet monoid.  Let
%
\begin{equation*}
  L = ((|M| \setminus \collapseset(M)) \times |N|) \cup (\collapseset(M) \times \{ \bot_N \})
\ \text{,}
\end{equation*}
%
i.e., the subset of pairs $(m, n) \in |M| \times |N|$ such that if
$m \in \collapseset(M)$, then $n = \bot_N$.  Define the binary
operation ${\cdot_L} : L \times L \to L$ by
%
\begin{equation*}
  (m_1, n_1) \cdot_L (m_2, n_2) = (m_1 \cdot_M m_2, n_1 \cdot_N n_2)
\ \text{.}
\end{equation*}
%
This is well-defined, i.e., for all $(m_1, n_1), (m_2, n_2) \in L$,
$(m_1 \cdot_M m_2, n_1 \cdot_N n_2) \in L$: Either
$m_1 \cdot_M m_2 \in |M| \setminus \collapseset(M)$ or
$m_1 \cdot_M m_2 \in \collapseset(M)$ and then we have to ensure that
$n_1 \cdot_N n_2 = \bot_N$.  But if
$m_1 \cdot_M m_2 \in \collapseset(M)$, then $m_1 \in \collapseset(M)$
or $m_2 \in \collapseset(M)$ by §\ref{par:lexmeetmonoids}.  Let first
$m_1 \in \collapseset(M)$.  Then $n_1 = \bot_N$, and
$n_1 \cdot_N n_2 = \bot_N \cdot_N n_2 = \bot_N$.  The case of
$m_2 \in \collapseset(M)$ is symmetric.

The binary operation $\cdot_L$ inherits associativity and
commutativity from $M$ and $N$.  Further define the element
$\varepsilon_L \in L$ by
%
\begin{equation*}
  \varepsilon_L = (\varepsilon_M, \varepsilon_N)
\ \text{,}
\end{equation*}
%
which is also well-defined, since $\varepsilon_M \notin \collapseset(M)$.  Also
$(m, n) \cdot_L \varepsilon_L = (m, n)$.

Finally, define the \emph{lexicographic ordering} ${\leq_L} \subseteq
L \times L$ on $L$ by
%
\begin{equation*}
  (m_1, n_1) \leq_L (m_2, n_2)
\iff
  (m_1 \neq m_2 \text{ and } m_1 \leq_M m_2) \text{ or } (m_1 = m_2 \text{ and } n_1 \leq_N n_2)
\ \text{.}
\end{equation*}

In order to show monotonicity of $\cdot_L$ w.r.t.\ $\leq_L$, i.e.,
that $(m_1, n_1) \cdot_L (m, n) \leq_L (m_2, n_2) \cdot_L (m, n)$
follows from $(m_1, n_1) \leq_L (m_2, n_2)$ for all $(m, n) \in L$,
the crucial insight is to note that $m$ might be collapsing and there
is nothing that then forces order-preservation in the second
component. More specifically, it might be the case that $m_1 <_M m_2$
but $n_1 >_N n_2$ which still yields $(m_1, n_1) \leq_L (m_2, n_2)$ as
we omit the second component. Were $m_1 \cdot_M m$ equal to
$m_2 \cdot_M m$, that is $m$ collapsing, we would have that
$(m_1, n_1) \cdot_L (m, n) >_L (m_2, n_2) \cdot_L (m,n)$, clearly
violating monotonicity.

\begin{lemma}
$(L, {\cdot_L}, \varepsilon_L, {\leq_L})$ is a meet monoid.
\end{lemma}
\begin{proof}
Since $(L, {\cdot_L}, \varepsilon_L)$ is a commutative monoid, it only
remains to show the monotonicity of $\cdot_L$ w.r.t.\ $\leq_L$.  Let
$(m_1, n_1) \leq_L (m_2, n_2)$ and an $(m, n) \in L$ be given.

\smallskip%
Case $m_1 <_M m_2$: If not $m_1 \cdot_M m <_M m_2 \cdot_M m$, i.e., $m_1
\cdot_M m = m_2 \cdot_M m$, then $m \in \collapseset(M)$ and thus $n =
\bot_N$.  Hence
\begin{gather*}
  (m_1, n_1) \cdot_L (m, n)
=
  (m_1 \cdot_M m, n_1 \cdot_N n)
=
  (m_1 \cdot_M m, \bot_N)
={}\\\qquad
  (m_2 \cdot_M m, \bot_N)
=
  (m_2 \cdot_M m, n_2 \cdot_N n)
=
  (m_2, n_2) \cdot_L (m, n)
\ \text{.}
\end{gather*}

\smallskip%
Case $m_1 = m_2$ and $n_1 \leq_N n_2$: Then $m_1 \cdot_M m = m_2 \cdot_M m$
and $n_1 \cdot_N n \leq_N n_2 \cdot_N n$.
\end{proof}

Let us write $M \ltimes N$ for
$(L, {\cdot_L}, \varepsilon_L, {\leq_L})$.  If both $M$ and $N$ are
bounded, then $M \ltimes N$ is bounded with
$\bot_{M \ltimes N} = (\bot_M, \bot_N)$.

\paragraph
Let $M$ and $N$ be meet monoids such that $N$ is bounded.  Then the
set of collapsing elements for the lexicographic combination of $M$
and $N$ is
%
\begin{equation*}
  \collapseset(M \ltimes N)
=
  (\collapseset(M) \times \{ \bot_N \}) \cup ((|M| \setminus \collapseset(M)) \times \collapseset(N))
\ \text{.}
\end{equation*}
%
Indeed, let $(m, n) \in \collapseset(M \ltimes N)$.  Then there are
$(m_1, n_1), (m_2, n_2) \in |M \ltimes N|$ with $(m_1, n_1) <_{M
  \ltimes N} (m_2, n_2)$, i.e., $m_1 <_M m_2$ or $m_1 = m_2$ and $n_1
<_N n_2$, and $(m_1, n_1) \cdot_{M \ltimes N} (m, n) = (m_2, n_2)
\cdot_{M \ltimes N} (m, n)$, i.e., $m_1 \cdot_M m = m_2 \cdot_M m$ and
$n_1 \cdot_N n = n_2 \cdot_N n$.  If $m_1 <_M m_2$, then $m \in
\collapseset(M)$ and therefore $n = \bot_N$; if $m_1 = m_2$, then $n
\in \collapseset(N)$.~--- Conversely, let first $(m, n) \in
\collapseset(M) \times \{ \bot_N \}$; then there are $m_1, m_2 \in
|M|$ with $m_1 <_M m_2$ and $m_1 \cdot_M m = m_2 \cdot_M m$, hence
$(m_1, \bot_N) <_{M \ltimes N} (m_2, \bot_N)$ with $(m_1, \bot_N)
\cdot_{M \ltimes N} (m, n) = (m_2, \bot_N) \cdot_{M \ltimes N} (m,
n)$.  Now, for the second case, let $(m, n) \in (|M| \setminus
\collapseset(M)) \times \collapseset(N)$; then there are $n_1, n_2 \in
|N|$ with $n_1 <_N n_2$ and $n_1 \cdot_N n = n_2 \cdot_N n$, hence
$(m, n_1) <_{M \ltimes N} (m, n_2)$ with $(m, n_1) \cdot_{M \ltimes N}
(m, n) = (m, n_2) \cdot_{M \ltimes N} (m, n)$.

Abbreviate $|M| \setminus \collapseset(M)$ by $R(M)$.  Then
%
\begin{gather*}
  |M \ltimes N| \setminus \collapseset(M \ltimes N)
=
  R(M \ltimes N)
={}\\\qquad
  ((R(M) \times |N|) \cup (\collapseset(M) \times \{ \bot_N \})) \setminus ((R(M) \times \collapseset(N)) \cup (\collapseset(M) \times \{ \bot_N \}))
={}\\\qquad
  R(M) \times R(N)
=
  (|M| \setminus \collapseset(M)) \times (|N| \setminus \collapseset(N))
\ \text{.}
\end{gather*}

Thus, for meet monoids $M$, $N$, and $O$, where $N$ and $O$ are bounded
%
\begin{gather*}
  |(M \ltimes N) \ltimes O|
={}\\\qquad
  (R(M \ltimes N) \times |O|) \cup (\collapseset(M \ltimes N) \times \{ \bot_O \})
={}\\\qquad
  (R(M) \times R(N) \times |O|) \cup (R(M) \times \collapseset(N) \times \{ \bot_O \}) \cup (\collapseset(M) \times \{ (\bot_N, \bot_O) \}) 
={}\\\qquad
  (R(M) \times ((R(N) \times |O|) \cup (\collapseset(N) \times \{ \bot_O \}))) \cup (\collapseset(M) \times \{ \bot_{N \ltimes O} \})
={}\\\qquad
  (R(M) \times |N \ltimes O|) \cup (\collapseset(M) \times \{ \bot_{N \ltimes O} \})
={}\\\qquad
  |M \ltimes (N \ltimes O)|
\ \text{,}
\end{gather*}
%
from which it follows that $\ltimes$ is associative.

\subsection{Cancellative Partially Ordered Commutative Monoids}

\paragraph
A partially ordered commutative monoid $M$ is \emph{cancellative} if
$m_1 \cdot_M n = m_2 \cdot_M n$ implies $m_1 = m_2$ for all
$m_1, m_2, n \in |M|$.

The full sub-category of partially ordered commutative monoids with the
cancellative partially ordered commutative monoids as objects is denoted
by $\cpocMoncat$.

\paragraph
Let $M$ be a partially ordered commutative monoid.  Consider the binary
relation ${\cancleq{M}} \subseteq |M| \times |M|$ with $m \cancleq{M} n$
iff there is an $o \in |M|$ with $m \cdot_M o \leq_M n \cdot_M o$.  Then
$\cancleq{M}$ is an admissible preordering for $\leq_M$, i.e.,
${\leq_M} \subseteq {\cancleq{M}}$ holds and $\cancleq{M}$ is reflexive
and transitive.  In particular, for transitivity, let
$m_1 \cancleq{M} m_2$ and $m_2 \cancleq{M} m_3$, and let
$o_1, o_2 \in |M|$ be such that $m_1 \cdot_M o_1 \leq_M m_2 \cdot_M o_1$
and $m_2 \cdot_M o_2 \leq_M m_3 \cdot_M o_2$.  Then
$m_1 \cdot_M (o_1 \cdot_M o_2) = (m_1 \cdot_M o_1) \cdot_M o_2 \leq_M
(m_2 \cdot_M o_1) \cdot_M o_2 = (m_2 \cdot_M o_2) \cdot_M o_1 \leq_M
(m_3 \cdot_M o_2) \cdot_M o_1 = m_3 \cdot_M (o_1 \cdot_M o_2)$.
The relation $\cancleq{M}$ also is a precongruence w.r.t.\ $\cdot_M$,
i.e., if $m_1 \cancleq{M} m_2$ and $n_1 \cancleq{M} n_2$, then
$m_1 \cdot_M n_1 \cancleq{M} m_2 \cdot_M n_2$.  In fact, let
$o, p \in |M|$ be such that $m_1 \cdot_M o \leq_M m_2 \cdot_M o$ and
$n_1 \cdot_M p \leq_M n_2 \cdot_M p$.  Then
$m_1 \cdot_M n_1 \cdot_M o \cdot_M p \leq_M m_2 \cdot_M n_2 \cdot_M o
\cdot_M p$.

Let ${\canceq{M}} = {\cancleq{M}} \cap {\cancleq{M}}^{-1}$.  Then
$m \canceq{M} n$ iff there is an $o \in |M|$ with
$m \cdot_M o = n \cdot_M o$.  Let $[m]_{\canceq{M}}$ denote the
equivalence class of $m \in |M|$ w.r.t.\ $\canceq{M}$.
  Then
%
\begin{equation*}
  M/{\canceq{}} = (\{ [m]_{\canceq{M}} \mid m \in |M| \}, {\cdot_{M/\canceq{}}}, [\varepsilon_M]_{\canceq{M}}, {\leq_{M/\canceq{}}})
\end{equation*}
%
with
$[m]_{\canceq{M}} \cdot_{M/\canceq{}} [n]_{\canceq{M}} = [m \cdot_M
n]_{\canceq{M}}$
and $[m]_{\canceq{M}} \leq_{M/{\canceq{}}} [n]_{\canceq{M}}$ iff
$m \cancleq{M} n$ forms a cancellative partially ordered commutative
monoid: the multiplication and the partial ordering on $|M/{\canceq{}}|$
are well"=formed~\cite[p.~34]{wechler:1992} and if
$[m]_{\canceq{M}} \cdot_{M/{\canceq{}}} [o]_{\canceq{M}} =
[n]_{\canceq{M}} \cdot_{M/{\canceq{}}} [o]_{\canceq{M}}$,
then $m \cdot_M o \canceq{M} n \cdot_M o$, i.e.,
$m \cdot_M o \cdot_M p = n \cdot_M o \cdot_M p$ for some $p \in |M|$,
hence $m \canceq{M} n$, and thus $[m]_{\canceq{M}} = [n]_{\canceq{M}}$.

Define the functor $\cpocMonfun : \pocMoncat \to \cpocMoncat$ by
%
\begin{gather*}
  \cpocMonfun(M) = M/{\canceq{}}
\ \text{,}\\
  \cpocMonfun(\varphi : M \to N) = \lambda [m]_{\canceq{M}} \,.\, [\varphi(m)]_{\canceq{N}}
\ \text{,}
\end{gather*}
%
which is well defined, since if $m \canceq{M} m'$, then there is an
$o \in |M|$ with $m \cdot_M o = m' \cdot_M o$ and
$\varphi(m) \cdot_N \varphi(o) = \varphi(m \cdot_M o) = \varphi(m'
\cdot_M o) = \varphi(m') \cdot_N \varphi(o)$,
i.e., $\varphi(m) \canceq{N} \varphi(m')$.

\paragraph
Let $M$ be a partially ordered commutative monoid.  Let $N$ be a
cancellative partially ordered commutative monoid and let
$\varphi : M \to N$ be a partially ordered commutative monoid morphism.
Define $\varphi^{:} : M/{\canceq{}} \to N$ by
$\varphi^{:}([m]_{\canceq{M}}) = \varphi(m)$; this is well defined,
since if $m \canceq{M} m'$, then there is an $o \in |M|$ with
$m \cdot_M o = m' \cdot_M o$ and thus
$\varphi(m) \cdot_N \varphi(o) = \varphi(m \cdot_M o) = \varphi(m'
\cdot_M o) = \varphi(m') \cdot_N \varphi(o)$,
which implies $\varphi(m) = \varphi(m')$ since $N$ is cancellative.
Further, define the partially ordered commutative monoid morphism
$\delta_M : M \to M/{\canceq{}}$ by $\delta_M(m) = [m]_{\canceq{M}}$.
Then $\varphi^{:}$ is the unique meet monoid morphism such that
$\varphi^{:} \compfun \delta_M = \varphi$.

The family
$\delta = (\delta_M : M \to M/{\canceq{}})_{M \in |\pocMoncat|}$ forms a
natural transformation from $\idfun{\pocMoncat}$ to $\cpocMonfun$.
Furthermore, each $\delta_M$ is surjective.

\begin{lemma}
$(\delta, {-}^{:})$ forms an epireflection of $\cpocMoncat$ in
$\pocMoncat$.\qed
\end{lemma}


\subsection{Idempotent Meet Monoids}

\paragraph
A meet monoid $M$ is \emph{idempotent} if $m \cdot_M m = m$ for all
$m \in |M|$.  For each meet monoid morphism $\varphi : M \to N$ with $M$
and $N$ idempotent, it holds that
$\varphi(m) = \varphi(m \cdot_M m) = \varphi(m) \cdot_N \varphi(m) =
\varphi(m)$.

The full sub-category of meet monoids with the idempotent meet monoids
as objects is denoted by $\imMoncat$.

\paragraph
Let $M$ be a meet monoid.  An element $m \in |M|$ is \emph{archimedean
  smaller} than an element $n \in |M|$, written as $m \archleq{M} n$, if
there is a $k \in \NZ$ such that $m^{(k)_M} \leq_M n$ (where we write
$m^{(k)_M}$ for the $k$-fold product of $m$ w.r.t.\ $\cdot_M$).  Two
elements $m, n \in |M|$ are \emph{archimedean equivalent}, written as
$\archeq{M}$, if $m \archleq{M} n$ and $n \archleq{M} m$.

Then $\archleq{M}$ is an admissible preordering for $\leq_M$, i.e.,
${\leq_M} \subseteq {\archleq{M}}$ holds and $\archleq{M}$ is reflexive
and transitive.  In particular, for transitivity, let
$m_1 \archleq{M} m_2$ and $m_2 \archleq{M} m_3$, and let
$k_1, k_2 \in \NZ$ be such that $m_1^{(k_1)_M} \leq_M m_2$ and
$m_2^{(k_2)_M} \leq_M m_3$.  Then
$m_1^{(k_1 \cdot k_2)_M} = (m_1^{(k_1)_M})^{(k_2)_M} \leq_M
m_2^{(k_2)_M} \leq_M m_3$.

The relation $\archleq{M}$ also is a precongruence w.r.t.\ $\cdot_M$,
i.e., if $m_1 \archleq{M} m_2$ and $n_1 \archleq{M} n_2$, then
$m_1 \cdot_M n_1 \archleq{M} m_2 \cdot_M n_2$.  In fact, let
$k, l \in \NZ$ be such that $m_1^{(k)_M} \leq_M m_2$ and
$n_1^{(l)_M} \leq_M n_2$.  Then
$(m_1 \cdot_M n_1)^{(k+l)_M} = m_1^{(k)_M} \cdot_M m_1^{(l)_M} \cdot_M
n_1^{(k)_M} \cdot_M n_1^{(l)_M} \leq_M m_1^{(k)_M} \cdot_M n_1^{(l)_M}
\leq_M m_2 \cdot_M n_2$ as $M$ is a meet monoid.

Let $[m]_{\archeq{M}}$ denote the equivalence class of $m \in |M|$
w.r.t.\ archimedean equivalence $\archeq{M}$.  Then
%
\begin{equation*}
  M/{\archeq{}} = (\{ [m]_{\archeq{M}} \mid m \in |M| \}, {\cdot_{M/\archeq{}}}, [\varepsilon_M]_{\archeq{M}}, {\leq_{M/\archeq{}}})
\end{equation*}
%
with
$[m]_{\archeq{M}} \cdot_{M/\archeq{}} [n]_{\archeq{M}} = [m \cdot_M
n]_{\archeq{M}}$
and $[m]_{\archeq{M}} \leq_{M/{\archeq{}}} [n]_{\archeq{M}}$ iff
$m \archleq{M} n$ forms an idempotent meet monoid: the multiplication
and the partial ordering on $|M/{\archeq{}}|$ are
well"=formed~\cite[p.~34]{wechler:1992} and
$m \cdot_M m \archeq{M} m$, i.e.,
$[m]_{\archeq{M}} \cdot_{M/{\archeq{}}} [m]_{\archeq{M}} = [m \cdot_M
m]_{\archeq{M}} = [m]_{\archeq{M}}$.

Define the functor $\imMonfun : \mMoncat \to \imMoncat$ by
%
\begin{gather*}
  \imMonfun(M) = M/{\archeq{}}
\ \text{,}\\
  \imMonfun(\varphi : M \to N) = \lambda [m]_{\archeq{M}} \,.\, [\varphi(m)]_{\archeq{N}}
\ \text{,}
\end{gather*}
%
which is well defined, since if $m \archeq{M} m'$, then there are
$k, k' \in \NZ$ with $m^{(k)_M} \leq_M m'$ and ${m'}^{(k')_M} \leq_M m$
and thus $\varphi(m^{(k)_M}) \leq_N \varphi(m')$ and
$\varphi({m'}^{(k')_M}) \leq_N \varphi(m)$, i.e.,
$(\varphi(m))^{(k)_N} \leq_N \varphi(m')$ and
$(\varphi(m'))^{(k')_N} \leq_N \varphi(m)$, that is,
$\varphi(m) \archeq{N} \varphi(m')$.

\paragraph
Let $M$ be a meet monoid.  Let $N$ be an idempotent meet monoid and let
$\varphi : M \to N$ be a meet monoid morphism.  Define
$\varphi^{\alpha} : M/{\archeq{}} \to N$ by
$\varphi^{\alpha}([m]_{\archeq{M}}) = \varphi(m)$; this is
well"=defined, since if $m \archeq{M} m'$, then there are $k, k' \in \NZ$
with $m^{(k)_M} \leq_M m'$ and ${m'}^{(k')_M} \leq_M m$ and thus
$\varphi(m^{(k)_M}) \leq_N \varphi(m')$ and
$\varphi({m'}^{(k')_M}) \leq_N \varphi(m)$, i.e.,
$(\varphi(m))^{(k)_N} \leq_N \varphi(m')$ and
$(\varphi(m'))^{(k')_N} \leq_N \varphi(m)$ which implies
$\varphi(m) \leq_N \varphi(m')$ and $\varphi(m') \leq_N \varphi(m)$ as
$N$ is idempotent, that is, $\varphi(m) = \varphi(m')$.  Further, define
the meet monoid morphism $\alpha_M : M \to M/{\archeq{}}$ by
$\alpha_M(m) = [m]_{\archeq{M}}$.  Then $\varphi^{\alpha}$ is the unique
meet monoid morphism such that
$\varphi^{\alpha} \compfun \alpha_M = \varphi$.

The family
$\alpha = (\alpha_M : M \to M/{\archeq{}})_{M \in |\mMoncat|}$ forms a
natural transformation from $\idfun{\mMoncat}$ to $\imMonfun$.
Furthermore, each $\alpha_M$ is surjective.

\begin{lemma}
$(\alpha, {-}^{\alpha})$ forms an epireflection of $\imMoncat$ in
$\mMoncat$.\qed
\end{lemma}

\subsection{Complete Meet Monoids}

\paragraph
A meet monoid $M$ is \emph{complete} if every subset $X$ of $|M|$ has a
supremum w.r.t.\ $\leq_M$, written as $\bigvee_M X$, and
$m \cdot_M \bigvee_M X = \bigvee_M m \cdot_M X$, where
$m \cdot_M X = \{ m \cdot_M n \mid n \in X \}$.  In particular, a
complete meet monoid $M$ is bounded with $\bigvee_M \emptyset$ the
bottom element.  A meet monoid morphism $\varphi : M \to N$ with $M$ and
$N$ complete meet monoids is \emph{continuous} if
$\varphi(\bigvee_M X) = \bigvee_N \varphi(X)$ for all $X \subseteq M$,
where $\varphi(X) = \{ \varphi(m) \mid m \in X \}$ for all
$X \subseteq M$.

The sub-category of meet monoids with the complete meet monoids as
objects and the continuous meet monoid morphisms as morphisms is denoted
by $\cmMoncat$.

\paragraph
Let $M$ be a meet monoid.  An $I \subseteq |M|$ is a \emph{downset} of
$M$ if $m' \in I$ implies $m \in I$ for all $m \leq_M m'$.  For an
$X \subseteq |M|$ we write $\downset{X}{M}$ for the smallest downset
containing $X$, i.e.,
$\downset{X}{M} = \{ m \in |M| \mid \exists m' \in X \,.\, m \leq_M m'
\}$.
For a set $\mathcal{I}$ of downsets of $M$, $\bigcup \mathcal{I}$ is a
downset of $M$, and it is the supremum of $\mathcal{I}$ w.r.t.\
$\subseteq$.

Define
$I \mtimes{M} J = \downset{\{ m \cdot_M n \mid m \in I,\ n \in J \}}{M}$
for two downsets $I, J$ of $M$, and
$I \mtimes{M} \mathcal{J} = \{ I \mtimes{M} J \mid J \in \mathcal{J} \}$
for a downset $I$ of $M$ and $\mathcal{J}$ a set of downsets of $M$.

Let $I$ be a downset of $M$ and $\mathcal{J}$ a set of downsets of
$M$.  Then
$I \mtimes{M} \bigcup \mathcal{J} = \bigcup I \mtimes{M} \mathcal{J}$.
Indeed, let first $m \in I \mtimes{M} \bigcup \mathcal{J}$.  Then
there is an $m_I \in I$ and an
$m_{\mathcal{J}} \in \bigcup \mathcal{J}$ such that
$m \leq_M m_I \cdot_M m_{\mathcal{J}}$.  Furthermore, there is a
$J \in \mathcal{J}$ such that $m_{\mathcal{J}} \in J$.  Thus
$m_I \cdot_M m_{\mathcal{J}} \in I \mtimes{M} \mathcal{J}$, which
yields $m \in I \mtimes{M} \mathcal{J}$, and hence
$m \in \bigcup I \mtimes{M} \mathcal{J}$.  Conversely, let
$m \in \bigcup I \mtimes{M} \mathcal{J}$.  Then there is a
$J \in \mathcal{J}$ such that $m \in I \mtimes{M} J$ and $m_I \in I$
and $m_J \in J$ with $m \leq_M m_I \cdot_M m_J$.  Thus
$m_J \in \bigcup \mathcal{J}$ and hence
$m \in I \mtimes{M} \bigcup \mathcal{J}$.

\begin{lemma}
$\downset{M}{} = (\{ \downset{X}{M} \mid X \subseteq |M| \},
\mtimes{M}, \downset{\{ \varepsilon_M \}}{M}, {\subseteq})$
is a complete meet monoid.\qed
\end{lemma}

Define the functor $\cmMonfun : \mMoncat \to \cmMoncat$ by
%
\begin{gather*}
  \cmMonfun(M) = \downset{M}{}
\ \text{,}\\
  \cmMonfun(\varphi : M \to N) = \lambda \downset{X}{M} \,.\, \downset{\varphi(X)}{N}
\ \text{.}
\end{gather*}
%
Therein $\cmMonfun(\varphi : M \to N)$ is (1)~well-defined, i.e., if
$\downset{X}{M} = \downset{X'}{M}$, then
$\downset{\varphi(X)}{N} = \downset{\varphi(X')}{N}$, and
(2)~continuous: (1)~If $n \in \downset{\varphi(X)}{N}$, there is an
$m \in X$ such that $n \leq_N \varphi(m)$, and hence an $m' \in X'$ such
that $m \leq_M m'$, i.e., $n \leq_N \varphi(m) \leq_N \varphi(m')$, from
which $n \in \downset{\varphi(X')}{N}$ follows; and the converse case is
symmetric. (2)~For each set of subsets $\mathcal{X}$ of $M$,
$n \in \varphi(\bigcup \downset{\mathcal{X}}{M})$ (where
$\downset{\mathcal{X}}{M} = \{ \downset{X}{M} \mid X \in \mathcal{X}
\}$)
if, and only if, there is an $X \in \mathcal{X}$ and an $m \in X$ such
that $n \leq_N \varphi(m)$ if, and only if,
$n \in \bigcup \downset{\varphi(\mathcal{X})}{N}$ (where
$\varphi(\mathcal{X}) = \{ \varphi(X) \mid X \in \mathcal{X} \}$).

\paragraph
Let $M$ be a meet monoid.  Let $N$ be a complete meet monoid and
$\varphi : M \to N$ a meet monoid morphism.  Define
$\varphi^{\downarrow} : \downset{M}{} \to N$ by
$\varphi^{\downarrow}(\downset{X}{M}) = \bigvee_N \varphi(X)$.  This is
well-defined, since $\downset{X}{M} = \downset{X'}{M}$ implies
$\bigvee_N \varphi(X) = \bigvee_N \varphi(X')$: If
$\bigvee_N \varphi(X) \leq_N n$, i.e., $\varphi(m) \leq_N n$ for all
$m \in X$, then also $\bigvee_N \varphi(X') \leq_N n$ as for each
$m' \in X'$ there is an $m \in X$ with $m' \leq_M m$, from which
$\varphi(m') \leq_N \varphi(m) \leq_N n$ follows; and the converse is
symmetric.  Further, define the meet monoid morphism
$\gamma_M : M \to \downset{M}{}$ by
$\gamma_M(m) = \downset{\{ m \}}{M}$.  Then $\varphi^{\downarrow}$ is
the unique continuous meet monoid morphism such that
$\varphi^{\downarrow} \compfun \gamma_M = \varphi$.  Indeed, let
$\psi : \downset{M}{} \to N$ be a continuous meet monoid morphism with
$\psi \compfun \gamma_M = \varphi$.  Then
%
\begin{gather*}\textstyle
  \varphi^{\downarrow}(\downset{X}{M})
=
  \bigvee_N \{ \varphi(m) \mid m \in X \})
=
  \bigvee_N \{ \varphi^{\downarrow}(\downset{\{ m \}}{M}) \mid m \in X \}
={}\\\qquad\textstyle
  \bigvee_N \{ \psi(\downset{\{ m \}}{M}) \mid m \in X \}
=
  \psi(\bigcup \{ \downset{\{ m \}}{M} \mid m \in X \})
=
  \psi(\downset{X}{M})
\ \text{.}
\end{gather*}

\begin{lemma}
$(\gamma, {-}^{\downarrow})$ forms a reflection of $\cmMoncat$ in
$\mMoncat$.\qed
\end{lemma}


\section{C-Semirings}
\label{sec:csemirings}

\paragraph
A \emph{c-semiring}~\cite{BistarelliMRSVF99} $(X, {\oplus}, {\otimes}, \mathbf{0}, \mathbf{1})$
is given by a set $X$, two binary operations ${\otimes}, {\oplus} : X
\times X \to X$, and two constants $\mathbf{0}, \mathbf{1} \in X$ such
that the following axioms are satisfied for all $x, y, z \in X$:
%
\begin{enumerate}
  \item $(x \oplus y) \oplus z = x \oplus (y \oplus z)$

  \item $x \oplus y = y \oplus x$

  \item $x \oplus \mathbf{1} = \mathbf{1}$

  \item $x \oplus \mathbf{0} = x$

  \item $(x \otimes y) \otimes z = x \otimes (y \otimes z)$

  \item $x \otimes y = y \otimes x$

  \item $x \otimes \mathbf{0} = \mathbf{0}$

  \item $x \otimes \mathbf{1} = x$

  \item $x \otimes (y \oplus z) = (x \otimes y) \oplus (x \otimes z)$
\end{enumerate}

In words, $\oplus$ is associative and commutative, has $\mathbf{1}$ as
annihilator and $\mathbf{0}$ as neutral element; $\otimes$ is associative and
commutative, has $\mathbf{0}$ as annihilator and $\mathbf{1}$ as neutral element; and
$\otimes$ distributes over $\oplus$.

A \emph{c-semiring homomorphism} $\varphi : A \to B$ from a c-semiring
$A = (|A|, {\oplus}_A, {\otimes}_A, \mathbf{0}_A, \mathbf{1}_A)$ to a
c-semiring $B = (|B|, {\oplus}_B, {\otimes}_B, \mathbf{0}_B,
\mathbf{1}_B)$ is given by a map $\varphi : |A| \to |B|$ such that for
all $a_1, a_2 \in |A|$:
%
\begin{enumerate}
  \item $\varphi(a_1 \oplus_A a_2) = \varphi(a_1) \oplus_B \varphi(a_2)$

  \item $\varphi(a_1 \otimes_A a_2) = \varphi(a_1) \otimes_B \varphi(a_2)$

  \item $\varphi(\mathbf{0}_A) = \mathbf{0}_B$

  \item $\varphi(\mathbf{1}_A) = \mathbf{1}_B$
\end{enumerate}

The category $\cSRngcat$ of c-semirings has the c-semirings as
objects and the c-semiring homomorphisms as morphisms.

\paragraph
In a c-semiring $(X, {\oplus}, {\otimes}, \mathbf{0}, \mathbf{1})$ the
operation $\oplus$ is idempotent:
%
\begin{equation*}
  x \oplus x
=
  (x \otimes \mathbf{1}) \oplus (x \otimes \mathbf{1})
=
  x \otimes (\mathbf{1} \oplus \mathbf{1})
=
  x \otimes \mathbf{1}
=
  x
\ \text{.}
\end{equation*}

Thus, there is a functor $\uSLfun : \cSRngcat \to \uSLcat$, defined by
%
\begin{gather*}
  \uSLfun(A) = (|A|, \oplus_A, \mathbf{0})
\ \text{,}
\\
  \uSLfun(\varphi : A \to B) = \varphi
\ \text{.}
\end{gather*}

For a c-semiring $A$, the thereby induced ordering
$\leq_{\uSLfun(A)}$, explicitly given by $a \leq_{\uSLfun(A)} b$ if,
and only if, $a \oplus_A b = b$, will be written as $\preceq_A$.

With this definition, for all $a, b, c \in |A|$ it holds that
%
\begin{enumerate}
  \item $\mathbf{0} \preceq_A a \preceq_A \mathbf{1}$;

  \item $a \preceq_A a \oplus_A b$ and $b \preceq_A a \oplus_A b$;

  \item if $a \preceq_A c$ and $b \preceq_A c$, then $a \oplus_A b \preceq_A c$.
\end{enumerate}
%
Also $\oplus_A$ is monotone w.r.t.\ $\preceq_A$ in both arguments, i.e.,
%
\begin{equation*}
  a \preceq_A a' \text{ and } b \preceq_A b' \text{ implies } a \oplus_A b \preceq_A a' \oplus_A b'
\ \text{.}
\end{equation*}

\paragraph
In a c-semiring $(X, {\oplus}, {\otimes}, \mathbf{0}, \mathbf{1})$ the
operation $\otimes$ is monotone w.r.t.\ the induced ordering
$\preceq$, since if $x \preceq x'$, i.e., $x \oplus x' = x'$, then
%
\begin{equation*}
  (x \otimes y) \oplus (x' \otimes y)
=
  (x \oplus x') \otimes y
=
  x' \otimes y 
\ \text{,}
\end{equation*}
i.e., $x \otimes y \preceq x' \otimes y$, from which it follows that
%
\begin{equation*}
  x \preceq x' \text{ and } y \preceq y' \text{ implies } x \otimes y \preceq x' \otimes y'
\ \text{.}
\end{equation*}

Furthermore, for all $x, y \in X$
%
\begin{equation*}
  x \otimes y \preceq x \text{ and } x \otimes y \preceq y
\ \text{,}
\end{equation*}
%
since
%
\begin{equation*}
  (x \otimes y) \oplus x
=
  (x \otimes y) \oplus (x \otimes \mathbf{1})
=
  x \otimes (y \oplus \mathbf{1})
=
  x \otimes \mathbf{1}
=
  x
\ \text{.}
\end{equation*}

Thus, there is a functor $\mMonfun : \cSRngcat \to \mMoncat$, given by
%
\begin{gather*}
  \mMonfun(A) = (|A|, {\otimes_A}, \mathbf{1}_A, {\preceq_A})
\ \text{,}
\\
  \mMonfun(\varphi : A \to B) = \varphi
\ \text{.}
\end{gather*}
%
Note that $\mMonfun(A)$ is a bounded meet monoid with
$\bot_{\mMonfun(A)} = \mathbf{0}_A$.

\subsection{Constructing C-Semirings from Meet Monoids}

\paragraph
Let $M$ be meet monoid.  We write $\incfinsets(M)$ for
$\incfinsets(\POfun(M))$.  Define the operations $\mplus{M},
\mtimes{M} : \incfinsets(M) \times \incfinsets(M) \to
\incfinsets(M)$ by
%
\begin{gather*}
  I \mplus{M} J
=
  \Max{\leq_M} (I \cup J)
\ \text{,}
\\
  I \mtimes{M} J
=
  \Max{\leq_M} \{ m \cdot_M n \mid m \in I,\ n \in J \}
\ \text{.}
\end{gather*}

\begin{lemma}\label{lem:join-monoid-2-c-semiring}
$(\incfinsets(M), \mplus{M}, \mtimes{M}, \emptyset, \{
\varepsilon_M \})$ is a c-semiring.
\end{lemma}
\begin{proof}
Let $I, J, K \in \incfinsets(M)$.

The operation $\mplus{M}$ is associative and commutative and has
$\emptyset$ as neutral element by §\ref{par:PO2uSL}.  Furthermore, $I
\mplus{M} \{ \varepsilon_M \} = \{ \varepsilon_M \}$, since
$\varepsilon_M$ is the greatest element of $|M|$ w.r.t.\ $\leq_M$.

For the associativity of ${\mtimes{M}}$ we have
%
\begin{gather*}
  I \mtimes{M} (J \mtimes{M} K)
={}\\\qquad
  \Max{\leq_M} \{ m_I \cdot_M m_{JK} \mid m_I \in I,\ m_{JK} \in \Max{\leq_M}\{ m_J \cdot_M m_K \mid m_J \in J,\ m_K \in K \} \}
={}\\\qquad
  \Max{\leq_M} \{ m_I \cdot_M m_J \cdot_M m_K \mid m_I \in I,\ m_J \in J,\ m_K \in K \} \}
={}\\\qquad
  \Max{\leq_M} \{ m_{IJ} \cdot_M m_K \mid m_{IJ} \in \Max{\leq_M}\{ m_I \cdot_M m_J \mid m_I \in I,\ m_J \in J \},\ m_K \in K \}
={}\\\qquad
  (I \mtimes{M} J) \mtimes{M} K
\ \text{,}
\end{gather*}
%
since
%
\begin{equation*}
  \Max{\leq_M}\{ m \cdot_M n \mid m \in I,\ n \in \Max{\leq_M}(X) \} =
  \Max{\leq_M}\{ m \cdot_M n \mid m \in I,\ n \in X \}
\end{equation*}
%
for all $X \in \finsets |M|$. 
Assume $n \in X$ but $n \not \in \Max{\leq_M}(X)$;
then there exists some (maximal) $n' \in X$ such that $n <_M n'$; 
Since $n \leq n'$ implies $m \cdot_M n \leq m \cdot_M n'$  by the monotonicity of
$\cdot_M$, $m \cdot_M n'$

Also $\mtimes{M}$ inherits commutativity from $\cdot_M$; $I \mtimes{M}
\emptyset = \emptyset$ by definition; and $I \mtimes{M} \{
\varepsilon_M \} = I$, since $\varepsilon_M$ is neutral in $M$.

Finally, $\mtimes{M}$ distributes over $\mplus{M}$:
%
\begin{gather*}
  I \mtimes{M} (J \mplus{M} K)
={}\\\qquad
  \Max{\leq_M} \{ m_I \cdot_M m_{JK} \mid m_I \in I,\ m_{JK} \in \Max{\leq_M}(J \cup K) \}
={}\\\qquad
  \Max{\leq_M} \{ m_I \cdot_M m_{JK} \mid m_I \in I,\ m_{JK} \in J \cup K \}
={}\\\qquad
  \Max{\leq_M}(\{ m_I \cdot_M m_J \mid m_I \in I,\ m_J \in J \} \cup \{ m_I \cdot_M m_K \mid m_I \in I,\ m_K \in K \})
={}\\\qquad\begin{split}
  \Max{\leq_M}(\Max{\leq_M} \{ &m_I \cdot_M m_J \mid m_I \in I,\ m_J \in J \} \cup{}\\[-.4ex]
               \Max{\leq_M} \{ &m_I \cdot_M m_K \mid m_I \in I,\ m_K \in K \}) ={}
\end{split}\\\qquad
  (I \mtimes{M} J) \mplus{M} (I \mtimes{M} K)
\ \text{,}
\end{gather*}
since
%
\begin{equation*}
  \Max{\leq_M}(I \cup \Max{\leq_M}(X)) = \Max{\leq_M}(I \cup X)
\end{equation*}
%
for all $X \in \finsets |M|$.
\end{proof}

Let $\varphi : M \to N$ be a meet monoid homomorphism.  For $X \in
\finsets |M|$, we have
%
\begin{equation*}
  \Max{\leq_N}(\varphi(\Max{\leq_M}(X))) = \Max{\leq_N}(\varphi(X))
\ \text{.}
\end{equation*}
%
Indeed, on the one hand, $\Max{\leq_N}(\varphi(\Max{\leq_M}(X)))
\subseteq \Max{\leq_N}(\varphi(X))$, since $\Max{\leq_M}(X) \subseteq
X$.  For $\Max{\leq_N}(\varphi(X)) \subseteq
\Max{\leq_N}(\varphi(\Max{\leq_M}(X)))$ it suffices to show that for
each $n \in \varphi(X)$ there is an $n' \in \varphi(\Max{\leq_M}(X))$
such that $n \leq_M n'$.  Thus, let $n \in \varphi(X)$, i.e., $n =
\varphi(m)$ for some $m \in X$.  Then there is an $m' \in
\Max{\leq_M}(X)$ with $m \leq_M m'$, hence $n = \varphi(m) \leq_N
\varphi(m')$, and $\varphi(m') \in \varphi(\Max{\leq_M}(X))$.

Define the functor $\cSRngfree{-} : \mMoncat \to
\cSRngcat$ by
%
\begin{gather*}
  \cSRngfree{M}
=
  (\incfinsets(M), {\mplus{M}}, {\mtimes{M}}, \emptyset, \{ \varepsilon_M \})
\ \text{,}
\\
  \cSRngfree{\varphi : M \to N}
=
  \lambda \{ m_1, \dots, m_k \} \in \incfinsets(M) \,.\, \Max{\leq_N} \{ \varphi(m_1), \dots, \varphi(m_k) \}
\ \text{.}
\end{gather*}
%
Indeed, $\cSRngfree{\varphi : M \to N}$ is a c-semiring homomorphism
from $\cSRngfree{M}$ to $\cSRngfree{N}$:
%
\begin{gather*}
  \cSRngfree{\varphi}(\emptyset)
=
  \emptyset
\ \text{,}
\\
  \cSRngfree{\varphi}(\{ \varepsilon_M \})
=
  \{ \varphi(\varepsilon_M) \}
=
  \{ \varepsilon_N \}
\ \text{,}
\\
  \cSRngfree{\varphi}(I_1 \mplus{M} I_2)
=
  \cSRngfree{\varphi}(\Max{\leq_M}(I_1 \cup I_2))
={}\\\qquad
  \Max{\leq_N}(\varphi(\Max{\leq_M}(I_1 \cup I_2)))
=
  \Max{\leq_N}(\varphi(I_1 \cup I_2))
=
  \Max{\leq_N}(\varphi(I_1) \cup \varphi(I_2))
={}\\\qquad
  \cSRngfree{\varphi}(I_1) \mplus{N} \cSRngfree{\varphi}(I_2)
\ \text{,}
\\
  \cSRngfree{\varphi}(I_1 \mtimes{M} I_2)
=
  \cSRngfree{\varphi}(\Max{\leq_M} \{ m_1 \cdot_M m_2 \mid m_1 \in I_1,\ m_2 \in I_2 \}))
={}\\\qquad
  \Max{\leq_N}(\varphi(\Max{\leq_M} \{ m_1 \cdot_M m_2 \mid m_1 \in I_1,\ m_2 \in I_2 \}))
={}\\\qquad
  \Max{\leq_N} \{ \varphi(m_1 \cdot_M m_2) \mid m_1 \in I_1,\ m_2 \in I_2 \}
={}\\\qquad
  \Max{\leq_N} \{ \varphi(m_1) \cdot_N \varphi(m_2) \mid m_1 \in I_1,\ m_2 \in I_2 \}
={}\\\qquad
  \Max{\leq_N} \{ n_1 \cdot_N n_2 \mid n_1 \in \varphi(I_1),\ n_2 \in \varphi(I_2) \}
={}\\\qquad
  \cSRngfree{\varphi}(I_1) \mtimes{N} \cSRngfree{\varphi}(I_2)
\ \text{.}
\end{gather*}

\paragraph\label{par:c-semiring-free}
For each $M \in |\mMoncat|$, define $\cSRngetaat{M} : M \to
\mMonfun(\cSRngfree{M})$ by $\cSRngetaat{M}(m) = \{ m \}$.  Then
$\cSRngeta = (\cSRngetaat{M})_{M \in |\mMoncat|}$ is a natural
transformation from $\idfun{\mMoncat}$ to $\mMonfun \compfun
\cSRngfree{-}$.

Let $M \in |\mMoncat|$, $A \in |\cSRngcat|$, and $\varphi : M \to
\mMonfun(A)$.  Define $\cSRnglift{\varphi} : \cSRngfree{M} \to A$ by
%
\begin{equation*}
  \cSRnglift{\varphi}(\{ m_1, \dots, m_n \}) = \varphi(m_1) \oplus_A \cdots \oplus_A \varphi(m_n)
\end{equation*}
%
for all $\{ m_1, \dots, m_n \} \in \incfinsets(M)$, where, if $n =
0$, the right hand side is to be understood as $\mathbf{0}_A$;
$\cSRnglift{\varphi}$ is indeed a c-semiring homomorphism, since for
each $\{ m_1', \dots,\allowbreak m_n' \} \in \finsets |M|$ we have
$\cSRnglift{\varphi}(\Max{\leq_M} \{ m_1', \dots, m_n' \}) =
\varphi(m_1') \oplus_A \cdots \oplus_A \varphi(m_n')$: if $m_i' \leq_M
m_j'$ then $\varphi(m_i') \leq_{\mMonfun(A)} \varphi(m_j')$, i.e.,
$\varphi(m_i') \preceq_A \varphi(m_j')$, and thus $\varphi(m_i')
\oplus_A \varphi(m_j') = \varphi(m_j')$.

Then $\mMonfun(\cSRnglift{\varphi})(\cSRngetaat{M}(m)) = \varphi(m)$
and $\cSRnglift{\varphi}$ is unique with this property.

\begin{lemma}
$\cSRngfree{M}$ is the free c-semiring over the meet monoid $M$.
\end{lemma}

\paragraph\label{par:frontier-algebra}
A similar construction of c-semiring addition and multiplication
operations has been introduced by E.~Rollón~\cite{rollon:phd:2008},
though starting from a given c-semiring.  She proves that when $A$ is
a c-semiring, its so-called \emph{frontier algebra}
$(\incsets(A) \setminus \{ \emptyset \}, \cplus{A}, \ctimes{A}, \{
\mathbf{0}_A \}, \{ \mathbf{1}_A \})$
again is a c-semiring, where $\incsets(A)$ are the non-empty subsets
of $|A|$ containing only pairwise incomparable elements w.r.t.\
$\preceq_A$, and
%
\begin{gather*}
  I \cplus{A} J
=
  \Max{\preceq_A} (I \cup J)
\ \text{,}
\\
  I \ctimes{A} J
=
  \Max{\preceq_A} \{ a \otimes_A b \mid a \in I,\ b \in J \}
\end{gather*}
%
for all $I, J \in \incsets(A) \setminus \{ \emptyset \}$.

The underlying set of the frontier algebra contains sets of arbitrary
cardinality, not only finite sets as in our approach, since no free
construction is intended.  The condition that only non-empty sets have
to be considered is missing in~\cite{rollon:phd:2008}; the empty set
has to be excluded, however, since otherwise
$\emptyset \cplus{A} \{ \mathbf{0}_A \} = \{ \mathbf{0}_A \}$ and
$\emptyset \ctimes{A} \{ \mathbf{0}_A \} = \emptyset$.  In fact, we
have to consider also $\emptyset$ in order to obtain a bottom element
of the free c-semiring over an arbitrary meet monoid.  If we only
applied the construction of a free c-semiring in
§\ref{par:c-semiring-free} to the sub-category of bounded meet
monoids, we also could exclude $\emptyset$ and would obtain
$\{ \bot_M \}$ as bottom element of the free c-semiring over the
bounded meet monoid $M$.

The finite case discussed here has been covered by H.~Fargier,
E.~Rollón and N.~Wilson for \emph{preference degree structures}, which
are our partially ordered commutative
monoids~\cite{fargier-rollon-wilson:cj:2010}; however, it is not
characterised as a free construction.

\paragraph
A meet monoid $M$ is \emph{total} if for all $m_1, m_2 \in |M|$:
%
\begin{equation*}
  \text{$m_1 \leq_M m_2$ or $m_1 = m_2$ or $m_2 \leq_M m_1$}
\ \text{.}
\end{equation*}

For a total bounded meet monoid
$M$, $(|M|, {\max^{\leq_M}}, {\cdot_M}, {\bot_M}, {\varepsilon_M})$ is
a c-semiring, and it is isomorphic to the free c-semiring
$\cSRngfree{M}$ over the bounded meet monoid $M$ in the sense of
§\ref{par:frontier-algebra}.

\paragraph
Let $M$ be a meet monoid with idempotent multiplication, i.e.,
$m \cdot_M m = m$ for all $m \in |M|$.  Then $\mtimes{M}$ is also
idempotent: Let $I \in \incfinsets(M)$.  For all $m, n \in I$, we
have $m \cdot_M n \leq_M n$.  Thus
%
\begin{gather*}\textstyle
  I \mtimes{M} I
=
  \Max{\leq_M} \{ m \cdot_M n \mid m \in I,\ n \in I \}
=
  \Max{\leq_M} \bigcup \{ \{ m \cdot_M n \mid m \in I \} \mid n \in I \}
={}\\\qquad\textstyle
  \Max{\leq_M} \bigcup \{ \Max{\leq_M} \{ m \cdot_M n \mid m \in I \} \mid n \in I \}
=
  \Max{\leq_M} \bigcup \{ \{ n \} \mid n \in I \}
={}\\\qquad\textstyle
  \Max{\leq_M} \{ n \mid n \in I \}
=
  I
\ \text{.}
\end{gather*}

\subsection{Initial, Terminal, and Direct Product C-Semirings}

\paragraph
Consider the c-semiring of \emph{boolean values} $\mathrm{B} = (\{
\bot, \top \}, {\lor}, {\land}, \bot, \top)$ where $\lor$ and $\land$
have their usual meaning.

\begin{lemma}
$\mathrm{B}$ is initial in $\cSRngcat$.\qed
\end{lemma}

\paragraph
Consider the c-semiring $\mathrm{T} = (\{ \ast \}, {\cdot}, {\cdot},
{\ast}, {\ast})$ with $\ast \cdot \ast = \ast$.

\begin{lemma}
$\mathrm{T}$ is terminal in $\cSRngcat$.\qed
\end{lemma}

\paragraph
Let $A = (|A|, {\oplus}_A, {\otimes}_A, \mathbf{0}_A, \mathbf{1}_A)$
and $B = (|B|, {\oplus}_B, {\otimes}_B, \mathbf{0}_B, \mathbf{1}_B)$
be c-semirings.  Define ${\oplus}_{A \times B},\allowbreak
{\otimes}_{A \times B} : (|A| \times |B|) \times (|A| \times |B|) \to
|A| \times |B|$ by
%
\begin{gather*}
  (a_1, b_1) \oplus_{A \times B} (a_2, b_2) = (a_1 \oplus_A a_2, b_1 \oplus_B b_2)
\\
  (a_1, b_1) \otimes_{A \times B} (a_2, b_2) = (a_1 \otimes_A a_2, b_1 \otimes_B b_2)
\end{gather*}

Then $A \times B = (|A| \times |B|, {\oplus}_{A
  \times B}, {\otimes}_{A \times B}, (\mathbf{0}_A, \mathbf{0}_B),
(\mathbf{1}_A, \mathbf{1}_B))$ is a c-semiring.

Define $\pi_1 : A \times B \to A$ by $\pi_1(a, b) = a$ and $\pi_2 : A
\times B \to B$ by $\pi_2(a, b) = b$.  Then $\pi_1$ and $\pi_2$ are
c-semiring homomorphisms.  Furthermore, for any c-semiring $C = (|C|,
{\oplus}_C, {\otimes}_C, \mathbf{0}_C, \mathbf{1}_C)$ and two
c-semiring homomorphisms $\varphi_1 : C \to A$ and $\varphi_2 : C \to
B$, the c-semiring homomorphism $\langle \varphi_1, \varphi_2\rangle :
C \to A \times B$ defined by $\langle \varphi_1, \varphi_2\rangle(c) =
(\varphi_1(c), \varphi_2(c))$ is unique for the property $\varphi_1 =
\langle \varphi_1, \varphi_2\rangle \mathbin{;} \pi_1$ and $\varphi_2
= \langle \varphi_1, \varphi_2\rangle \mathbin{;} \pi_2$.

\begin{lemma}
$\cSRngcat$ has finite products.\qed
\end{lemma}

\subsection{Lexicographic Products of C-Semirings}

\paragraph
The \emph{collapsing elements} $\collapseset(A)$ of a c-semiring $A$
are the collapsing elements of $\mMonfun(A)$.

A c-semiring $A$ is \emph{total} if for all $a, a_1, a_2 \in |A|$:
%
\begin{equation*}
  a_1 \prec_A a_2 \text{ or } a_1 = a_2 \text{ or } a_2 \prec_A a_1
\ \text{.}
\end{equation*}

Let $A$ and $B$ be c-semirings where $A$ is total.  Let $L = ((|A|
\setminus \collapseset(A)) \times |B|) \cup (\collapseset(A) \times \{
\mathbf{0}_B \})$.  Define ${\oplus_{A \ltimes B}}, {\otimes}_{A
  \ltimes B} : L \times L \to L$ by
%
\begin{gather*}
  (a_1, b_1) \oplus_{A \ltimes B} (a_2, b_2)
=
\begin{cases}
  (a_1, b_1) & \text{if $a_2 \prec_A a_1$}\\
  (a_2, b_2) & \text{if $a_1 \prec_A a_2$}\\
  (a_1, b_1 \oplus_B b_2) & \text{if $a_1 = a_2$}
\end{cases}
\ \text{,}\\
  (a_1, b_1) \otimes_{A \ltimes B} (a_2, b_2)
= 
  (a_1 \otimes_A a_2, b_1 \otimes_B b_2)
\ \text{.}
\end{gather*}
%
Then $(a, b) \otimes_{A \ltimes B} ((a_1, b_1) \oplus_{A \ltimes B}
(a_2, b_2)) = ((a, b) \otimes_{A \ltimes B} (a_1, b_1)) \oplus_{A
  \ltimes B} ((a, b) \otimes_{A \ltimes B} (a_2, b_2))$ for all $(a,
b), (a_1, b_1), (a_2, b_2) \in L$, where the right hand side is $(a
\otimes_A a_1, b \otimes_B b_1) \oplus_{A \ltimes B} (a \otimes a_2, b
\otimes_B b_2)$: If $a_1 = a_2$, then $(a_1, b_1) \oplus_{A \ltimes B}
(a_2, b_2) = (a_1, b_1 \oplus_B b_2)$ and $a \otimes_A a_1 = a
\otimes_A a_2$, from which the claim follows by the distributivity of
$\otimes_B$ over $\oplus_B$.  If $a_2 \prec_A a_1$ (the case $a_1
\prec_A a_2$ is symmetric), then $(a_1, b_1) \oplus_{A \ltimes B}
(a_2, b_2) = (a_1, b_1)$.  If additionally $a \notin \collapseset(A)$,
then $a \otimes_A a_2 \prec_A a \otimes_A a_1$; if $a \in
\collapseset(A)$, then $b = \mathbf{0}_B$, and in both cases the claim
follows.
 
Thus $A \ltimes B = (L, {\oplus_{A \ltimes B}}, {\otimes_{A \ltimes
    B}}, (\mathbf{0}_A, \mathbf{0}_B), (\mathbf{1}_A, \mathbf{1}_B))$
is a c-semiring, the \emph{lexicographic product} of $A$ and $B$.

\paragraph
Let $A$ be a c-semiring.  Let $Z_A$ be a subset of $|A|$ with: (1)~if
$a \otimes_A b \in Z_A$, then $a \in Z_A$ or $b \in Z_A$; and (2)~if
$a \oplus_A b \in Z_A$, then $a \in Z_A$ and $b \in Z_A$.

Let $B$ be another c-semiring.  Let $L_{Z_A} = ((|A|
\setminus Z_A) \times |B|) \cup (Z_A \times \{
\mathbf{0}_B \})$.  Define ${\oplus_{A \ltimes B}^{Z_A}}, {\otimes}_{A
  \ltimes B}^{Z_A} : L_{Z_A} \times L_{Z_A} \to L_{Z_A}$ by
%
\begin{gather*}
  (a_1, b_1) \oplus_{A \ltimes B}^{Z_A} (a_2, b_2)
=
  (a_1 \oplus_A a_2, b_1 \oplus_B b_2)
\ \text{,}\\
  (a_1, b_1) \otimes_{A \ltimes B}^{Z_A} (a_2, b_2)
= 
  (a_1 \otimes_A a_2, b_1 \otimes_B b_2)
\ \text{.}
\end{gather*}

Then $\oplus_{A \ltimes B}^{Z_A}$ and $\otimes_{A \ltimes B}^{Z_A}$
are well defined.


\section{Soft Constraints}

\paragraph
We recapitulate essential notions introduced
in~\cite{Rossi2006Handbook}.  A \emph{constraint domain} $(X, D)$ is
given by a set $X$ of \emph{variables} and a family $D = (D_x)_{x \in
  X}$ of \emph{variable domains} where each $D_x$ is a set.  A
constraint domain $(X, D)$ is \emph{finite} if $X$ and $\bigcup_{x \in
  X} D_x$ are finite.

A \emph{valuation} for a constraint domain $(X, D)$ is a dependent map
$v \in \Pi x \in X \,.\, D_x$, i.e., $v(x) \in D_x$; we abbreviate
$\Pi x \in X \,.\, D_x$ by $[X \to D]$.

A \emph{constraint} $c$ over a constraint domain $(X, D)$, or
\emph{$(X, D)$-constraint}, is given by a map $c : [X \to D] \to
\mathbb{B}$.  We also write $v \models c$ for $c(v) = \mathit{tt}$.

\paragraph
Given a constraint domain $(X, D)$ and a c-semiring $G$, a
\emph{$G$-soft constraint} $\gamma$ over $(X, D)$, or \emph{$(X,
  D)$-$G$-soft constraint}, is given by a map $\gamma : [X \to D] \to
|G|$.  In particular, a constraint over $(X, D)$ can be considered a
$\mathrm{B}$-soft constraint over $(X, D)$.

Let $\Gamma$ be a finite set of $(X, D)$-$G$-soft constraints.  For a
$v \in [X \to D]$ let the \emph{solution degree} for $\Gamma$ of $v$
be
%
\begin{equation*}\textstyle
  \Gamma(v) = \bigotimes_G \{ \gamma(v) \mid \gamma \in \Gamma \}
\ \text{.}
\end{equation*}

Define a binary relation ${\preccurlyeq_\Gamma} \subseteq [X \to D]
\times [X \to D]$ by
%
\begin{equation*}\textstyle
  w \preccurlyeq_\Gamma v
\iff
  \Gamma(w) \preceq_G \Gamma(v)
\ \text{,}
\end{equation*}
%
meaning that valuation $v$ is a \emph{better solution} for $\Gamma$
than the valuation $w$.

The \emph{maximum solution degrees} of $\Gamma$ are
given by
%
\begin{equation*}\textstyle
  \Gamma^* = \Max{\preceq_G} \{ \Gamma(v) \mid v \in [X \to D] \}
\ \text{,}
\end{equation*}
%
and the \emph{maximum solutions} by
%
\begin{equation*}
  \Max{\preccurlyeq_\Gamma} [X \to D] = \{ v \in [X \to D] \mid \Gamma(v) \in \Gamma^* \}
\ \text{.}
\end{equation*}

\paragraph\label{par:admissible}
A set of $(X, D)$-$G$-soft constraints is \emph{admissible} if
$\Gamma$ is finite and for each $v \in [X \to D]$ there is a $g \in
\Gamma^*$ such that $\Gamma(v) \preceq_G g$.

\begin{example}\label{ex:admissible}
Let $\Gamma$ be a finite set of $(X, D)$-$G$-soft constraints.

\smallskip%
(1)~If $(X, D)$ is finite, then $\Gamma$ is admissible.

\smallskip%
(2)~If $\prec_G$ has no infinite ascending chains, then $\Gamma$ is
admissible.

\smallskip%
(3)~Let $X = \{ \mathrm{x} \}$, $D_{\mathrm{x}} = [0, 1]$,
$G = ([0, 1], \max, +^1, 1, 0)$ with $r +^1 s = \min \{ 1, r+s \}$.
Let $\gamma : [X \to D] \to |G|$ be defined by
$\gamma(\{ \mathrm{x} \mapsto r \}) = r$ if $r > 0$, and
$\gamma(\{ \mathrm{x} \mapsto 0 \}) = 1$.  Let
$\Gamma = \{ \gamma \}$.  Then $\Gamma^* = \emptyset$, since the set
of solution degrees is the open interval
$\mathopen{]}0, 1\mathclose{]}$ with $0$ the optimum, which, however,
cannot be obtained.  Thus $\Gamma$ is not admissible.\qed
\end{example}

\paragraph
For a constraint domain $(X, D)$ we fix an \emph{extended} constraint
domain $(X, D^{?})$ setting $D^{?} = (D^{?}_x)_{x \in X}$ with
$D^{?}_x = D_x \uplus \{ {?} \}$, where $?$ is fresh.

A valuation $p \in \Pi x \in X \,.\, D^{?}_x = [X \to D^{?}]$ is
called a \emph{partial valuation} for $(X, D)$.

The \emph{domain of definition} $\defdom(p)$ of a partial valuation
$p$ for $(X, D)$ is the set $\{ x \in X \mid p(x) \neq {?} \}$.  For
$p, q \in [X \to D^{?}]$, we write $p \sqsubseteq q$ if $x \in
\defdom(p)$ implies $x \in \defdom(q)$ and $q(x) = p(x)$ for each $x
\in X$; by $p{\uparrow}$ we denote the set $\{ v \in [X \to D] \mid p
\sqsubseteq v \}$ of $(X, D)$-valuations.

\paragraph\label{par:maxSolDegs}
An $(X, D^{?})$-$G$-soft constraint $\omega : [X \to D^{?}] \to |G|$
is \emph{bounding} if $\omega(v) \preceq_G \omega(p)$ for all $p \in
[X \to D^{?}]$ and $v \in p{\uparrow}$.  A bounding $(X,
D^{?})$-$G$-soft constraint $\omega$ is \emph{tight} for a finite set
of $(X, D)$-$G$-soft constraints $\Gamma$ if $\omega(v) = \Gamma(v)$
for all $v \in [X \to D]$.

A pair $(\pi, \omega)$ of $(X, D^{?})$-$G$-soft constraints forms a
\emph{bounding pair} if $\omega$ is bounding and for each $p\{ x
\mapsto d \} \in [X \to D^{?}]$ there is a $v \in p{\uparrow}$ with
$\pi(p\{ x \mapsto d \}) \preceq_G \omega(v)$; a bounding pair $(\pi,
\omega)$ is \emph{tight} for a finite set of $(X, D)$-$G$-soft
constraints $\Gamma$ if $\omega$ is tight for $\Gamma$.

For a bounding pair $(\pi, \omega)$ of $(X, D^{?})$-$G$-soft
constraints the following ``branch \& bound'' algorithm
$\macit{maxSolDegs}_{(\pi, \omega)}$ computes, given a partial
valuation $p \in [X \to D^{?}]$ and a finite set of lower bounds $L
\subseteq |G|$ (which we assume to contain only elements which are
pairwise incomparable w.r.t.\ $\preceq_G$), the maximum degrees in $L
\cup \{ \omega(v) \mid v \in p{\uparrow} \}$ w.r.t.\ $\preceq_G$,
i.e., in particular, if $p = (\lambda x \in X \,.\, {?})$ and $L =
\emptyset$, the maximum degrees in $\{ \omega(v) \mid v \in [X \to D]
\}$:
%
\bgroup
\vskip\abovedisplayskip
\newcommand{\CHOOSE}{\textbf{choose}}
\newcommand{\IF}{\textbf{if}}
\newcommand{\THEN}{\textbf{then}}
\newcommand{\tHEN}{\phantom{\textbf{then}}}
\newcommand{\ELSE}{\textbf{else}}
\newcommand{\FI}{\textbf{fi}}
\newcommand{\WHILE}{\textbf{while}}
\newcommand{\FOR}{\textbf{for}}
\newcommand{\DO}{\textbf{do}}
\newcommand{\OD}{\textbf{od}}
\newcommand{\dO}{\phantom{\textbf{do}}}
\newcommand{\RETURN}{\textbf{return}}
\begin{list}{}{\leftmargin=\mathindent}\item
\textit{Assume}: -- $(X, D)$ finite constraint domain\\
\phantom{\textit{Assume}: }-- $G$ c-semiring\\
\phantom{\textit{Assume}: }-- $(\pi, \omega)$ bounding pair of $(X, D^{?})$-$G$-soft constraints\\
\textit{In}: -- $p \in [X \to D^{?}]$ partial valuation for $(X, D)$\\
\phantom{\textit{In}: }-- $L \subseteq |G|$ finite and pairwise incomparable w.r.t.\ $\preceq_G$\\
\textit{Return}: $\Max{\preceq_G} (L \cup \omega(p{\uparrow}))$\\[1.2ex]
$\macit{maxSolDegs}_{(\pi, \omega)}(p, L) \equiv{}$\\
\hspace*{0cm}\quad$\IF\ \forall x \in X \,.\, p(x) \neq {?}$\\
\hspace*{0cm}\quad$\THEN\ \RETURN\ \Max{\preceq_G} (L \cup \{ \omega(p) \})$\\
\hspace*{0cm}\quad$x \gets \CHOOSE\ \{ x \in X \mid p(x) = {?} \}$\\
\hspace*{0cm}\quad$L \gets \Max{\preceq_G} (L \cup \{ \pi(p\{ x \mapsto d \}) \mid d \in D_x \})$\\
\hspace*{0cm}\quad$\FOR\ d \in D_x$\\
\hspace*{0cm}\quad$\DO\ \IF\ \neg \exists l \in L \,.\, \omega(p\{ x \mapsto d \}) \preceq_G l$\\
\hspace*{0cm}\quad$\dO\ \THEN\ L \gets \macit{maxSolDegs}_{(\pi, \omega)}(p\{ x \mapsto d \}, L)\ \FI\ \OD$\\
\hspace*{0cm}\quad$\RETURN\ L$
\end{list}
\vskip\belowdisplayskip
\egroup

We prove the claim that
%
\begin{equation*}
  \macit{maxSolDegs}_{(\pi, \omega)}(p, L) = \Max{\preceq_G} (L \cup \omega(p{\uparrow}))
\end{equation*}
%
by a first induction on the cardinality $n$ of $\{ x \in X \mid p(x) =
{?}  \}$.  If $n = 0$, i.e., $p \in [X \to D]$, then
$\macit{maxSolDegs}_{(\pi, \omega)}(p, L) = \Max{\preceq_G} (L \cup \{
\omega(p) \})$ and $\{ \omega(p) \} = \omega(p{\uparrow})$.  If $n >
0$, then let $x \in X$ with $p(x) = {?}$, and let $d_1, \ldots, d_r$
be an enumeration of $D_x$.  Let $P = \{ \pi(p\{ x \mapsto d_i \})
\mid 1 \leq i \leq r \}$ and define
%
\begin{equation*}
  L_0 = \Max{\preceq_G} (L \cup P)
\ \text{,}
\end{equation*}
%
and inductively
%
\begin{equation*}
  L_k =
\begin{cases}
  L_{k-1} & \text{if $\exists l \in L_{k-1} \,.\, \omega(p\{ x \mapsto d_k \}) \preceq_G l$}
\\
  \macit{maxSolDegs}_{\omega}(p\{ x \mapsto d_k \}, L_{k-1}) & \text{otherwise}
\end{cases}
\end{equation*}
%
for $1 \leq k \leq r$.  We prove the sub-claim that
%
\begin{equation*}\textstyle
  L_k = \Max{\preceq_G} (L \cup P \cup \bigcup_{1 \leq j \leq k} \omega(p\{ x \mapsto d_j \}{\uparrow}))
\end{equation*}
%
for all $0 \leq k \leq r$ by a second induction on $k$: For $k = 0$,
$L_0 = \Max{\preceq_G} (L \cup P)$ by definition.  For $k > 0$, let
there first be an $l \in L_{k-1}$ with $\omega(p\{ x \mapsto d_k \})
\preceq_G l$.  Since $\omega(v) \preceq_G \omega(p\{ x \mapsto d_k \})
\preceq_G l$ for all $v \in p\{ x \mapsto d_k \}{\uparrow}$ and $l \in
L_{k-1} = \Max{\preceq_G} (L \cup P \cup \bigcup_{1 \leq i \leq k-1}
\omega(p\{ x \mapsto d_i \}{\uparrow}))$ by the second induction
hypothesis, the sub-claim follows.  Otherwise, if no such $l \in
L_{k-1}$ exists, $\macit{maxSolDegs}_{\omega}(p\{ x \mapsto d_k \},
L_{k-1}) = \Max{\preceq_G} (L_{k-1} \cup \omega(p\{ x \mapsto d_k
\}{\uparrow}))$ by the first induction hypothesis, which is applicable
since, by the second induction hypothesis, $L_{k-1} = \Max{\preceq_G}
(L \cup P \cup \bigcup_{1 \leq i \leq k-1} \omega(p\{ x \mapsto d_i
\}{\uparrow}))$, and hence $L_{k-1}$ is pairwise incomparable w.r.t.\
$\preceq_G$; therefore,
%
\begin{gather*}
\textstyle
  L_k
=
  \macit{maxSolDegs}_{\omega}(p\{ x \mapsto d_k \}, L_{k-1})
={}\\\textstyle\qquad
  \Max{\preceq_G} (L_{k-1} \cup \omega(p\{ x \mapsto d_k \}{\uparrow}))
={}\\\textstyle\qquad
  \Max{\preceq_G} ((\Max{\preceq_G} (L \cup P \cup \bigcup_{1 \leq i \leq k-1} \omega(p\{ x \mapsto d_i \}{\uparrow}))) \cup \omega(p\{ x \mapsto d_k \}{\uparrow}))
={}\\\textstyle\qquad
  \Max{\preceq_G} (L \cup P \cup \bigcup_{1 \leq i \leq k} \omega(p\{ x \mapsto d_i \}{\uparrow}))
\ \text{,}
\end{gather*}
%
which establishes the sub-claim.  Thus,
$L_r = \Max{\preceq_G} (L \cup P
\cup \bigcup_{1 \leq i \leq r} \omega(p\{ x \mapsto d_i \}{\uparrow}))
= \Max{\preceq_G} (L \cup \omega(p{\uparrow})) = \macit{maxSolDegs}_{\omega}(p, L)$, since $d_1, \ldots,
d_r$ is an enumeration of $D_x$ and for each $1 \leq i \leq r$ there
is a $v \in p{\uparrow}$ with $\pi(p\{ x \mapsto d_i \}) \preceq_G
\omega(v)$, which yields the claim.

In particular, if $(\Gamma_{?}, \Gamma^{?})$ is a tight bounding pair of $(X,
D^{?})$-$G$-soft constraints for a finite set of $(X, D)$-$G$-soft
constraints $\Gamma$, then
%
\begin{gather*}
  \macit{maxSolDegs}_{(\Gamma_{?}, \Gamma^{?})}(\lambda x \in X \,.\, {?}, \emptyset)
=
  \Max{\preceq_G} (\emptyset \cup \Gamma^{?}((\lambda x \in X \,.\, {?}){\uparrow}))
={}\\\qquad
  \Max{\preceq_G} \{ \Gamma^{?}(v) \mid v \in [X \to D] \}
=
  \Max{\preceq_G} \{ \Gamma(v) \mid v \in [X \to D] \}
=
  \Gamma^*
\ \text{.}
\end{gather*}

\paragraph
Given a meet monoid $M$ and a constraint domain $(X, D)$, an
\emph{$M$-soft constraint} $\mu$ over $(X, D)$, or \emph{$(X,
  D)$-$M$-soft constraint}, is given by a map $\mu : [X \to D] \to
|M|$.  Each $(X, D)$-$M$-soft constraint induces an $(X,
D)$-$\cSRngfree{M}$-soft constraint $\cSRngetaat{M} \compfun \mu$
(viz., $(\cSRngetaat{M} \compfun \mu)(v) = \{ \mu(v) \}$).

Let $\mathsf{M}$ be a finite set of $(X, D)$-$M$-soft constraints for
a meet monoid $M$.  Then
%
\begin{equation*}\textstyle
  \bigotimes_{\cSRngfree{M}} \{ \cSRngetaat{M}(\mu(v)) \mid \mu \in \mathsf{M} \}
=
  \{ \prod_M \{ \mu(v) \mid \mu \in \mathsf{M} \} \}
\ \text{,}
\end{equation*}
%
and thus
%
\begin{gather*}\textstyle
  w \preccurlyeq_{\cSRngetaat{M} \compfun \mathsf{M}} v
\iff
  \{ \prod_M \{ \mu(w) \mid \mu \in \mathsf{M} \} \} \preceq_{\cSRngfree{M}} \{ \prod_M \{ \mu(v) \mid \mu \in \mathsf{M} \} \}  
\iff{}\\\qquad\textstyle
  \prod_M \{ \mu(w) \mid \mu \in \mathsf{M} \} \leq_M \prod_M \{ \mu(v) \mid \mu \in \mathsf{M} \} 
\ \text{.}
\end{gather*}
%
We write $\cSRngfree{\mathsf{M}}$ for the set of $(X,
D)$-$\cSRngfree{M}$-soft constraints $\{ \cSRngetaat{M} \compfun \mu
\mid \mu \in \mathsf{M} \}$.

In analogy to the notions for soft constraints, we define the
\emph{solution degree} for a $v \in [X \to D]$ by
%
\begin{equation*}\textstyle
\textstyle
  \mathsf{M}(v) = \prod_M \{ \mu(v) \mid \mu \in \mathsf{M} \}
\ \text{;}
\end{equation*}
%
the \emph{better solution} relation ${\lesssim_{\mathsf{M}}} \subseteq [X \to
D] \times [X \to D]$ by
%
\begin{equation*}\textstyle
  w \lesssim_{\mathsf{M}} v
\iff
  \mathsf{M}(w) \leq_M \mathsf{M}(v)
\ \text{;}
\end{equation*}
%
the \emph{maximum solution degrees} by
%
\begin{equation*}
  \mathsf{M}^* = \Max{\leq_M} \{ \mathsf{M}(v) \mid v \in [X \to D] \}
\ \text{;}
\end{equation*}
%
and the \emph{maximum solutions} by
%
\begin{equation*}
  \Max{\lesssim_{\mathsf{M}}} [X \to D] = \{ v \in [X \to D] \mid \mathsf{M}(v) \in \mathsf{M}^* \}
\ \text{.}
\end{equation*}

A set $\mathsf{M}$ of $(X, D)$-$M$-soft constraints for a meet monoid
$M$ is \emph{admissible} if it is finite and for all $v \in [X \to D]$
there is an $m \in \mathsf{M}^*$ such that $\mathsf{M}(v) \leq_M m$.

\paragraph\label{par:meetMonoidMaxSolDegs}
The algorithm $\macit{maxSolDegs}_{(\pi, \omega)}$ from
§\ref{par:maxSolDegs} for a bounding pair $(\pi, \omega)$ of $(X,
D^{?})$-$G$-soft constraints for a c-semiring $G$ in fact also works
under the assumptions that $(X, D)$ is a constraint domain, $M$ is a
meet monoid, and $\alpha$ and $\zeta$ are $(X, D^{?})$-$M$-soft
constraints, such that $(\cSRngetaat{M} \compfun \alpha,
\cSRngetaat{M} \compfun \zeta)$ is a bounding pair.  We call $(\alpha,
\zeta)$ itself a \emph{bounding pair} if $\zeta(v) \leq_M \zeta(p)$
for all $p \in [X \to D^{?}]$ and $v \in p{\uparrow}$, and if for each
$p\{ x \mapsto d \} \in [X \to D^{?}]$ there is a $v \in p{\uparrow}$
such that $\alpha(p\{ x \mapsto d \}) \leq_M \zeta(v)$.
%
\bgroup
\vskip\abovedisplayskip
\newcommand{\CHOOSE}{\textbf{choose}}
\newcommand{\IF}{\textbf{if}}
\newcommand{\THEN}{\textbf{then}}
\newcommand{\tHEN}{\phantom{\textbf{then}}}
\newcommand{\ELSE}{\textbf{else}}
\newcommand{\FI}{\textbf{fi}}
\newcommand{\WHILE}{\textbf{while}}
\newcommand{\FOR}{\textbf{for}}
\newcommand{\DO}{\textbf{do}}
\newcommand{\OD}{\textbf{od}}
\newcommand{\dO}{\phantom{\textbf{do}}}
\newcommand{\RETURN}{\textbf{return}}
\begin{list}{}{\leftmargin=\mathindent}\item
\textit{Assume}: -- $(X, D)$ finite constraint domain\\
\phantom{\textit{Assume}: }-- $M$ meet monoid\\
\phantom{\textit{Assume}: }-- $(\alpha, \zeta)$ bounding pair of $(X, D^{?})$-$M$-soft constraints\\
\textit{In}: -- $p \in [X \to D^{?}]$ partial valuation for $(X, D)$\\
\phantom{\textit{In}: }-- $L \subseteq |M|$ finite and pairwise incomparable w.r.t.\ $\leq_M$\\
\textit{Return}: $\Max{\leq_M} (L \cup \zeta(p{\uparrow}))$\\[1.2ex]
$\macit{maxSolDegs}_{(\alpha, \zeta)}(p, L) \equiv{}$\\
\hspace*{0cm}\quad$\IF\ \forall x \in X \,.\, p(x) \neq {?}$\\
\hspace*{0cm}\quad$\THEN\ \RETURN\ \Max{\leq_M} (L \cup \{ \zeta(p) \})\ \FI$\\
\hspace*{0cm}\quad$x \gets \CHOOSE\ \{ x \in X \mid p(x) = {?} \}$\\
\hspace*{0cm}\quad$L \gets \Max{\leq_M} (L \cup \{ \alpha(p\{ x \mapsto d \}) \mid d \in D_x \})$\\
\hspace*{0cm}\quad$\FOR\ d \in D_x$\\
\hspace*{0cm}\quad$\DO\ \IF\ \neg \exists l \in L \,.\, \zeta(p\{ x \mapsto d \}) \leq_M l$\\
\hspace*{0cm}\quad$\dO\ \THEN\ L \gets \macit{maxSolDegs}_{(\alpha, \zeta)}(p\{ x \mapsto d \}, L)\ \FI\ \OD$\\
\hspace*{0cm}\quad$\RETURN\ L$
\end{list}
\vskip\belowdisplayskip
\egroup

Note that $\macit{maxSolDegs}_{(\alpha, \zeta)}(p, L) =
L \oplus_{\cSRngfree{M}} \Max{\leq_M} \zeta(p{\uparrow})$.

A bounding pair $(\alpha, \zeta)$ of $(X, D^{?})$-$M$-soft constraints
is \emph{tight} for a finite set of $(X, D)$-$M$-soft constraints
$\mathsf{M}$ if $\zeta(v) = \mathsf{M}(v)$ for all $v \in [X \to D^{?}]$.  For a
tight bounding pair $(\mathsf{M}_{?}, \mathsf{M}^{?})$ for a finite set of $(X,
D)$-$M$-soft constraints $\mathsf{M}$ we again obtain
%
\begin{equation*}
  \macit{maxSolDegs}_{(\mathsf{M}_{?}, \mathsf{M}^{?})}(\lambda x \in X \,.\, ?, \emptyset)
=
  \Max{\leq_M} \{ \mathsf{M}(v) \mid v \in [X \to D] \}
=
  \mathsf{M}^*
\ \text{.}
\end{equation*}


\section{Constraint Hierarchies}

\paragraph
A \emph{constraint hierarchy}~\cite{borning1992hierarchies} $H = (C_k)_{1 \leq k \leq n}$ over a
constraint domain $(X, D)$, or \emph{$(X, D)$-constraint hierarchy},
is given by a family of sets of $(X, D)$-constraints.  The constraints
in \emph{level} $1 \leq k \leq n$ are considered as \emph{strictly
  more important} than the constraints in level $k+1$.  An $(X,
D)$-constraint hierarchy is \emph{finite} if $\bigcup_{1 \leq k \leq
  n} C_k$ is finite.

Let $H = (C_k)_{1 \leq k \leq n}$ be a finite $(X, D)$-constraint
hierarchy, let $L = (M_i)_{1 \leq k \leq n}$ be a corresponding family
of meet monoids, and let for each $1 \leq k \leq n$ and each $c \in
C_k$, $\mu(c)$ be an $(X, D)$-$M_k$-soft constraint.  We call $\mathsf{H} =
(\mathsf{M}_k)_{1 \leq k \leq n}$ with $\mathsf{M}_k = \{ \mu(c) \mid c \in C_k \}$
for $1 \leq k \leq n$ an \emph{$(X, D)$-$L$-soft constraint hierarchy}.
For a $v \in [X \to D]$ the \emph{solution degree} for $(\mathsf{M}_k)_{1
  \leq k \leq n}$ of $v$ is defined to be $(\mathsf{M}_k(v))_{1 \leq k \leq
  n}$.  Define a binary relation ${<_{\mathsf{H}}} \subseteq [X \to D]
\times [X \to D]$ by
%
\begin{equation*}
  w <_{\mathsf{H}} v
\iff
  \exists 1 \leq k \leq n \,.\, (\forall 1 \leq i \leq k-1 \,.\, \mathsf{M}_i(w) = \mathsf{M}_i(v)) \land \mathsf{M}_k(w) <_{M_k} \mathsf{M}_k(v)
\ \text{,}
\end{equation*}
%
saying that the valuation $v$ is \emph{strictly better} than the
valuation $w$, and denote its reflexive closure on $[X \to D]$ by
$\leq_{\mathsf{H}}$, which is the lexicographic order on the set $\{
(\mathsf{M}_k(v))_{1 \leq k \leq n} \mid v \in [X \to D] \}$.  In particular,
%
\begin{equation*}
  w <_{\mathsf{H}} v \iff (\mathsf{M}_k(w))_{1 \leq k \leq n} <_{M_1 \ltimes \ldots \ltimes M_n} (\mathsf{M}_k(v))_{1 \leq k \leq n}
\end{equation*}
%
if, on the one hand, every $M_k$ is a bounded meet monoid for all $2
\leq k \leq n$, and, on the other hand, $\mathsf{M}_k(v), \mathsf{M}_k(w) \notin
\collapseset(M_k)$ for all $1 \leq k \leq n$, or, equivalently, if
$\mu(c)(v), \mu(c)(w) \notin \collapseset(M_k)$ for each $c \in C_k$,
$1 \leq k \leq n$.  The first requirement, that each $M_k$ is bounded,
can be achieved by moving from $M_k$ to its lifted variant
$(M_k)_{\bot}$.

\paragraph
Let $(X, D)$ be a constraint domain, and $M$ and $N$ meet monoids.
Let $\mathsf{M}$ be a finite set of $(X, D)$-$M$-soft constraints and let
$\mathsf{N}$ be a finite set of $(X, D)$-$N$-soft constraints.

We say that $\mathsf{M}$ and $\mathsf{N}$ are \emph{optima equivalent},
written as $\mathsf{M} \approx \mathsf{N}$, if
$\Max{\lesssim_{\mathsf{M}}} [X \to D] = \Max{\lesssim_{\mathsf{N}}} [X
\to D]$.
A sufficient criterion for optima equivalence is that
$\mathsf{M}(w) \leq_M \mathsf{M}(v)$ if, and only if,
$\mathsf{N}(w) \leq_N \mathsf{N}(v)$ for all $v, w \in [X \to D]$.

$\mathsf{N}$ \emph{optima simulates} $\mathsf{M}$, written as
$\mathsf{M} \preccurlyeq \mathsf{N}$, if for each
$v_{\mathsf{M}} \in \Max{\lesssim_{\mathsf{M}}} [X \to D]$ there is a
$v_{\mathsf{N}} \in \Max{\lesssim_{\mathsf{N}}} [X \to D]$ with
$\mathsf{M}(v_{\mathsf{M}}) = \mathsf{M}(v_{\mathsf{N}})$, and, vice
versa, if for each
$v_{\mathsf{N}} \in \Max{\lesssim_{\mathsf{N}}} [X \to D]$ there is a
$v_{\mathsf{M}} \in \Max{\lesssim_{\mathsf{M}}} [X \to D]$ with
$\mathsf{M}(v_{\mathsf{M}}) = \mathsf{M}(v_{\mathsf{N}})$.  Obviously,
$\mathsf{M} \approx \mathsf{N}$ if, and only if,
$\mathsf{M} \preccurlyeq \mathsf{N}$ and
$\mathsf{N} \preccurlyeq \mathsf{M}$.

\begin{lemma}\label{lem:simulation}
Let $(X, D)$ be a constraint domain, and let $M$ and $N$ be meet
monoids.  Let $\mathsf{M}$ be an admissible set of $(X, D)$-$M$-soft
constraints and $\mathsf{N}$ an admissible set of $(X, D)$-$N$-soft
constraints such that
%
\begin{gather*}
  \mathsf{M}(v) <_M \mathsf{M}(v')
\quad\text{implies}\quad
  \mathsf{N}(v) <_N \mathsf{N}(v')
\\
  \mathsf{M}(v) \parallel_M \mathsf{M}(v')
\quad\text{implies}\quad
  \mathsf{N}(v) \parallel_N \mathsf{N}(v')
\end{gather*}
%
for all $v, v' \in [X \to D]$.  Then $\mathsf{M} \preccurlyeq \mathsf{N}$.
\end{lemma}
\begin{proof}
Let first $v_{\mathsf{M}} \in \Max{\lesssim_{\mathsf{M}}} [X \to D]$.
Let $v_{\mathsf{M}} \notin \Max{\lesssim_{\mathsf{N}}} [X \to D]$.
Then, since $\mathsf{N}$ is admissible, there is a $v_{\mathsf{N}} \in
\Max{\lesssim_{\mathsf{N}}} [X \to D]$ with
$\mathsf{N}(v_{\mathsf{M}}) <_N \mathsf{N}(v_{\mathsf{N}})$.
Moreover, there is a $v_{\mathsf{M}}' \in \Max{\lesssim_{\mathsf{M}}}
[X \to D]$ with $\mathsf{M}(v_{\mathsf{N}}) \leq_M
\mathsf{M}(v_{\mathsf{M}}')$, since $\mathsf{M}$ is admissible.  But
$\mathsf{M}(v_{\mathsf{N}}) <_M \mathsf{M}(v_{\mathsf{M}}')$ is
impossible, since then also $\mathsf{N}(v_{\mathsf{N}}) <_N
\mathsf{N}(v_{\mathsf{M}}')$ contradicting $\mathsf{N}(v_{\mathsf{N}})
\in \mathsf{N}^*$.  Thus $\mathsf{M}(v_{\mathsf{N}}) =
\mathsf{M}(v_{\mathsf{M}}')$.  Moreover, either
$\mathsf{M}(v_{\mathsf{M}}) \parallel_M \mathsf{M}(v_{\mathsf{M}}')$
or $\mathsf{M}(v_{\mathsf{M}}) = \mathsf{M}(v_{\mathsf{M}}')$ since
both $\mathsf{M}(v_{\mathsf{M}})$ and $\mathsf{M}(v_{\mathsf{M}}')$
are elements of $\mathsf{M}^*$.  But
$\mathsf{M}(v_{\mathsf{M}}) \parallel_M \mathsf{M}(v_{\mathsf{M}}')$
is impossible, since we would have
$\mathsf{M}(v_{\mathsf{M}}) \parallel_M \mathsf{M}(v_{\mathsf{N}}) =
\mathsf{M}(v_{\mathsf{M}}')$ and $\mathsf{N}(v_{\mathsf{M}}) <_N
\mathsf{N}(v_{\mathsf{N}})$.  Thus $\mathsf{M}(v_{\mathsf{N}}) =
\mathsf{M}(v_{\mathsf{M}}') = \mathsf{M}(v_{\mathsf{M}})$.

Now let $v_{\mathsf{N}} \in \Max{\lesssim_{\mathsf{N}}} [X \to D]$. If
$v_{\mathsf{N}} \notin \Max{\lesssim_{\mathsf{M}}} [X \to D]$, there
would be, since $\mathsf{M}$ is admissible, a $v_{\mathsf{M}} \in
\Max{\lesssim_{\mathsf{M}}} [X \to D]$ such that
$\mathsf{M}(v_{\mathsf{N}}) <_M \mathsf{M}(v_{\mathsf{M}})$, i.e.\
$\mathsf{N}(v_{\mathsf{N}}) <_N \mathsf{N}(v_{\mathsf{M}})$,
contradicting $\mathsf{N}(v_{\mathsf{N}}) \in \mathsf{N}^*$.
\end{proof}

From the conditions of the lemma it follows that
$\mathsf{N}(v) \leq_N \mathsf{N}(v')$ implies
$\mathsf{M}(v) \leq_M \mathsf{M}(v')$ for all $v, v' \in [X \to D]$.
Indeed, let the conditions of the lemma hold and let
$\mathsf{N}(v) \leq_N \mathsf{N}(v')$ but
$\mathsf{M}(v) \not\leq_M \mathsf{M}(v')$.  Then either
$\mathsf{M}(v) >_M \mathsf{M}(v')$ or
$\mathsf{M}(v) \parallel_M \mathsf{M}(v')$, implying
$\mathsf{N}(v) >_N \mathsf{N}(v')$ or
$\mathsf{N}(v) \parallel_N \mathsf{N}(v')$ which both contradict
$\mathsf{N}(v) \leq_N \mathsf{N}(v')$.  Moreover, from the requirement
that $\mathsf{N}(v) \leq_N \mathsf{N}(v')$ implies
$\mathsf{M}(v) \leq_M \mathsf{M}(v')$ for all $v, v' \in [X \to D]$ it
follows that $\mathsf{M}(v) \parallel_M \mathsf{M}(v')$ implies
$\mathsf{N}(v) \parallel_N \mathsf{N}(v')$ for all $v, v' \in [X \to
D]$.  Thus, the conditions of the lemma can be equivalently replaced by
%
\begin{gather*}
  \mathsf{M}(v) <_M \mathsf{M}(v')
\quad\text{implies}\quad
  \mathsf{N}(v) <_N \mathsf{N}(v')
\\
  \mathsf{N}(v) \leq_N \mathsf{N}(v')
\quad\text{implies}\quad
  \mathsf{M}(v) \leq_M \mathsf{M}(v')
\end{gather*}
%
for all $v, v' \in [X \to D]$.

(S.~Bistarelli, Ph.~Codognet, and F.~Rossi discuss abstractions of
c-semiring-based soft constraint problems by means of Galois
connections~\cite{bistarelli-codognet-rossi:ai:2002}.)

\paragraph
Let $C$ be a finite set of $(X, D)$-constraints. The
\emph{locally"=predicate"=better} (\emph{LPB}) \emph{level comparator}
for $C$ corresponds to requiring
%
\begin{equation*}
  w <_C^{\mathrm{LPB}} v \iff \{ c \in C \mid v \not\models c \} \subset \{ c \in C \mid w \not\models c \}
\ \text{.}
\end{equation*}

This can be expressed by choosing the meet monoid $M = (\finsets(C),
{\cup}, \emptyset, \supseteq)$ and the set of $(X, D)$-$M$-soft
constraints $\mathsf{M} = \{ \mu(c) \mid c \in C \}$ with $\mu(c)(v) = \{ c
\}$ if $v \not\models c$ and $\mu(c)(v) = \emptyset$ otherwise, for
each $c \in C$.  However, all elements of $M$ are idempotent, and thus
the collapsing elements of $M$ are $\finsets(C) \setminus \{ \emptyset
\}$.  Hence, $M$ is not suitable for a lexicographic product.

Choosing instead the meet monoid $N = (\finmsets(C), {\mcup}, \lbag
\rbag, \supmseteq)$ which has no collapsing elements and the set of
$(X, D)$-$N$-soft constraints $\mathsf{N} = \{ \nu(c) \mid c
\in C \}$ with $\nu(c)(v) = \lbag c \rbag$ if $v \not\models c$ and
$\nu(c)(v) = \lbag \rbag$ otherwise, for each $c \in C$, deviates this
situation, since we have
%
\begin{equation*}
  \mathsf{M}(w) \leq_M \mathsf{M}(v) \iff \mathsf{N}(w) \leq_N \mathsf{N}(v)
\end{equation*}
%
for all $v, w \in [X \to D]$.  Thus $\mathsf{M} \approx \mathsf{N}$,
i.e., $\mathsf{M}$ and $\mathsf{N}$ are optima equivalent.

\paragraph
A \emph{real} meet monoid $R$ has $0 \in |R| \subseteq \RZp$ for its
underlying set, has $0$ as its neutral element, and the (inverted)
usual ordering on the real numbers $\geq$ as its ordering.

If $|R| = \RZp$ and $(t \cdot r) \cdot_R (t \cdot s) = t \cdot (r
\cdot_R s)$ holds in the real meet monoid $R$ for all $r, s, t \in
\RZp$ (where $\cdot$ is the usual multiplication), then, according to
a theorem by H.~F.\ Bohnenblust~\cite{bohnenblust:duke:1940}, either
$1 \cdot_R 1 = 1$ and $r \cdot_R s = \max\{ r, s \}$ for all $r, s \in
\RZp$, or $1 \cdot_R 1 > 1$ and $r \cdot_R s = (r^p + s^p)^{1/p}$ for
all $r, s \in \RZp$ for some $p > 0$.

Thus, the following are examples of real meet monoids:
%
\begin{itemize}[label={--},leftmargin=*]
  \item \emph{Weighted sum}: $\mathrm{R}_1 = (\RZp, {\cdot_1}, 0,
{\geq})$ with $r \cdot_1 s = r + s$;

  \item \emph{Least squares}: $\mathrm{R}_2 = (\RZp, {\cdot_2}, 0,
{\geq})$ with $r \cdot_2 s = \sqrt{r^2 + s^2}$;

  \item \emph{$p$-norm} for $p > 0$: $\mathrm{R}_p = (\RZp,
{\cdot_p}, 0, {\geq})$ with $r \cdot_p s = (r^p + s^p)^{1/p}$;

  \item \emph{Worst case}: $\mathrm{R}_{\infty} = (\RZp, {\cdot_{\infty}}, 0,
{\geq})$ with $r \cdot_{\infty} s = \max\{ r, s \}$.
\end{itemize}
%
The notation $\mathrm{R}_{\infty}$ is justified by the well-known fact
that $\lim_{p \to \infty} (r^p + s^p)^{1/p} = \max\{ r, s \}$.

For a $V \subseteq \RZp$ and $p > 0$, let $\langle V\rangle_p$ be the
smallest subset of $\RZp$ such that $0 \in \langle V\rangle_p$ and $r
\cdot_p s \in \langle V\rangle_p$ if $r, s \in \langle V\rangle_p$.
Then we obtain a real meet monoid $(\langle V\rangle_p, {\cdot_p}, 0,
{\geq})$. For a $V \subseteq \RZp$, let $V_{\infty}$ denote the meet
monoid $(V \cup \{ 0 \}, {\cdot_{\infty}}, 0, {\geq})$, and let
$V_{p}$ denote the meet monoids $(\langle V\rangle_p, {\cdot_p}, 0,
{\geq})$ for $p > 0$.

All real meet monoids $R$ with ${\cdot_R} = {\cdot_p}$ for some $p >
0$ have no collapsing elements, since $r \cdot_p s = (r^p +
s^p)^{1/p}$ is strictly monotonic in both arguments.  For real meet
monoids with ${\cdot_R} = {\cdot_{\infty}}$, however, $\collapseset(R)
= |R| \setminus \{ 0 \}$, since $\cdot_{\infty}$ is idempotent.

\paragraph\label{par:max-norm}
Let $0 \in V \subseteq \RZp$, $M$ a meet monoid, and $\tau :
V_{\infty} \to M$ a meet monoid homomorphism.  For a $\vec{r} =
(r_i)_{1 \leq i \leq n} \in V^n$, we write $\tau(\vec{r})$ for
$(\tau(r_i))_{1 \leq i \leq n}$; for an $m \in |M|$, we write
$m^{(n)_M}$ for the $n$-fold product of $m$ w.r.t.\ $\cdot_M$.  For
each $n \geq 1$ we have
%
\begin{equation}\tag{$\ast$}\label{eq:max-norm-1}\textstyle
  \prod_{\infty} \vec{r} < \prod_{\infty} \vec{s}
\quad\text{implies}\quad
  \prod_{M} \tau(\vec{r}) >_M \prod_{M} \tau(\vec{s})
\quad\text{for all $\vec{r}, \vec{s} \in V^n$}
\end{equation}
%
if, and only if,
%
\begin{equation}\tag{$\ast\ast$}\label{eq:max-norm-2}\textstyle
  r < s
\quad\text{implies}\quad
  \tau(r)^{(n)_M} >_M \tau(s)
\quad\text{for all $r, s \in V.$}
\end{equation}
%
Indeed, let first \eqref{eq:max-norm-1} hold and let $r, s \in V$ with
$r < s$.  Choose $r_1 = \ldots = r_n = r$, $s_1 = \ldots = s_{n-1} =
0$, and $s_n = s$.  Then $\prod_{\infty} (r_i)_{1 \leq i \leq n} = r <
s = \prod_{\infty} (s_i)_{1 \leq i \leq n}$, and thus $\tau(r)^{(n)_M} =
\prod_{M} \tau((r_i)_{1 \leq i \leq n}) >_M \prod_{M} \tau((s_i)_{1
  \leq i \leq n}) = \tau(s)$ since $\tau(s_i) = \varepsilon_M$ if $1 \leq i < n$
and $\tau(s_n) = \tau(s)$. --- Now, let \eqref{eq:max-norm-2} hold
and let $r = \prod_{\infty} (r_i)_{1 \leq i \leq n} < \prod_{\infty}
(s_i)_{1 \leq i \leq n} = s$.  Define $r_1' = \ldots = r_n' = r$ and
$s_1' = \ldots = s_{n-1}' = 0$, $s_n' = s$.  Then $\prod_M
\tau((r_i)_{1 \leq i \leq n}) \geq_M \prod_M \tau((r_i')_{1 \leq i
  \leq n}) = \tau(r)^{(n)_M}$, since $\tau(r_i) \geq_M \tau(r)$ for all
$1 \leq i \leq n$, and $\tau(s) = \prod_M \tau((s_i')_{1 \leq i \leq
  n}) \geq_M \prod_M \tau((s_i)_{1 \leq i \leq n})$, since $\tau(0)
\geq_M \tau(s_i)$ for all $1 \leq i \leq n$.  Then $\prod_M
\tau((r_i)_{1 \leq i \leq n}) \geq_M \tau(r)^{(n)_M} >_M \tau(s) \geq_M
\prod_M \tau((s_i)_{1 \leq i \leq n})$.

Now \eqref{eq:max-norm-2} yields some restrictions on useful choices
of $V$, $M$, and $\tau$ when it comes to lexicographic products.  We
have $\tau(r) \geq_M \tau(r)^{(n)_M}$.  If $\tau(r) =
\tau(r)^{(n)_M}$, then $\tau(r)$ is idempotent, since $\tau(r) \geq_M
\tau(r) \cdot_M \tau(r) = \tau(r)^{(2)_M} \geq_M \ldots \geq_M
\tau(r)^{(n)_M} = \tau(r)$.  If $\tau(r) >_M \tau(r)^{(n)_M}$, then
there must be no $r' \in V$ with $r < r'$ and $\tau(r) \geq_M \tau(r')
\geq_M \tau(r)^{(n)_M}$, since otherwise, by choosing $r'$ for $s$ in
\eqref{eq:max-norm-2}, $\tau(r') \geq_M \tau(r)^{(n)_M} >_M \tau(r')$
would have to hold.

\paragraph
Consider $\overline{\mathrm{R}} = (\finmsets(\RZpos), {\mcup},
\lbag\rbag, {\sqsubseteq})$ with $T \sqsubseteq U$ if, and only if, 
there is a $q > 0$ such that $\prod_p T \geq \prod_p U$ for all
$p > q$ (where $\prod_p \lbag\rbag = 0$ and $\prod_p (T \mcup \lbag r
\rbag) = r \cdot_p \prod_p T$).

Then $\sqsubseteq$ is a partial order, where reflexivity and
transitivity are obvious, and we thus only have to demonstrate
antisymmetry: Let $T, U \in \finmsets(\RZpos)$ with $T \sqsubseteq U$
and $U \sqsubseteq T$.  Then there is a $q_{TU} > 0$ such that
$\prod_p T \geq \prod_p U$ for all $p > q_{TU}$, and a $q_{UT} > 0$
such that $\prod_p U \geq \prod_p T$ for all $p > q_{UT}$.  Hence
$\prod_p T = \prod_p U$ for all $p > \max \{ q_{TU}, q_{UT} \}$.
Since $\lim_{p \to \infty} \prod_p T = \prod_{\infty} T$, we either
have that $T = \lbag\rbag = U$ or that there is an $r \in \RZpos$ with
$\max T = r = \max U$.  In the latter case, with $T = T' \mcup \lbag r
\rbag$, $U = U' \mcup \lbag r \rbag$, we have $r \cdot_p \prod_p T' =
r \cdot_p \prod_p U'$ for all $p > \max \{ q_{TU}, q_{UT} \}$ and thus
$\prod_p T' = \prod_p U'$ for all $p > \max \{ q_{TU}, q_{UT} \}$.
Therefore, $T = U$ follows by induction on the size of $T$.

Furthermore, if $T \sqsubseteq U$, then $T \mcup V \sqsubseteq U \mcup
V$: Let $\prod_p T \geq \prod_p U$ for all $p > q$ for $q > 0$.  Then
$\prod_p (T \mcup V) = (\prod_p T) \cdot_p (\prod_p V) \geq (\prod_p
U) \cdot_p (\prod_p V) = \prod_p (U \mcup V)$ for all $p > q$, since
$\cdot_p$ is (strictly) monotonic in both arguments.

Summing up, $\overline{\mathrm{R}}$ is a meet monoid.  Define
$\tau : |\mathrm{R}_{\infty}| \to |\overline{\mathrm{R}}|$ by
$\tau(r) = \lbag r\rbag$ for $r \neq 0$ and $\tau(0) = \lbag\rbag$;
then $\tau : \mathrm{R}_{\infty} \to \overline{\mathrm{R}}$ is a meet
monoid homomorphism.  Moreover, let $r, s \in \RZp$ with $r < s$, and
let $n \geq 1$.  If $r = 0$, then
$\tau(s) = \lbag s \rbag \sqsubset \lbag\rbag =
\tau(r)^{(n)_{\overline{\mathrm{R}}}}$.
If $r \neq 0$, then there is a $q > 0$ such that $s > n^{1/p} \cdot r$
for all $p > q$, and hence
$\tau(s) = \lbag s \rbag \sqsubset \lbag n r \rbag =
\tau(r)^{(n)_{\overline{\mathrm{R}}}}$.
Thus \eqref{eq:max-norm-2} holds for all $n \geq 1$.

In fact, checking whether $T \sqsubseteq U$ need not involve any
$p$-norms: Let $T, U \in \finmsets(\RZpos)$ be given.  If $U =
\lbag\rbag$, then $T \sqsubseteq U$, and if $T = \lbag\rbag$, then $T
\sqsupseteq U$.  Thus we are left with the case that $T \neq
\lbag\rbag$ and $U \neq \lbag\rbag$; in particular, $\max T > 0$ and
$\max U > 0$.  If $\max T > \max U$, then $T \sqsubset U$, since then
$\lim_{p \to \infty} \prod_p T = \max T > \max U = \lim_{p \to \infty}
\prod_p U$; conversely, if $\max T < \max U$, then $T \sqsupset U$.
Hence, we are now left with the case that $\max T = \max U$.  But then
we proceed recursively for $T'$ and $U'$ with $T = \lbag \max T \rbag
\mcup T'$ and $U = \lbag \max U \rbag \mcup U'$.

However, a soft constraint problem that is admissible when working in
$\mathrm{R}_{\infty}$ need not be admissible when transferring the
problem to $\overline{\mathrm{R}}$.  Let
$\mathrm{X} = \{ \mathrm{x} \}$, $\mathrm{D} = [0, 1]$;
$\mu : [\mathrm{X} \to \mathrm{D}] \to \RZp$ with
$\mu(\{ \mathrm{x} \mapsto r \}) = r$ if $r \neq 0$, and
$\mu(\{ \mathrm{x} \mapsto 0 \}) = 1$; and
$\nu : [\mathrm{X} \to \mathrm{D}] \to \RZp$ with
$\nu(\{ \mathrm{x} \mapsto r \}) = 1$.  Then
$\mathsf{M} = \{ \mu, \nu \}$ is a finite set of
$(\mathrm{X}, \mathrm{D})$-$\mathrm{R}_{\infty}$-soft constraints,
which is also admissible, since $\mathsf{M}^* = \{ 1 \}$ and
$\mathsf{M}(v) = 1$ for all $v \in [\mathrm{X} \to \mathrm{D}]$.
However, for
$\mathsf{N} = \{ \tau \compfun \mu, \tau \compfun \nu \}$, which is a
finite set of $(\mathrm{X}, \mathrm{D})$-$\overline{\mathrm{R}}$-soft
constraints, we get $\mathsf{N}^* = \emptyset$, since each
$v_r = \{ \mathrm{x} \mapsto r \}$ with
$\mathsf{N}(v_r) = \lbag r, 1 \rbag$ can be improved to some
$v_{r/2} = \{ \mathrm{x} \mapsto r/2 \}$ with
$\mathsf{N}(v_{r/2}) = \lbag r/2, 1 \rbag$ and there is no
$v \in [\mathrm{X} \to \mathrm{D}]$ with
$\mathsf{N}(v) = \lbag 1 \rbag$.


\paragraph\label{par:max-p-norm}
As special cases of the equivalence of \eqref{eq:max-norm-1} and
\eqref{eq:max-norm-2} of §\ref{par:max-norm} for $0 \in V \subseteq
\RZp$, $p > 0$, and $\tau : V_{\infty} \to V_p$ with $\tau(r) = r$ we
obtain for each $n \geq 1$:
%
\begin{equation*}\tag{$\ast_p$}\label{eq:p-norm-1}\textstyle
  \prod_{\infty} \vec{r} < \prod_{\infty} \vec{s}
\quad\text{implies}\quad
  \prod_{p} \vec{r} < \prod_{p} \vec{s}
\quad\text{for all $\vec{r}, \vec{s} \in V^n$}
\end{equation*}
%
if, and only if,
%
\begin{equation}\tag{$\ast\ast_p$}\label{eq:p-norm-2}\textstyle
  r < s
\quad\text{implies}\quad
  n^{1/p} \cdot r < s
\quad\text{for all $r, s \in V.$}
\end{equation}

A $0 \in V \subseteq \RZp$ is \emph{$\delta$-separated} for some
$\delta > 1$ if $s/r \geq \delta$ for all $0 \neq r < s \in V$.  For
each $\delta$-separated $V$ and $n \geq 1$, \eqref{eq:p-norm-2} holds
if $p > \ln n/\ln \delta$, i.e.\ $n^{1/p} < \delta$: Let $r < s$ for
$r, s \in V$.  Then either $r = 0$, and thus $n^{1/p} \cdot r = 0 = r
< s$, or $r \neq 0$, and thus $n^{1/p} \cdot r < \delta \cdot r \leq
s$.  Conversely, if $0 \in V \subseteq \RZp$ for each $\delta > 1$
shows $0 \neq r < s \in V$ with $s/r < \delta$, then
\eqref{eq:p-norm-2} is violated for each $p > 0$: Let $p > 0$ be given
and choose $0 \neq r < s \in V$ such that $s/r < n^{1/p}$.  Then
$n^{1/p} \cdot r > s$.

\begin{example}\label{ex:separated}
(1)~Let $0 \in V \subseteq \RZp$ be finite. Then there is a
$\varepsilon > 0$ such that $|r_1 - r_2| \geq \varepsilon$ for all $r_1
\neq r_2 \in V$.  Let $0 \neq r < s \in V$.  Then $s/r \geq
(r+\varepsilon)/r = 1 + \varepsilon/r \geq 1 + \varepsilon/\max V$.
Thus $V$ is $(1 + \varepsilon/\max V)$-separated.

\smallskip%
(2)~Let $c \in \RZ$ with $c > 1$ and let $V^c = \{ c^n \mid n \in \NZ
\} \cup \{ 0 \}$.  If $0 \neq r < s \in V^c$, then there are $m < n$
with $r = c^m$ and $s = c^n$; then $c^n/c^m = c^{n-m} \geq c$.  Thus
$V^c$ is $c$-separated and unbounded.

\smallskip%
(3)~Let $d \in \RZ$ with $d > 1$ and let
$V^{1/d} = \{ d^{-n} \mid n \in \NZ \} \cup \{ 0 \}$.  If
$0 \neq r < s \in V^{1/d}$, then there are $m < n$ with $r = d^{-n}$
and $s = d^{-m}$.  Then $d^{-m}/d^{-n} = d^{n-m} \geq d$ holds.  In
addition, $0 < d^{-n} \leq d$ for all $n \in \NZ$. Hence $V^{1/d}$ is
$d$-separated and bounded.\qed
\end{example}

\paragraph
A \emph{weighting} for a finite set $C$ of $(X, D)$-constraints is
given by a function $g : C \times [X \to D] \to \RZp$ with $g(c, v) =
0$ if, and only if, $v \models c$ for $v \in [X \to D]$ and $c \in C$.
Given a real meet monoid $R$ and a weighting $g : C \times [X \to D]
\to |R|$, the $(R, g)$-\emph{weight} of a $v \in [X \to D]$ is given
by $W^g_R(v) = \prod_{R} \{ g(c, v) \mid c \in C \}$.  Several level
comparators can be defined in terms of weights:
%
\begin{itemize}[label={--},leftmargin=*]
  \item \emph{Weighted sum}: $W^g_{\mathrm{R}_1}(v) = \sum_{c \in C} g(c, v)$.

  \item\emph{Least squares}: $W^g_{\mathrm{R}_2}(v) = \sqrt{\sum_{c \in C} g(c, v)^2}$.

  \item\emph{Worst case}: $W^g_{\mathrm{R}_{\infty}}(v) = \max \{ g(c, v) \mid c \in C \}$.
\end{itemize}
%
In each case, 
%
\begin{equation*}
  w <^{W^g_R}_C v \iff W^g_R(w) > W^g_R(v)
\end{equation*}
%
for all $v, w \in [X \to D]$.

The cases of weighted sums and least squares use real meet monoids
without collapsing elements, and thus these are readily usable in
lexicographic products.  The worst case, however, involves
$\mathrm{R}_{\infty}$ where $\collapseset(\mathrm{R}_{\infty}) = \RZp
\setminus \{ 0 \}$. Assume for this case that $C$ has three different
constraints $c_1$, $c_2$, and $c_3$; that there are valuations $v_1$
violating only $c_1$, $v_2$ violating only $c_2$, $v_{13}$ violating
exactly $c_1$ and $c_3$, and $v_{23}$ violating exactly $c_2$ and
$c_3$; that $g(c, v) = 0$ if, and only if, $c \in \{ c_1, c_2, c_3 \}$
is satisfied by $v \in \{ v_1, v_2, v_{13}, v_{23} \}$; and that the
weightings are independent of the valuation, i.e., $g(c_1, v_1) =
g(c_1, v_{13})$ and $g(c_2, v_2) = g(c_2, v_{23})$ and $g(c_3, v_{13})
= g(c_3, v_{23})$.  Also assume that the level weightings for the
valuations $v_1$, $v_2$, $v_{13}$, and $v_{23}$ for the worst case are
related by
%
\begin{gather*}
  W^g_{\mathrm{R}_{\infty}}(v_1) = g(c_1, v_1) > g(c_2, v_2) = W^g_{\mathrm{R}_{\infty}}(v_2)
\ \text{,}
\\
  W^g_{\mathrm{R}_{\infty}}(v_{13}) = \max \{ g(c_1, v_{13}), g(c_3, v_{13}) \} = \max \{
  g(c_2, v_{23}), g(c_3, v_{23}) \} = W^g_{\mathrm{R}_{\infty}}(v_{23})
\ \text{.}
\end{gather*}
%
Any set of $(X, D)$-$M$-soft constraints
$\mathsf{M} = \{ \mu(c) \mid c \in C \}$ reflecting the ordering
induced by $W^g_{\mathrm{R}_{\infty}}$ on valuations, i.e.,
$\mathsf{M}(w) \leq_M \mathsf{M}(v) \iff W^g_{\mathrm{R}_{\infty}}(w)
\geq W^g_{\mathrm{R}_{\infty}}(v)$,
would thus have $\mu(c_3)$ as collapsing element in $M$.

\begin{proposition}
Let $(X, D)$ be a constraint domain, $0 \in V \subseteq \RZp$
$\delta$-separated, $\mathsf{M}_{\infty}$ an admissible set of
$(X, D)$-$V_{\infty}$-soft constraints, and
$p > \ln |\mathsf{M}_{\infty}|/\ln \delta$.  Define
$\tau_p : |V_{\infty}| \to |V_p|$ by $\tau_p(r) = r$ and the finite
set of $(X, D)$-$V_p$-soft constraints $\mathsf{M}_p$ by
$\mathsf{M}_p = \{ \tau_p \compfun \mu \mid \mu \in
\mathsf{M}_{\infty} \}$.
If $\mathsf{M}_p$ is admissible, then
$\mathsf{M}_{\infty} \preccurlyeq \mathsf{M}_{p}$.
\end{proposition}
\begin{proof}
The claim that $\mathsf{M}_{\infty} \preccurlyeq \mathsf{M}_{p}$
follows from Lem.~§\ref{lem:simulation} by the choice of $p$ and the
totality of the order in $V_{\infty}$.
\end{proof}

For a finite set of $(X, D)$-constraints $C$ and a weighting
$g : C \times [X \to D] \to \RZp$, let
$V_0 = \{ g(c, v) \mid c \in C,\ v \in [X \to D] \}$.  Assume that
$V_0 \cup \{ 0 \}$ is $\delta$-separated for some $\delta > 1$ and let
$p > \ln |C|/\ln \delta$.  For each $c \in C$, define the
$(X, D)$-$(V_0)_{\infty}$-soft constraint $c_{\infty}^g$ by
$c_{\infty}^g(v) = g(c, v)$ and the $(X, D)$-$(V_0)_{p}$-soft
constraint $c_p^g$ by $c_p^g(v) = g(c, v)$.  Then, since $C$ is
finite, we obtain
$\{ c_{\infty}^g \mid c \in C \} \preccurlyeq \{ c_p^g \mid c \in C
\}$,
provided that both $\{ c_{\infty}^g \mid c \in C \}$ and
$\{ c_p^g \mid c \in C \}$ are admissible.

% \paragraph 
% As an example of a collapsing meet monoid, we consider the real meet
% monoid which is obtained by using $\max$ as the combination operation.
% We look for another, collapse-free, meet monoid that simulates it.

% Let us start with the actual maximum meet monoid $L_{\max} = (M, \max, 0 \geq)$.
% Our prime requirement is to ask that $0 \in M \subseteq \mathbb{R}_{\geq 0}$.
% If $M = \mathbb{R}_{\geq 0}$, we have seen that $L_{\max}$ shows collapsing
% elements.

% We search for a simulating meet monoid $(\bar{M} \otimes, 0, \geq)$ with $0 \in \bar{M} \subseteq 
% \mathbb{R}_{\geq 0}$ with $\tau$ being the identity such that:
% for all $(v_1, \ldots, v_n)$, $(w_1, \ldots, w_n) \in M^n$ with $n = |\mathsf{M}|$.
% %
% \[
% \max\{v_1, \ldots, v_n\} < \max\{w_1, \ldots, w_n\} \quad \text{implies} \quad v_1 \otimes \ldots \otimes v_n < w_1 \otimes \ldots \otimes w_n
% \]
% %
% Let us consider a particular case, i.e., $\forall i \{1, \ldots, n\} v_i = v > 0$ and $w_i = 0$ if $i \neq n$ and $w_n = w$
% with $w > v$.  Since then the precondition of our requirements holds, i.e., $\max\{v\} = v 
% < \max\{0,w\} = w$ and we have to satisfy $v \otimes v \ldots v \otimes v < 0 \otimes \ldots \otimes 0 \otimes w = w$.
% Since $(\bar{M}, \otimes, 0, \geq)$ is required to be a meet monoid we have that
% $v \leq v \otimes v \leq v \otimes v \otimes v, \ldots$ \added{think of the big-kid-analogy}.

% Consequently, we have to at least satisfy $v \leq v \otimes v < w$.
% There are two cases
% \begin{enumerate}
% \item if $v = v \otimes v$, $v$ is a collapsing element since it is idempotent and we assumed $v > 0$.
% \item if $v < v \otimes v$ and $v \otimes v \in M$ and there exists an $m \in M$ with
% $v < m \leq v \otimes v$ then $w$ could be chosen to 
% be $m$ to make our original requirement unsatisfiable since then $m \leq v < m$ would have to hold.
% In particular, if $M = \mathbb{R}_{\geq 0}$ and $\tau = \lambda. x = x$, we get that there can be no meet monoid
% without collapsing elements simulating the meet monoid $(M, \max, 0, \geq)$.
% \end{enumerate}

% This last requirement leaves us with the chance of finding restrictions on $M$ to
% make our construction work. Indeed, let us call $M$ $\varepsilon$-separated if
% for all $m_1$, $m_2 \in M$, $0 \neq m_1 < m_2$ implies $\frac{m_2}{m_1} > 1 +
% \varepsilon$. We show that for such an $\varepsilon$-separated $M$, we can find a
% $p > 0$ such that $\otimes$ is realized by the $p$-norm.

% Consider our original goal for an arbitrary $p$:
% %
% \[
% \max\{v_1, \ldots v_n\} < \max\{w_1, \ldots w_n\} \Longrightarrow (\sum_{i=1}^{n} v_i^p)^{\frac{1}{p}} < 
% (\sum_{i=1}^{n} w_i^p)^{\frac{1}{p}}
% \]
% %
% Now, let us consider again the ``closest case'': $\forall i \in \{1, \ldots, n\} v_i = v$ and $w_i = 0$ for $i \neq n$
% and $w_n = w$. It is ``closest'' in the sense that it maximizes $(\sum_{i=1}^{n} v_i^p)^{\frac{1}{p}}$ 
% and minimizes $(\sum_{i=1}^{n} w_i^p)^{\frac{1}{p}}$; in other words, the distance of
% the two $p$-norm values is minimal.

% In that case, we are required to find a $p$ such that $n^{\frac{1}{p}} v < w$, for 
% arbitrary $v$ and $w$ out of $M$. If $v = 0$, any $p$ will do. Otherwise, $v > 0$; 
% then the requirement for $p$ is: $n^{\frac{1}{p}} < \frac{w}{v}$.
% Since we assumed that $\frac{w}{v} > 1 + \varepsilon$, as $w > v$, it suffices 
% to find $p > 0$ such that $n^{\frac{1}{p}} 
% < 1 + \varepsilon$. Any $p > \frac{\ln n}{\ln (1+\varepsilon)}$ will do.

% Note that any finite $M$ satisfies our $\varepsilon$-separatedness for suitable 
% $\varepsilon$. There exists $\delta > 0$ s.t. $\forall m_1 \neq m_2 |m_1 - m_2| > \delta$;
% Let $m_2 > m_1 \neq 0$, then $\frac{m_2}{m_1} > \frac{m_1 + \delta}{m_1} = 1 + 
% \frac{\delta}{m_1} \geq 1 + \frac{\delta}{\max M}$, i.e., $\varepsilon = 1 + \frac{\delta}
% {\max M}$.

% An example of an unbounded $\varepsilon$-separable $M$: Let $c \in \mathbb{R}_{> 1}$;
% consider $M_c = \{ c^n \mid n \in \mathbb{N} \} \cup \{0\}$: if $n > m$, then $c^n > c^m$
% and $\frac{c^n}{c^m} = c^{n-m} \geq c > 1$ and thus $c = 1 + \varepsilon$ for $\varepsilon
% = c - 1$.

\section{Constraint Relationships}
\label{sec:constraint-relationships}
\paragraph
A \emph{constraint relationship} over a constraint domain $(X, D)$, or
\emph{$(X, D)$-constraint relationship}, is given by a dag $C$, where
$|C|$ is a set of $(X, D)$-constraints.  We think of a constraint $c'
\in |C|$ as \emph{more important} than another constraint $c \in |C|$
if $c \rightarrow_C c'$.  An $(X, D)$-constraint relationship $C$ is
\emph{finite} if $|C|$ is finite.

% A \emph{constraint relationship} over a constraint domain $(X, D)$, or
% \emph{$(X, D)$-constraint relationship}, is given by a graph $(C, R)$,
% where $C$ is a set of $(X, D)$-constraints and $R \subseteq C \times
% C$ is acyclic.  We think of a constraint $c' \in C$ as \emph{more
%   important} than another constraint $c \in C$ if $c \mathrel{R} c'$.
% An $(X, D)$-constraint relationship $(C, R)$ is \emph{finite} if $C$
% is finite.

% For two sets $V, W \subseteq C$, which we think of being sets of
% violated constraints representing, e.g., $(X, D)$-valuations, we want
% to express that $W$ is \emph{worse} than $V$ w.r.t.\ $R$.  We consider
% three kinds of such liftings of the binary relation $R$ to subsets of
% $C$, represented by three \emph{dominance properties} $p$:
% \emph{single-predecessor} dominance ($p = \mathrm{SPD}$),
% \emph{direct-predecessors} dominance ($p = \mathrm{DPD}$), and
% \emph{transitive-predecessors} dominance ($p = \mathrm{TPD}$); we
% write $V \XPDrel{(C, R)} W$ for ``$W$ is worse than $V$ for dominance
% property $p$ over $R$''.  All three dominance properties share the
% worsening rule:
% %
% \bgroup\mathindent48pt
% \begin{align}
% \tag{W}\label{eq:W}
%    V &\XPDrel{(C, R)} V \uplus \{ c \} && \hspace*{-2.5cm}\text{if $c \in C$}\hspace*{1cm}
% \\
% \intertext{The remaining rules for $\mathrm{SPD}$, $\mathrm{DPD}$, and $\mathrm{TPD}$ are:}
% \tag{SPD}\label{eq:SPD}
%   V \uplus \{ c \} &\SPDrel{(C, R)} V \uplus \{ c' \} && \hspace*{-2.5cm}\text{if $c \mathrel{R} c'$}\hspace*{1cm}
% \\
% \tag{DPD}\label{eq:DPD}
%   V \uplus \{ c_1, \ldots, c_k \} &\DPDrel{(C, R)} V \uplus \{ c' \} && \hspace*{-2.5cm}\text{if $\forall c \in \{ c_1, \ldots, c_k \} \,.\, c \mathrel{R} c'$}\hspace*{1cm}
% \\
% \tag{TPD}\label{eq:TPD}
%   V \uplus \{ c_1, \ldots, c_k \} &\TPDrel{(C, R)} V \uplus \{ c' \} && \hspace*{-2.5cm}\text{if $\forall c \in \{ c_1, \ldots, c_k \} \,.\, c \mathrel{R^+} c'$}\hspace*{1cm}
% \end{align}
% \egroup

\paragraph
Let $C$ be an $(X, D)$-constraint relationship and let $M$ be a meet
monoid with a partial order homomorphism $\varphi : \POfree{C} \to
\POfun(M)$.  For each $c \in |C|$, define the $(X, D)$-$M$-soft
constraint $c_{M, \varphi} : [X \to D] \to |M|$ by
%
\begin{equation*}
  c_{M, \varphi}(v)
=
\begin{cases}
  \varphi(c) & \text{if $v \not\models c$}\\
  \varepsilon_M & \text{otherwise}
\end{cases}
\ \text{.}
\end{equation*}
%
We write $C_{M, \varphi}$ for the set of $(X, D)$-$M$-soft constraints
$\{ c_{M, \varphi} \mid c \in |C| \}$.

\begin{example}
Let $C$ be a finite constraint relationship over $(X, D)$.

\smallskip

(1)~We first consider the single-predecessor lifting introduced in
§\ref{par:spd}.

Let $M_C = \mMonfun(\jMonfree{\POfree{C}}) =
\mMonfree{\POfree{C}^{-1}}$ and define $m_C : \POfree{C} \to
\POfun(M_C)$ by $m_C(c) = \mMonetaat{\POfree{C}^{-1}}(c) = \lbag c
\rbag$; in particular $\varepsilon_{M_C} = \lbag \rbag$.  Then for $v,
w \in [X \to D]$
%
\begin{gather*}
\textstyle
  w \lesssim_{C_{M_C, m_C}} v
\iff{}\\\textstyle\qquad
  \prod_{M_C} \{ c_{M_C, m_C}(w) \mid c \in |C| \} \leq_{M_C} \prod_{M_C} \{ c_{M_C, m_C}(v) \mid c \in |C| \}
\iff{}\\\textstyle\qquad
  \lbag c \mid w \not\models c,\ c \in |C| \rbag \uppersubmseteq{\POfree{C}^{-1}} \lbag c \mid v \not\models c,\ c \in |C| \rbag
\iff{}\\\textstyle\qquad
  \lbag c \mid v \not\models c,\ c \in |C| \rbag \lowersubmseteq{\POfree{C}} \lbag c \mid w \not\models c,\ c \in |C| \rbag
\ \text{.}
\end{gather*}
%
Thus, $v$ is considered a better solution than $w$ if each constraint that is
not satisfied by $v$ can be paired off with a constraint that is not
satisfied by $w$ and which is more important.

\smallskip

(2)~We now consider the transitive-predecessors lifting introduced in
§\ref{par:tpd}.

Let $U_C = \mMonfun(\jMonfun(\uSLfree{\POfree{C}}))$
and define $u_C : \POfree{C} \to \POfun(U_C)$ by $u_C(c) = \{ c \}$;
in particular, $\varepsilon_{U_C} = \emptyset$.  Then for $v, w \in [X
\to D]$
%
\begin{gather*}
\textstyle
  w \lesssim_{C_{U_C, u_C}} v
\iff{}\\\textstyle\qquad
  \prod_{U_C} \{ c_{U_C, u_C}(w) \mid c \in |C| \} \leq_{U_C} \prod_{U_C} \{ c_{U_C, u_C}(v) \mid c \in |C| \}
\iff{}\\\textstyle\qquad
  \Max{\leq_{\POfree{C}}} \{ c \mid w \not\models c,\ c \in |C| \} \lowersupseteq{\POfree{C}} \Max{\leq_{\POfree{C}}} \{ c \mid v \not\models c,\ c \in |C| \}
\iff{}\\\textstyle\qquad
  \Max{\leq_{\POfree{C}}} \{ c \mid v \not\models c,\ c \in |C| \} \lowersubseteq{\POfree{C}} \Max{\leq_{\POfree{C}}} \{ c \mid w \not\models c,\ c \in |C| \}
\iff{}\\\textstyle\qquad
  \forall c \in \{ c_v \in |C| \mid v \not\models c_v \} \,.\, \exists c' \in \{ c_w \in |C| \mid w \not\models c_w \} \,.\, c \leq_{\POfree{C}} c'
\ \text{.}
\end{gather*}
%
Thus, $v$ is considered a better solution than $w$ if each constraint
that is not satisfied by $v$ can be covered by a constraint that is
not satisfied by $w$ and which is more important.\qed
\end{example}

\paragraph
The \emph{scope} of a constraint $c$ over a constraint domain $(X, D)$
is given by the set of variables it depends on, i.e.,
%
\begin{equation*}
  \scope(c) = \{ x \in X \mid \exists v \in [X \to D], d_1 \neq d_2 \in D_x \,.\, c(v\{ x \mapsto d_1 \}) \neq c(v\{ x \mapsto d_2 \}) \}
\ \text{.}
\end{equation*}
%
For a partial valuation $p \in [X \to D^{?}]$, we write $p \not\models
c$ if $\scope(c) \subseteq \defdom(p)$ and $v \not\models c$ for some
$v \in p{\uparrow}$ (which is well-defined, since then $c$ only
depends on variables that are in the domain of definition of $p$).

Let $C$ be a finite constraint relationship over $(X, D)$, let $M$ be
a meet monoid, and let $\varphi : \POfree{C} \to \POfun(M)$.  Define
$\alpha_{M, \varphi}, \zeta_{M, \varphi} : [X \to D^{?}] \to |M|$ by
%
\begin{gather*}
\textstyle
  \alpha_{M, \varphi}(p) = \prod_{M} \{ \varphi(c) \mid c \in |C|,\ \scope(c) \subseteq \defdom(p),\ p \not\models c \} \cdot_M \prod_{M} \{ \varphi(c) \mid \scope(c) \not\subseteq \defdom(p) \}
\ \text{,}
\\
\textstyle
  \zeta_{M, \varphi}(p) = \prod_{M} \{ \varphi(c) \mid c \in |C|,\ \scope(c) \subseteq \defdom(p),\ p \not\models c \}
\ \text{.}
\end{gather*}

\begin{lemma}
$(\alpha_{M, \varphi}, \zeta_{M, \varphi})$ is a tight bounding pair
of $(X, D^{?})$-$M$-soft constraints for $C_{M, \varphi}$.
\end{lemma}
\begin{proof}
For a $p \in [X \to D^{?}]$ let $V(p) = \{ c \in |C| \mid \scope(c)
\subseteq \defdom(p),\ p \not\models c \}$ and $W(p) = \{ c \in |C|
\mid \scope(c) \not\subseteq \defdom(p) \}$.  Then $\alpha_{M,
  \varphi}(p) = \prod_M \varphi(V(p)) \cdot_M \prod_M \varphi(W(p))$
and $\zeta_{M, \varphi}(p) = \prod_M \varphi(V(p))$ for all $p \in [X
\to D^{?}]$.

Let $p \in [X \to D^{?}]$ and $v \in p{\uparrow}$ be given.  Then
$V(p) \subseteq V(v)$, and thus $\zeta_{M, \varphi}(v) \leq_M
\zeta_{M, \varphi}(p)$: For a $c \in V(p)$, i.e., $\scope(c) \subseteq
\defdom(p)$ and $p \not\models c$, also $c \in V(v)$, since $v \in
p{\uparrow}$ and thus $v \not\models c$.

Now let $p' = p\{ x \mapsto d \} \in [X \to D^{?}]$ and let $v \in
p{\uparrow}$ be arbitrary.  Then $V(v) \subseteq V(p') \cup W(p')$,
and thus $\alpha_{M, \varphi}(p') \leq_M \zeta_{M, \varphi}(v)$: Let
$c \in V(v)$, i.e., $v \not\models c$.  If $\scope(c) \subseteq
\defdom(p')$, then $p' \not\models c$, and hence $c \in V(p')$;
otherwise $c \in W(p')$.

Finally, $\zeta_{M, \varphi}(v) = C_{M, \varphi}(v)$ and thus
$(\alpha_{M, \varphi}, \zeta_{M, \varphi})$ is tight for $C_{M, \varphi}$.
\end{proof}

\begin{example}
Consider the constraint domain $(\mathrm{X}, \mathrm{D})$ given by
%
\begin{gather*}
  \mathrm{X} = \{ \mathrm{x}, \mathrm{y}, \mathrm{z} \}
\ \text{,}\\
  \mathrm{D}_{\mathrm{x}} = \mathrm{D}_{\mathrm{y}} = \mathrm{D}_{\mathrm{z}} = \{ 1, 2, 3 \}
\end{gather*}
%
as well as the constraint relationship $\mathrm{C} = (\{ \mathrm{c}_1,
\mathrm{c}_2, \mathrm{c}_3 \}, \{ (\mathrm{c}_2, \mathrm{c}_1),
(\mathrm{c}_3, \mathrm{c}_1) \})$, i.e., $\mathrm{c}_2
\rightarrow_{\mathrm{C}} \mathrm{c}_1$ and $\mathrm{c}_3
\rightarrow_{\mathrm{C}} \mathrm{c}_1$, with
%
\begin{gather*}
  \mathrm{c}_1 : \mathrm{x} + 1 = \mathrm{y}
\ \text{,}\\
  \mathrm{c}_2 : \mathrm{z} = \mathrm{y} + 2
\ \text{,}\\
  \mathrm{c}_3 : \mathrm{x} + \mathrm{y} \leq 3
\ \text{.}
\end{gather*}

Let $M_{\mathrm{C}} = \mMonfree{\POfree{\mathrm{C}}^{-1}}$ and $m_{\mathrm{C}}
= \mMonetaat{\POfree{\mathrm{C}}^{-1}}$.  Then
%
\begin{gather*}
  \alpha_{M_{\mathrm{C}}, m_{\mathrm{C}}}(p) = \lbag c \in |\mathrm{C}| \mid \scope(c) \subseteq \defdom(p),\ p \not\models c \rbag \mcup \lbag c \in |\mathrm{C}| \mid \scope(c) \not\subseteq \defdom(p) \rbag
\ \text{,}
\\
  \zeta_{M_{\mathrm{C}}, m_{\mathrm{C}}}(p) = \lbag c \in |\mathrm{C}| \mid \scope(c) \subseteq \defdom(p),\ p \not\models c \rbag
\ \text{,}
\end{gather*}
such that, for example,
%
\begin{equation*}
\renewcommand{\arraystretch}{1.2}
\begin{array}[t]{@{}l@{\qquad}l@{}}
  \alpha_{M_{\mathrm{C}}, m_{\mathrm{C}}}(\{ \mathrm{x} \mapsto 1, \mathrm{y} \mapsto 1, \mathrm{z} \mapsto {?} \}) = \lbag \mathrm{c}_1, \mathrm{c}_2 \rbag
\ \text{,}
&
  \zeta_{M_{\mathrm{C}}, m_{\mathrm{C}}}(\{ \mathrm{x} \mapsto 1, \mathrm{y} \mapsto 1, \mathrm{z} \mapsto {?} \}) = \lbag \mathrm{c}_1 \rbag
\ \text{,}
\\
  \alpha_{M_{\mathrm{C}}, m_{\mathrm{C}}}(\{ \mathrm{x} \mapsto 1, \mathrm{y} \mapsto 2, \mathrm{z} \mapsto {?} \}) = \lbag \mathrm{c}_2 \rbag
\ \text{,}
&
  \zeta_{M_{\mathrm{C}}, m_{\mathrm{C}}}(\{ \mathrm{x} \mapsto 1, \mathrm{y} \mapsto 2, \mathrm{z} \mapsto {?} \}) = \lbag \rbag
\ \text{,}
\\
  \alpha_{M_{\mathrm{C}}, m_{\mathrm{C}}}(\{ \mathrm{x} \mapsto 2, \mathrm{y} \mapsto 3, \mathrm{z} \mapsto {?} \}) = \lbag \mathrm{c}_2, \mathrm{c}_3 \rbag
\ \text{,}
&
  \zeta_{M_{\mathrm{C}}, m_{\mathrm{C}}}(\{ \mathrm{x} \mapsto 2, \mathrm{y} \mapsto 3, \mathrm{z} \mapsto {?} \}) = \lbag \mathrm{c}_3 \rbag
\ \text{;}
\end{array}
\end{equation*}
%
in particular
%
\begin{equation*}
  \zeta_{M_{\mathrm{C}}, m_{\mathrm{C}}}(\{ \mathrm{x} \mapsto 1, \mathrm{y} \mapsto 1, \mathrm{z} \mapsto {?} \})
=
  \lbag \mathrm{c}_1 \rbag <_{M_{\mathrm{C}}} \lbag \mathrm{c}_2 \rbag
=
  \alpha_{M_{\mathrm{C}}, m_{\mathrm{C}}}(\{ \mathrm{x} \mapsto 1, \mathrm{y} \mapsto 2, \mathrm{z} \mapsto {?} \})
\ \text{.}
\end{equation*}

We abbreviate $\alpha_{M_{\mathrm{C}}, m_{\mathrm{C}}}$ by $\alpha$
and $\zeta_{M_{\mathrm{C}}, m_{\mathrm{C}}}$ by $\zeta$.  We follow an
execution of $\macit{maxSolDegs}_{(\alpha, \zeta)}(\{ \mathrm{x}
\mapsto {?}, \mathrm{y} \mapsto {?}, \mathrm{z} \mapsto{?} \},
\emptyset)$, choosing the variables $\mathrm{x}$, $\mathrm{y}$, and
$\mathrm{z}$ in this order and running through $\{ 1, 2, 3 \}$ in the
natural order; we select $\mathrm{x}$ and $\mathrm{y}$ first, since
$\scope(\mathrm{c}_1) = \{ \mathrm{x}, \mathrm{y} \}$ and
$\mathrm{c}_1$ is the top element in $\mathrm{C}$.

The first step in
evaluating $\macit{maxSolDegs}_{(\alpha, \zeta)}(\lambda x \in
\mathrm{X} \,.\, {?}, \emptyset)$ is to evaluate
$\macit{maxSolDegs}_{(\alpha, \zeta)}(\{ \mathrm{x} \mapsto 1,
\mathrm{y} \mapsto {?}, \mathrm{z} \mapsto {?}  \}, \lbag
\mathrm{c}_1, \mathrm{c}_2, \mathrm{c}_3 \rbag)$, since
%
\begin{equation*}
  \alpha((\lambda x \in \mathrm{X} \,.\, {?})\{ x \mapsto d \})
=
  \lbag \mathrm{c}_1, \mathrm{c}_2, \mathrm{c}_3 \rbag
\ \text{,}
\end{equation*}
%
for all $d \in \{ 1, 2, 3 \}$ and $\zeta((\lambda x \,.\,
\mathrm{X})\{ x \mapsto 1 \}) = \lbag \rbag$; which leads to the
following graph:
%
\begin{equation*}
\begin{tikzpicture}[scale=1,auto]
\tikzstyle{state}=[rectangle,draw,inner sep=2pt,minimum size=6pt,align=center,font={\fontsize{7pt}{7pt}\selectfont}]
\tikzstyle{transition}=[draw,-latex',font={\fontsize{7pt}{7pt}\selectfont}]
\tikzstyle{annotation}=[outer sep=0pt,inner sep=1pt,font={\fontsize{7pt}{7pt}\selectfont}]
\node[state,double] (1nn) {$1, {?}, {?}$\\ $\lbag \mathrm{c}_1, \mathrm{c}_2, \mathrm{c}_3 \rbag$\\ $\lbag \rbag$}
  [-latex',sibling distance=5.5cm,level distance=1.7cm]
  child {
    node[state] (11n) {$1, 1, {?}$\\ $\lbag \mathrm{c}_1, \mathrm{c}_2 \rbag$\\ $\lbag \mathrm{c}_1 \rbag$}
    edge from parent[draw=none]
  }
  child {
    node[state,double] (12n) {$1, 2, {?}$\\ $\lbag \mathrm{c}_2 \rbag$\\ $\lbag \rbag$}
      [sibling distance=1.8cm,level distance=1.7cm]
    child {
      node[state] (121) {$1, 2, 1$\\ $\lbag \mathrm{c}_2 \rbag$\\ $\lbag \mathrm{c}_2 \rbag$}
      edge from parent[draw=none]
    }
    child {
      node[state] (122) {$1, 2, 2$\\ $\lbag \mathrm{c}_2 \rbag$\\ $\lbag \mathrm{c}_2 \rbag$}
      edge from parent[draw=none]
    }
    child {
      node[state] (123) {$1, 2, 3$\\ $\lbag \mathrm{c}_2 \rbag$\\ $\lbag \mathrm{c}_2 \rbag$}
      edge from parent[draw=none]
    }
    edge from parent[draw=none]
  }
  child {
    node[state] (13n) {$1, 3, {?}$\\ $\lbag \mathrm{c}_1, \mathrm{c}_2, \mathrm{c}_3 \rbag$\\ $\lbag \mathrm{c}_1, \mathrm{c}_3 \rbag$}
    edge from parent[draw=none]
  };
\path[transition] ($ (1nn.north) + (0, .5) $) -- node[annotation,anchor=east,xshift=-1pt] {$\lbag \mathrm{c}_1, \mathrm{c}_2, \mathrm{c}_3 \rbag$} (1nn.north) ;
%
\path[transition] (1nn) to node[annotation,anchor=south east] (1nn11n) {$\lbag \mathrm{c}_2 \rbag$} (11n) ;
\path[transition] (1nn) to node[annotation,anchor=east,xshift=-1pt] (1nn12n) {$\lbag \mathrm{c}_2 \rbag$} (12n) ;
\path[transition] (1nn) to node[annotation,anchor=south west,xshift=1pt] (1nn13n) {$\lbag \mathrm{c}_2 \rbag$} (13n) ;
%
\path[transition] (12n) to[bend right=15] node[annotation,anchor=south east] (12n121) {$\lbag \mathrm{c}_2 \rbag$} (121) ;
\path[transition] (12n) to node[annotation,anchor=east] (12n122) {$\lbag \mathrm{c}_2 \rbag$} (122) ;
\path[transition] (12n) to[bend left=15] node[annotation,anchor=south west] (12n123) {$\lbag \mathrm{c}_2 \rbag$} (123) ;
%
\path[transition,dashed] (12n) to[bend left=15] (1nn11n) ;
\path[transition,dashed] (1nn11n) to[bend right=15] (1nn12n) ;
\path[transition,dashed] (12n121) to[bend right=15] (12n122) ;
\path[transition,dashed] (12n122) to[bend right=15] (12n123) ;
\path[transition,dashed] (12n123) to[bend right=10] ($ (12n.north east) + (.1, -.2) $) ;
\path[transition,dashed] (12n) to[bend left=10] (1nn13n) ;
%
\path[transition,dashed] (1nn13n) to[bend right=10] ($ (1nn.north east) + (.1, -.2) $) ;
\end{tikzpicture}
\end{equation*}
%
The annotations on the solid edges represent the current values of the
lower bound set $L$, where we have omitted the set braces.  Each node
gives the respective partial valuation $p$ for $\mathrm{x}$,
$\mathrm{y}$, and $\mathrm{z}$ at the top, $\alpha(p)$ in the middle,
and $\zeta(p)$ at the bottom.  Doubly outlined nodes represent calls
to $\macit{maxSolDegs}_{(\alpha, \zeta)}(p, L)$; singly outlined nodes
represent the successful test whether $\zeta(p)$ already is dominated
by a lower bound in $L$.  Finally, the dashed edges show the flow of
the lower bounds.

Thus, $\macit{maxSolDegs}_{(\alpha, \zeta)}(\{ \mathrm{x} \mapsto 1, \mathrm{y}
\mapsto {?}, \mathrm{z} \mapsto {?}  \}, \emptyset) = \{ \lbag
\mathrm{c}_2 \rbag \}$, and $\macit{maxSolDegs}_{(\alpha, \zeta)}(\{ \mathrm{x} \mapsto 2, \mathrm{y}
\mapsto {?}, \mathrm{z} \mapsto {?}  \}, \{ \lbag
\mathrm{c}_2 \rbag \})$ is executed:
%
\begin{equation*}
\begin{tikzpicture}[scale=1,auto]
\tikzstyle{state}=[rectangle,draw,inner sep=2pt,minimum size=6pt,align=center,font={\fontsize{7pt}{7pt}\selectfont}]
\tikzstyle{transition}=[draw,-latex',font={\fontsize{7pt}{7pt}\selectfont}]
\tikzstyle{annotation}=[outer sep=0pt,inner sep=1pt,font={\fontsize{7pt}{7pt}\selectfont}]
\node[state,double] (2nn) {$2, {?}, {?}$\\ $\lbag \mathrm{c}_1, \mathrm{c}_2, \mathrm{c}_3 \rbag$\\ $\lbag \rbag$}
  [-latex',sibling distance=5.5cm,level distance=1.7cm]
  child {
    node[state] (21n) {$2, 1, {?}$\\ $\lbag \mathrm{c}_1, \mathrm{c}_2 \rbag$\\ $\lbag \mathrm{c}_1 \rbag$}
    edge from parent[draw=none]
  }
  child {
    node[state] (22n) {$2, 2, {?}$\\ $\lbag \mathrm{c}_1, \mathrm{c}_2, \mathrm{c}_3 \rbag$\\ $\lbag \mathrm{c}_1, \mathrm{c}_3 \rbag$}
    edge from parent[draw=none]
  }
  child {
    node[state,double] (23n) {$2, 3, {?}$\\ $\lbag \mathrm{c}_2, \mathrm{c}_3 \rbag$\\ $\lbag \mathrm{c}_3 \rbag$}
      [sibling distance=1.8cm,level distance=1.7cm]
    child {
      node[state] (231) {$2, 3, 1$\\ $\lbag \mathrm{c}_2, \mathrm{c}_3 \rbag$\\ $\lbag \mathrm{c}_2, \mathrm{c}_3 \rbag$}
      edge from parent[draw=none]
    }
    child {
      node[state] (232) {$2, 3, 2$\\ $\lbag \mathrm{c}_2, \mathrm{c}_3 \rbag$\\ $\lbag \mathrm{c}_2, \mathrm{c}_3 \rbag$}
      edge from parent[draw=none]
    }
    child {
      node[state] (233) {$2, 3, 3$\\ $\lbag \mathrm{c}_2, \mathrm{c}_3 \rbag$\\ $\lbag \mathrm{c}_2, \mathrm{c}_3 \rbag$}
      edge from parent[draw=none]
    }
    edge from parent[draw=none]
  };
\path[transition] ($ (2nn.north) + (0, .5) $) -- node[annotation,anchor=east,xshift=-1pt] {$\lbag \mathrm{c}_2 \rbag$} (2nn.north) ;
%
\path[transition] (2nn) to node[annotation,anchor=south east] (2nn21n) {$\lbag \mathrm{c}_2 \rbag$} (21n) ;
\path[transition] (2nn) to node[annotation,anchor=east,xshift=-1pt] (2nn22n) {$\lbag \mathrm{c}_2 \rbag$} (22n) ;
\path[transition] (2nn) to node[annotation,anchor=south west,xshift=1pt] (2nn23n) {$\lbag \mathrm{c}_2 \rbag$} (23n) ;
%
\path[transition,dashed] (2nn21n) to[bend right=15] (2nn22n) ;
\path[transition,dashed] (2nn22n) to[bend right=15] (2nn23n) ;
%
\path[transition] (23n) to[bend right=15] node[annotation,anchor=south east] (23n231) {$\lbag \mathrm{c}_2 \rbag$} (231) ;
\path[transition] (23n) to node[annotation,anchor=east] (23n232) {$\lbag \mathrm{c}_2 \rbag$} (232) ;
\path[transition] (23n) to[bend left=15] node[annotation,anchor=south west] (23n233) {$\lbag \mathrm{c}_2 \rbag$} (233) ;
\path[transition,dashed] (23n231) to[bend right=15] (23n232) ;
\path[transition,dashed] (23n232) to[bend right=15] (23n233) ;
\path[transition,dashed] (23n233) to[bend right=10] ($ (23n.north east) + (.1, -.2) $) ;
\path[transition,dashed] (23n.north) to[bend right=10] ($ (2nn.north east) + (.1, -.2) $) ;
% \path[transition,dashed] (12n123) to[bend left=10] (1nn13n) ;
% %
% \path[transition,dashed] (1nn13n) to[bend left=10] ($ (1nn.north) + (.3, .5) $) ;
\end{tikzpicture}
\end{equation*}

Hence, $\macit{maxSolDegs}_{(\alpha, \zeta)}(\{ \mathrm{x} \mapsto 2,
\mathrm{y} \mapsto {?}, \mathrm{z} \mapsto {?}  \}, \{ \lbag
\mathrm{c}_2 \rbag \}) = \{ \lbag \mathrm{c}_2 \rbag \}$ and
$\macit{maxSolDegs}_{(\alpha, \zeta)}(\{ \mathrm{x} \mapsto 3, \mathrm{y}
\mapsto {?}, \mathrm{z} \mapsto {?}  \}, \{ \lbag \mathrm{c}_2 \rbag
\})$ is executed:
%
\begin{equation*}
\begin{tikzpicture}[scale=1,auto]
\tikzstyle{state}=[rectangle,draw,inner sep=2pt,minimum size=6pt,align=center,font={\fontsize{7pt}{7pt}\selectfont}]
\tikzstyle{transition}=[draw,-latex',font={\fontsize{7pt}{7pt}\selectfont}]
\tikzstyle{annotation}=[outer sep=0pt,inner sep=1pt,font={\fontsize{7pt}{7pt}\selectfont}]
\node[state,double] (3nn) {$3, {?}, {?}$\\ $\lbag \mathrm{c}_1, \mathrm{c}_2, \mathrm{c}_3 \rbag$\\ $\lbag \rbag$}
  [-latex',sibling distance=5.5cm,level distance=1.7cm]
  child {
    node[state] (31n) {$3, 1, {?}$\\ $\lbag \mathrm{c}_1, \mathrm{c}_2, \mathrm{c}_3 \rbag$\\ $\lbag \mathrm{c}_1, \mathrm{c}_3 \rbag$}
    edge from parent[draw=none]
  }
  child {
    node[state] (32n) {$3, 2, {?}$\\ $\lbag \mathrm{c}_1, \mathrm{c}_2, \mathrm{c}_3 \rbag$\\ $\lbag \mathrm{c}_1, \mathrm{c}_3 \rbag$}
    edge from parent[draw=none]
  }
  child {
    node[state] (33n) {$3, 3, {?}$\\ $\lbag \mathrm{c}_1, \mathrm{c}_2, \mathrm{c}_3 \rbag$\\ $\lbag \mathrm{c}_1, \mathrm{c}_3 \rbag$}
    edge from parent[draw=none]
  };
\path[transition] ($ (3nn.north) + (0, .5) $) -- node[annotation,anchor=east,xshift=-1pt] {$\lbag \mathrm{c}_2 \rbag$} (3nn.north) ;
%
\path[transition] (3nn) to node[annotation,anchor=south east] (3nn31n) {$\lbag \mathrm{c}_2 \rbag$} (31n) ;
\path[transition] (3nn) to node[annotation,anchor=east,xshift=-1pt] (3nn32n) {$\lbag \mathrm{c}_2 \rbag$} (32n) ;
\path[transition] (3nn) to node[annotation,anchor=south west,xshift=1pt] (3nn33n) {$\lbag \mathrm{c}_2 \rbag$} (33n) ;
%
\path[transition,dashed] (3nn31n) to[bend right=15] (3nn32n) ;
\path[transition,dashed] (3nn32n) to[bend right=15] (3nn33n) ;
%
\path[transition,dashed] (3nn33n) to[bend right=10] ($ (3nn.north east) + (.1, -.2) $) ;
\end{tikzpicture}
\end{equation*}

Therefore, $\macit{maxSolDegs}_{(\alpha, \zeta)}(\{ \mathrm{x} \mapsto
3, \mathrm{y} \mapsto {?}, \mathrm{z} \mapsto {?}  \}, \{ \lbag
\mathrm{c}_2 \rbag \}) = \{ \lbag \mathrm{c}_2 \rbag \}$, and we have
as the final result that $\macit{maxSolDegs}_{(\alpha, \zeta)}(\{
\mathrm{x} \mapsto ?, \mathrm{y} \mapsto {?}, \mathrm{z} \mapsto {?}
\}, \emptyset) = \{ \lbag \mathrm{c}_2 \rbag \}$.\qed
\end{example}

%\bibliographystyle{plain}
%\bibliography{c-semirings}
\printbibliography

\end{document}


%%% Local Variables:
%%% mode: LaTeX
%%% mode: TeX-PDF
%%% mode: TeX-source-correlate
%%% TeX-master: t
%%% mode: flyspell
%%% ispell-local-dictionary: "british"
%%% End:

%  LocalWords:  dag dags
